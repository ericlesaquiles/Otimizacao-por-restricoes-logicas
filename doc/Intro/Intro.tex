\documentclass{article}
\usepackage[utf8]{inputenc}
\usepackage[portuguese]{babel}
\usepackage[utf8]{inputenc}
\usepackage{placeins}

\usepackage{amsthm}
\usepackage[ruled,vlined,linesnumbered]{algorithm2e}

\usepackage{listings}
\lstset{language=Prolog}
\renewcommand{\lstlistingname}{Código}

%\theoremstyle{definition}{indent=1cm}
\newtheorem{definition}{Definição}[section]

\theoremstyle{remark}
\newtheorem*{remark}{Remark}

\usepackage[ruled,vlined,linesnumbered]{algorithm2e}

\setlength{\parskip}{.5em}

\usepackage{epigraph}
\setlength\epigraphwidth{0.9\linewidth}
\setlength\epigraphrule{0pt}



%
% TODO: revisar terminar esta seção!
%
%


\begin{document}

\section{Introdução}

\subsection{Motivação}

\subsection{Notação}
  Ao decorrer do presente documento, usaremos as seguintes convenções:

  \begin{itemize}
    \item Palavras em língua estrangeira estarão em itálico, com exceções explícitamente notadas;
    \item Na ocasião da primeira utilização de algum termo (exceto se o termo estiver em algum programa ou trecho de código), o termo estará em negrito;
    \item A eventual definição de algum termo ou expressão terá o termo ou definição em negrito na margem da página\marginpar{\textbf{Assim}}, para facilitar sua localização;
    \item Quando se quiser destacar algum termo por qualquer outro motivo, ele estará em itálico.
    \item Para trechos de código, ou similar, escritos no meio do texto, será usada {\tt esta fonte}.
  \end{itemize}

\end{document}
