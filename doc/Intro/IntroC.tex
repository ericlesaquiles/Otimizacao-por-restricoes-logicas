%\documentclass{article}

%\usepackage[utf8]{inputenc}
\usepackage{placeins}
\usepackage{graphicx}
\usepackage[ruled,vlined,linesnumbered]{algorithm2e}
\usepackage{listings}
\usepackage{amsthm}
\usepackage{amsmath}
\usepackage{caption}
\usepackage{epigraph}

\captionsetup[figure]{font=scriptsize}

\newtheorem{definition}{Definição}[section]
\theoremstyle{remark}
\newtheorem*{remark}{Observações}
\theoremstyle{theorem}
\newtheorem{theorem}{Teorema}[section]

\setlength{\parskip}{.5em}

\renewcommand\refname{Leituras adicionais}
\renewcommand\figurename{Figura}
\renewcommand{\lstlistingname}{Código}

\lstdefinestyle{prosty}{
  language=Prolog,
  basicstyle=\small
}





%
% TODO: revisar terminar esta seção!
%
%


%\begin{document}

\section*{Introdução}

\subsection*{Motivação}

``Optimization and constraint programming are beginning to converge,  despite  their very
di erent origins.  Optimization is primarily associated with mathematics and engineering,
while constraint programming developed much more recently in the computer science and
artificial intelligence communities.  The two  fields evolved more or less independently until
a very few years ago.   Yet they have much in common and are applied to many of the
same problems.  Both have enjoyed considerable commercial success.  Most importantly for
present purposes,  they have complementary  strengths,  and the last few years have  seen
growing efforts to combine them.
Constraint programming, for example, offers a more flexible modeling framework than
mathematical programming. It not only permits more succinct models, but the models allow
one to exploit problem structure and direct the search. It relies on such logic-based methods
as domain reduction and constraint propagation to accelerate the search.  Conversely, opti-
mization brings to the table a number of specialized techniques for highly structured problem
classes, such as linear programming problems or matching problems. It also provides a wide
range of relaxations, which are often indispensable for proving optimality.'' -- J.N.Hooker

\subsection*{Notação}
  Ao decorrer do presente documento, usaremos as seguintes convenções:

  \begin{itemize}
    \item Palavras em língua estrangeira estarão em itálico, com exceções explícitamente notadas;
    \item Na ocasião da primeira utilização de algum termo (exceto se o termo estiver em algum programa ou trecho de código), o termo estará em negrito;
    \item A eventual definição de algum termo ou expressão terá o termo ou definição em negrito na
      margem da página, como assim \marginpar{\textbf{Assim}}, para facilitar sua localização;
    \item Quando se quiser destacar algum termo por qualquer outro motivo, ele estará em itálico.
    \item Para trechos de código (ou algo similar) escritos no meio do texto, será usada {\tt esta fonte}.
  \end{itemize}

\subsection*{Documentação de códigos}

  As observações nesta subseção servem mais como referência e seriam melhor apreciadas depois de
  lidos os primeiros capítulos (em particular, o capítulo inicial).

  Uma particularidade do Prolog é que, no caso geral, a mesma variável pode ser tanto ``de entrada''
  quanto ``de saída'' e, como argumento de um funtor, pode ser tanto instanciada como não
  instanciada. No entanto, frequentemente um programa é otimizado ou construído para apenas um dos
  casos. Isto é, frequentemente espera-se que uma variável já esteja instanciada ao ser associada a
  um funtor (ou o contrário, espera-se que ela não esteja instanciada), ou que uma variável só sirva
  como variável de saída.

  Dada a forma como programas lógicos são escritos, esses casos podem não ser claros e deseja-se que
  sejam documentados. Para tanto, usaremos a convenção de, na documentação do funtor ou restrição (geralmente, na
  forma de um comentário logo antes), indicar os seguintes modos:

  \begin{itemize}
    \item ``Variáveis de entrada'', isto é, unificadas a outro termo do programa, serão indicadas
      por um prefixo ``+'' (como em \codigo{f(+X)});
    \item Exceto quando são esperadas estarem unificadas com termos base, quando serão indicadas
      pelo prefixo ``++'' (como em \codigo{f(++X)});
    \item ``Variáveis de saída'', isto é, cuja expectativa é de não estarem unificadas a termo
      algum, serão indicadas pelo prefixo ``-'' (como em \codigo{f(-X)});
    \item Quando não for necessária uma distinção entre ``variável de entrada'' ou ``de saída'', a
      variável será indicada pelo prefixo ``?'', como o \var{Y} em \codigo{f(++X, ?Y, -Z)}.
  \end{itemize}


%\end{document}
