%\documentclass{article}

%\usepackage[utf8]{inputenc}
\usepackage{placeins}
\usepackage{graphicx}
\usepackage[ruled,vlined,linesnumbered]{algorithm2e}
\usepackage{listings}
\usepackage{amsthm}
\usepackage{amsmath}
\usepackage{caption}
\usepackage{epigraph}

\captionsetup[figure]{font=scriptsize}

\newtheorem{definition}{Definição}[section]
\theoremstyle{remark}
\newtheorem*{remark}{Observações}
\theoremstyle{theorem}
\newtheorem{theorem}{Teorema}[section]

\setlength{\parskip}{.5em}

\renewcommand\refname{Leituras adicionais}
\renewcommand\figurename{Figura}
\renewcommand{\lstlistingname}{Código}

\lstdefinestyle{prosty}{
  language=Prolog,
  basicstyle=\small
}




%\begin{document}

% \setcounter{tocdepth}{1} % Show sections
% \setcounter{tocdepth}{2} % + subsections
% \setcounter{tocdepth}{3} % + subsubsections
% \setcounter{tocdepth}{4} % + paragraphs
\setcounter{tocdepth}{5} % + subparagraphs

\tableofcontents

\section*{Introdução}

\subsection*{Motivação}

``Optimization and constraint programming are beginning to converge,
despite their very different origins. Optimization is primarily
associated with mathematics and engineering, while constraint
programming developed much more recently in the computer science and
artificial intelligence communities. The two fields evolved more or
less independently until a very few years ago. Yet they have much in
common and are applied to many of the same problems. Both have enjoyed
considerable commercial success. Most importantly for present
purposes, they have complementary strengths, and the last few years
have seen growing efforts to combine them. Constraint programming, for
example, offers a more flexible modeling framework than mathematical
programming. It not only permits more succinct models, but the models
allow one to exploit problem structure and direct the search. It
relies on such logic-based methods as domain reduction and constraint
propagation to accelerate the search.  Conversely, optimization brings
to the table a number of specialized techniques for highly structured
problem classes, such as linear programming problems or matching
problems. It also provides a wide range of relaxations, which are
often indispensable for proving optimality.'' -- J.N.Hooker

\subsection*{Notação}
  Ao decorrer do presente texto, usaremos as seguintes convenções:

  \begin{itemize}
    \item Palavras em língua estrangeira estarão \foreign{like this},
      exceto quando explicitamente dito o contrário;
    % \item Na ocasião da primeira utilização de algum termo (exceto se
    %   o termo estiver em algum programa ou trecho de código), o termo
    %   estará \definicao{assim};
    \item A eventual definição de algum termo ou expressão terá o
      termo ou definição \definicao{assim}, com o objetivo de
      facilitar sua localização;
    \item Quando quisermos destacar algum termo por qualquer outro
      motivo, ele estará \enphasis{assim};
    \item Para trechos de código escritos no meio do texto, será usada
      \codigo{esta fonte}.
  \end{itemize}

\subsection*{Documentação de códigos}

As observações nesta subseção servem mais como referência e seriam
melhor apreciadas depois de lidos os primeiros capítulos (em
particular, o capítulo inicial).

Uma particularidade do Prolog é que, no caso geral, a mesma variável
pode ser tanto ``de entrada'' quanto ``de saída'' e, como argumento de
um funtor, pode ser tanto instanciada como não instanciada. No
entanto, com alguma frequência, um programa é otimizado ou construído
para apenas um dos casos. Isto é, frequentemente espera-se que uma
variável já esteja instanciada ao ser associada a um funtor (ou o
contrário, espera-se que ela não esteja instanciada), ou que uma
variável só sirva como variável de saída.

Dada a forma como programas lógicos são escritos, esses casos podem
não ser claros e deseja-se que sejam documentados. Para tanto,
usaremos a convenção de, na documentação do funtor ou restrição
(quando existente, geralmente na forma de um comentário logo antes do
funtor ou restrição), indicar os seguintes modos:

\begin{itemize}
\item ``Variáveis de entrada'', isto é, unificadas a outro termo
  do programa, serão indicadas por um prefixo ``+'' (como em
  \codigo{f(+X)});
\item Exceto quando são esperadas estarem unificadas com termos
  base, quando serão indicadas pelo prefixo ``++'' (como em
  \codigo{f(++X)});
\item ``Variáveis de saída'', isto é, cuja expectativa é de não
  estarem unificadas a termo algum, serão indicadas pelo prefixo
  ``-'' (como em \codigo{f(-X)});
\item Quando não for necessária uma distinção entre ``variável de
  entrada'' ou ``de saída'', a variável será indicada pelo prefixo
  ``?'', como o \var{Y} em \codigo{f(++X, ?Y, -Z)}.
\end{itemize}

Adicionalmente, o \eclipse disponibiliza \funtor{comment/2}, muito
útil tanto para a documentação do código quanto para, possível auxílio
do compilador. Não usaremos essa funcionalidade neste texto, por
questões de claridade, mas sua utilização é incentivada para códigos
``em produção'' (consulte o manual para detalhes, assim como para
obter um \foreign{style guide}, que pode ser de interesse).


\subsection*{Organização}

Nosso objetivo principal aqui é introduzir o paradigma de programação
por restrições como uma extensão natural do paradigma de programação
lógica, com um foco em resolução de problemas, em particular problemas
de otimização. Para isso, começamos do básico, tentando dar uma
justificativa às escolhas feitas no texto e usando, para isso, de
exemplos e código da bibliografia. Caso seja encontrado qualquer
irregularidade, erro, ou queira fazer alguma observação, sinta-se
livre para enviar um e-mail a ericlesaquileslima at gmail dot com, ou
contatá-lo pelo meio que te for mais conveniente.

O texto pode ser razoavelmente bem dividido em três partes:
\begin{itemize}
\item Capítulos do 0 ao 6: Desenvolvimento de conceitos básicos e
  introdução a Programação Lógica, com alguns exemplos;
\item Capítulos 7 ao 11: Desenvolvimento de alguns conceitos básicos
  de programação por restrições, com alguns exemplos;
\item Capítulos 8 adiante: Desenvolvimento de mais conceitos de
  programação por restrições, com um maior foco em resolução de
  problemas e otimização.
\end{itemize}

Vale notar que, apesar de usarmos nos exemplos, principalmente, Prolog
e \eclipse e desenvolvermos no texto os conceitos básicos necessários
para sua compreensão, este texto não pretende substituir um manual ou
qualquer outro texto feito especificamente para esses sistemas.

\subsection*{Agradecimentos}

Isto que está lendo provavelmente não existiria não fosse o auxílio
proveniente do Programa Unificado de Bolsas da USP e aos incentivos
da Marina Andretta, à época minha orientadora, que resolvou apoiar e
fazer o que pôde para ajudar o no desenvolvimento projeto. A eles, só
tenho gratidão a expressar.\\
-- Éricles Lima

Imagem de capa retirada de\\
\url{https://kathleenhalme.com/explore/oak-clipart-tamarind-tree/#gal_post_875_oak-clipart-tamarind-tree-1.png}, acesso em Setembro de 2018.

%\end{document}
