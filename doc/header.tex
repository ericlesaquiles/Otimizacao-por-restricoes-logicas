\usepackage[ruled,vlined,linesnumbered]{algorithm2e}
\usepackage{amsfonts}
\usepackage{amsmath}
\usepackage{amsthm}
\usepackage{caption}
\usepackage{epigraph}
\usepackage[linguistics]{forest}
\usepackage{graphicx}
\usepackage{hyperref}
\usepackage[utf8]{inputenc}
\usepackage{listings}
\usepackage{minted}
\usepackage{placeins}
\usepackage[linguistics]{forest}
\usepackage{caption}
\usepackage{qtree}

\tikzstyle{decision} = [diamond, minimum width=1cm, minimum height=1cm, text centered, draw=black, fill=green!30]
\tikzstyle{arrow} = [thick,->,>=stealth]

\captionsetup[figure]{name=Árvore}
\setlength{\belowcaptionskip}{1pt plus 1pt minus 1pt}
\setlength{\abovecaptionskip}{1pt plus 1pt minus 1pt}

\forestset{
  tree angle/.style={
    tikz={\path () coordinate (A) -- (!u) coordinate (B) -- (!n) coordinate (C) pic [draw] {angle};}
  }
}
\usetikzlibrary{arrows.meta,angles}
\usetikzlibrary{shapes.geometric, arrows}


\captionsetup[figure]{font=scriptsize}

\newtheorem{definition}{Definição}[section]
\theoremstyle{remark}
\newtheorem*{remark}{Observações}
\theoremstyle{theorem}
\newtheorem{theorem}{Teorema}[section]

\setlength{\parskip}{.5em}

\renewcommand\refname{Leituras adicionais}
\renewcommand\figurename{Figura}
\renewcommand{\lstlistingname}{Código}

\let\oldsection\section
\renewcommand\section{\clearpage\oldsection}

\newcommand{\eclipse}{$ECL^iPS^e$}
\newcommand{\definicao}[1]{\textbf{#1}\marginpar{\small \textbf{#1}}}
\newcommand{\variavel}[1]{\textit{#1}}
\newcommand{\var}[1]{\textit{#1}}
\newcommand{\str}[1]{\textit{#1}}
\newcommand{\foreign}[1]{\textit{#1}}
\newcommand{\codigo}[1]{{\tt #1}}
\newcommand{\funtor}[1]{{\tt #1}}
\newcommand{\enphasis}[1]{\textit{#1}}
\newcommand{\enphasisb}[1]{\textbf{#1}}
\newcommand{\technical}[1]{\textit{#1}}
\newcommand{\cit}[1]{\textit{#1}}
\newcommand{\backtracking}{\textit{backtracking}}

\newcommand{\boundd}{\textit{bounds}($\mathbb{D}$)}
\newcommand{\boundz}{\textit{bounds}($\mathbb{Z}$)}
\newcommand{\boundr}{\textit{bounds}($\mathbb{R}$)}

\lstdefinestyle{prosty}{
  language=Prolog,
  basicstyle=\small
}

\patchcmd{\thebibliography}
  {\chapter*}
  {\section*}
  {}
  {}
