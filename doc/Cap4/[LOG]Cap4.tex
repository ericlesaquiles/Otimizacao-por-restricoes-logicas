\documentclass{article}

\usepackage[utf8]{inputenc}
\usepackage{placeins}
\usepackage{graphicx}
\usepackage[ruled,vlined,linesnumbered]{algorithm2e}
\usepackage{listings}
\usepackage{amsthm}
\usepackage{amsmath}
\usepackage{caption}
\usepackage{epigraph}

\captionsetup[figure]{font=scriptsize}

\newtheorem{definition}{Definição}[section]
\theoremstyle{remark}
\newtheorem*{remark}{Observações}
\theoremstyle{theorem}
\newtheorem{theorem}{Teorema}[section]

\setlength{\parskip}{.5em}

\renewcommand\refname{Leituras adicionais}
\renewcommand\figurename{Figura}
\renewcommand{\lstlistingname}{Código}

\newcommand{\eclipse}{$ECL^iPS^e$}
\newcommand{\definicao}[1]{\textbf{#1}\marginpar{\small \textbf{#1}}}

\lstdefinestyle{prosty}{
  language=Prolog,
  basicstyle=\small
}



%
% TODO: revisar terminar esta seção!
%
%


\begin{document}

\section{Alguns predicados não lógicos}

Mesmo depois de termos visto tudo o que vimos, ainda temos alguns pontos a esclarecer. A começar pela avaliação do programa: computadores comuns não vem equipados com uma placa de clarevidência, e, assim, podem não conseguir adivinhar bem o caminho para a prova de um goal. Isto é, a hipótese de não-determinismo não segue na prática\footnote{No sentido de que o programa pode não saber qual é ``é a melhor escolha''. Vale lembrar que, em outros contextos, não-determinismo indica apenas a
  presença de escolhas. }. O ideal seria que, se um goal pode ser provado por um programa, o processo de prova seguisse um caminho direto, sem tentativas de unificações infrutíferas. Na prática, isso só é realizável para situações muito específicas e, no geral, serão tentadas
diversas unificações infrutíferas antes de se chegar ao objetivo.

Para esse tipo de situação é criado o \definicao{choice point} logo antes de se tentar uma unificação. Se a tentativa resulta em falha, o processo volta ao estado anterior à tentativa, o que chamamos de \definicao{backtracking}, a possibilidade daquela unificação é eliminada e o processo continua (isto é, a mesma unificação não é tentada de novo). O goal falha quando todas as possibilidades foram eliminadas e tem sucesso quando não existirem mais unificações a serem feitas.

Os \foreign{choice points} são um ponto chave na execução de um programa lógico. A quantidade deles está diretamente relacionada com a eficiência do programa: quanto mais \foreign{choice points}, em geral, mais ineficiente o programa é. Assim, se o mesmo goal puder ser provado de mais de uma forma diferente, é necessária a criação de \foreign{choice points} a mais, o que deve ser evitado (em outras palavras, programas determinísticos costumam ser mais eficientes). Da mesma forma, se existe mais
de um resultado para uma computação (isto é, se o mesmo goal admite diversas substituições que resultam em sucesso), a eliminação das opções a mais implicariam na eliminação de \foreign{choice points}, o que seria vantajoso.

A quantidade de \foreign{choice points} tem muito a ver com a de ``axiomas''. Mais especificamente, tem a ver com a quantidade de cláusulas e com tamanho do corpo das cláusulas. Geralmente, um programa com menos axiomas resulta em melhor legibilidade e maior eficiência, no entanto, alguns desses axiomas podem representar restrições que podem levar uma falha mais cedo na computação, o que, na prática, diminuiria a quantidade de \foreign{choice points} (os que seriam criados, não houvesse essa
falha, não serão mais) e de unificações
desnecessárias, aumentando a eficiência do programa.

Veremos, então, métodos para lidar com essas situações. O mais importante, e mais polêmico, é o
\definicao{corte}, \funtor{!/0} (um funtor de nome ``!'' e aridade 0). Intuitivamente, o que ele faz é se comprometer com a escolha atual, descartando as demais. Poderemos compreender melhor como ele funciona depois que desenvolvermos a interpretação de uma computação lógica como a busca em uma árvore, mas, enquanto isso, nossa interpretação intuitiva será o suficiente.

Como um exemplo de seu uso, considere o seguinte programa:

    \begin{listing}
\inputminted{prolog}{../Exemplos/Cap4/prog1_member.pl}
\caption{Member 0}
    \end{listing}

O símbolo ``\_'' representa uma \definicao{variável anônima}. A utilizamos quando o nome daquela variável não é importante, o que acontece quando não a utilizarmos novamente (isso serve para não nos distrairmos com variáveis que não serão utilizadas: a variável anônima cumpre um papel meramente formal, de \foreign{placeholder}). Toda variável anônima é diferente, apesar de serem
escritas iguais.

O goal \codigo{member(X, Xs)?} resulta em sucesso se \var{X} é um elemento de
\var{Xs}\footnote{Estamos assumindo tacitamente, neste programa e nos demais, que a ordem de execução é ``de cima para baixo, da direita para a esquerda'' (o interpretador ``enxerga'' primeiro a cláusula que aparece ``antes''), que é como as implementações usuais de Prolog funcionam. É importante, entretanto, notar que não é assim por necessidade e existem outras ``ordens'' de avaliar programas Prolog}. Esse programa é usado primariamente de duas formas: para checar se um elemento é membro de
uma lista; ou para gerar em \var{X} os elementos da lista. Por exemplo, para o goal \codigo{member(X, [a, b, c])?}, \var{X} pode assumir os valores de a, b ou c.

Se a programadora quiser saber apenas se um certo \var{X} faz parte de uma certa lista \var{Xs}, ela pode usar o corte, ``cortando'' as demais soluções:

    \begin{listing}
\inputminted{prolog}{../Exemplos/Cap4/prog2_member2.pl}
\caption{Member 1}

    \end{listing}

O que o corte ``diz'' essencialmente é: ``Tudo bem. Agora que chegamos a este ponto, não olhe para trás''. A utilização desse novo programa para \foreign{member} é o mesmo do anterior, exceto que, agora, ao gerar membros da lista, só é gerado um elemento. Ao checar se outro elemento pertence a lista, assim que é obtido sucesso, o processo para.

Mencionamos anteriormente que o corte é usado para fins de eficiência. Quando ele é usado somente para esse fim, ou seja, se ele só serve para fazer o programa rodar melhor, não interferindo no seu significado, dizemos que é um \definicao{corte verde}.
Quando o corte não é verde, dizemos que ele é vermelho. O que acabamos de ver foi um \definicao{corte vermelho}: goals como \codigo{member(b,[a,b,c])?} e \codigo{member(c,[a,b,c])?} não estão mais no significado. Perceba que o corte não é um predicado lógico, mas sim operacional: ele nos diz algo sobre ``como'' rodar um programa mas, a princípio, não sobre ``o que'' ele é.

O corte é, provavelmente, o predicado não lógico mais importante e polêmico no Prolog. O que o torna polêmico é justamente o fato de ser não lógico. Dissemos anteriormente que a ideia da programação lógica era lidar com a parte da lógica (o ``o que fazer''), abstraindo a parte procedural (o ``como fazer''). O corte é um exemplo em que isso não foi obtido.

Apesar disso, se deixarmos de lado nosso preconceito, o corte pode contribuir de maneiras interessantes para a lógica do programa.

Por exemplo, o \funtor{not/1} pode ser implementado de maneira similar à seguinte:

    \begin{listing}
\inputminted{prolog}{../Exemplos/Cap4/prog3_not.pl}
\caption{Not}
    \end{listing}

Onde o predicado \funtor{fail/0} é um \foreign{built-in} do Prolog que falha sempre.

Similarmente, podemos, a partir do corte, gerar outras formas de controle, familiares a programadores de linguagens procedurais:

    \begin{listing}
\inputminted{prolog}{../Exemplos/Cap4/prog4_ifthenelse.pl}
\caption{SES}
    \end{listing}

    \begin{listing}
\inputminted{prolog}{../Exemplos/Cap4/prog5_or.pl}
\caption{OR}
    \end{listing}

Perceba que, diferentemente do que ocorre em programas puramente lógicos\footnote{Você pode se perguntar se algum do programa nesse texto é puramente lógico. Alguns dos mais simples, como o \enphasis{Natural}, do Capítulo 2, %COMMENT: lembrar de colocar referencia ao capitulo, no lugar do "2". E sempre que for fazer referencia a um capitulo, secao, tabela, figura, codigo, etc, com o numero na sequencia, usa letra maiuscula. Por exemplo, "Capítulo 1", "Código 3", etc.
podem ser, mas a situação geral é que os programas que veremos são só ``meio que'' lógicos.}, a ordem das cláusulas e dos termos em cada cláusula podem ser fundamentais. Pode ser que, por exemplo, na cláusula \codigo{p(a) :- b, c.}, b resulte em falha e c em um
processo interminável. Neste caso, a mudança de ordem seria fatal.

Os programas acima demonstram uma possível forma de se definir, respectivamente, o ``se, então, senão'' e o ``Ou''\footnote{Na verdade, poderíamos pensar como o processo de \technical{backtracking} como o nosso ``ou'' natural, mas às vezes pode ser conveniente usar um ``ou'' em uma cláusula, seja por efeito de legibilidade ou para evitar o \technical{backtracking}.}, mas essas construções já existem no Prolog por padrão, como:

\begin{itemize}
  \item se, então, senão: \codigo{A $->$ B ; R.} (lê-se: se A, então B, senão R);
  \item ou: \codigo{A ; B.} (lê-se: A ou B).
\end{itemize}


\subsection{Outras Opções}

Na realidade, é preciso notar que a perspectiva anterior (usando \technical{choice points} e
\technical{backtrackings}) não é a única possível (mas é a usada no Prolog padrão). Que ela poderia não
ser a única possível deve ser intuitivo, já que, a princípio, se começamos a descrever a programação lógica
de forma realmente lógica, a ordem em que as cláusulas são postas, assim como a ordem dos termos em
cada cláusula, não deveriam importar (na lógica clássica, $A \vee B$ é o mesmo que $B \vee A$), o
que deixa de ser o caso quando introduzimos artifícios tais como \technical{choice points} e
\technical{backtracking}.

Artifícios desse tipo são necessários (ou, ao menos, parecem necessários), no entanto, quando
queremos transformar um programa lógico em algo que possa ser executado. Em outras palavras, não é
suficiente termos um conjunto de relações que impliquem alguma outra relação: precisamos de um
mecanismo para construir uma prova de que o conjunto de relações implica de fato a outra relação.
Uma prova (como interpretamos aqui) é uma sequência de passos que deriva inferências\footnote{Convém
  termos uma ideia da diferença entre ``implicação'' e ``inferência'': ``A implica B'' não implica
  que ``de A se infere B''. Ter uma prova de ``A implica B'', sim.} a partir de
relações conhecidas anteriormente.

O que isso significa é que a escolha do \technical{backtracking} para resolver ``tentativas
infrutíferas'', como posto acima, é uma entre outras e, possivelmente, não a mais eficiente para
todos os casos. Atualmente, existem outros métodos como \technical{dynamic backtracking}, \technical{partial order dynamic
  backtracking} e \technical{backjumping}, com funcionamentos diferentes.

No entanto, entendemos que, no que se segue, fazer uso de outro mecanismo que não o
\technical{backtracking} adicionaria complexidade na exposição e pouco \foreign{insight} ao
entendimento. Caso a leitora interessada queira consultar detalhes, uma boa fonte de consulta é
\cite{dyn}




  \begin{thebibliography}{2}

    \bibitem{dyn}
    Final Technical Report,
    ``DYNAMIC BACKTRACKING'',
    February 1997,
    University of Oregon


  \end{thebibliography}

\end{document}
