\documentclass{article}
\usepackage[utf8]{inputenc}
\usepackage[portuguese]{babel}
\usepackage{placeins}

\usepackage{amsthm}
\usepackage[ruled,vlined,linesnumbered]{algorithm2e}

\usepackage{listings}
\lstset{language=Prolog}
\renewcommand{\lstlistingname}{Código}

%\newtheoremstyle{definition}% name of the style to be used
  %{3pt}% measure of space to leave above the theorem. E.g.: 3pt
  %{3pt}% measure of space to leave below the theorem. E.g.: 3pt
  %{}% name of font to use in the body of the theorem
  %{2cm}% measure of space to indent
  %{}% name of head font
  %{:}% punctuation between head and body
  %{.5em}% space after theorem head; " " = normal interword space
  %{}% Manually specify head


%\theoremstyle{definition}{indent=1cm}
\newtheorem{definition}{Definição}[section]

\theoremstyle{remark}
\newtheorem*{remark}{Remark}

\setlength{\parskip}{.5em}

%\documentclass[tikz,border=5]{standalone}
%\usetikzlibrary{graphs,graphdrawing,arrows.meta}
%\usegdlibrary{trees}

\usepackage[linguistics]{forest}

%
% TODO: revisar terminar esta seção!
%
%


\begin{document}

\section{Alguns predicados não lógicos}

Até agora, temos nos ocupado mais com aspectos teóricos de programação lógica. Na prática, ainda temos alguns problemas a resolver.

A começar pela avaliação do programa: computadores comuns não vem equipados com uma placa de clarevidência, e, assim, podem não conseguir adivinhar bem o caminho para a prova de um goal. Isto é, a hipótese de não-determinismo não segue na prática. O ideal seria que, se um goal pode ser provado por um programa, o processo de prova seguisse um caminho direto, sem tentativas de unificações infrutíferas. Na prática, isso só é realizável para situações muito específicas e, no geral, serão tentadas
diversas unificações infrutíferas antes de se chegar ao objetivo.

Para lidar com isso é criado um \textbf{\textit{choice point}}\marginpar{\textbf{Choice point}} logo antes de se tentar uma unificação. Se a tentativa resulta em falha, o processo volta ao estado de antes da tentativa, no que chamamos de \textbf\textit{{backtracking}}\marginpar{\textbf{Backtracking}}, a possibilidade daquela unificação é eliminada e o processo continua. O goal falha quando todas as possibilidades foram eliminadas e tem sucesso quando não existirem mais unificações a serem feitas.

Os \textit{choice points} são um ponto chave na execução de um programa lógico. A quantidade deles está diretamente relacionada com a eficiência do programa: quanto mais choice points, em geral, mais ineficiente o programa é. Assim, se o mesmo goal puder ser provado de mais de uma forma diferente, é necessária a criação de \textit{choice points} a mais, o que deve ser evitado. Da mesma forma, se existe mais de um resultado para uma computação (isto é, se o mesmo goal admite
diversas substituições que resultam em sucesso)
, a eliminação das opções a mais implicariam na eliminação de \textit{choice points}, o que seria vantajoso.

A quantidade de \textit{choice points} tem muito a ver com a de ``axiomas'' (isto é, tem a ver com a quantidade de cláusulas e com tamanho do corpo das cláusulas). Geralmente, um programa com menos axiomas resulta em melhor legibilidade e maior eficiência. No entanto, alguns desses axiomas representam restrições que podem levar uma falha mais cedo na computação, o que, na prática, diminui a quantidade de \textit{choice points} (os que seriam criados, não houvesse essa
falha, não serão mais) e de unificações
desnecessárias, aumentando a eficiência do progama.

Veremos, então, métodos para lidar com essas situações. O mais importante, e mais polêmico, é o \textbf{corte}, o !/0 (um funtor de nome ``!'' e aridade 0). Intuitivamente, o que ele faz é se comprometer com a escolha atual, descartando as demais. Só poderemos compreender completamente como ele funciona depois que desenvolvermos a interpretação de uma computação lógica como a busca em uma árvore, mas, enquanto isso, nossa interpretação intuitiva será o suficiente.

Como um exemplo de seu uso, considere o seguinte programa:

\lstinputlisting[caption=Member]{../Exemplos/Cap4/prog1_member.pl}

O underline ``\_'' representa uma \textbf{variável anônima}\marginpar{\textbf{Variável anônima}}. A utilizamos quando o nome daquela variável não é importante, o que acontece quando não a utilizarmos novamente (isso serve para não nos distrairmos com variáveis que não serão utilizadas: a variável anônima cumpre um papel meramente formal). Toda variável anônima é diferente.

O goal {\tt member(X, Xs)?} resulta em sucesso se X é um elemento de Xs\footnote{Note que estamos assumindo tácitamente, neste programa e nos demais, que a ordem de execução é ``de cima para baixo, da direita para a esquerda'' (o interpretador ``enxerga'' primeiro a cláusula que aparece ``antes''), que é como as implementações usuais de Prolog funcionam}. Esse programa é usado primariamente de duas formas: para checar se um elemento é membro de uma lista; ou para gerar em X os elementos da lista. Por exemplo, para o goal {\tt member(X, [a, b, c])?}, X pode assumir o valor de a, b ou c.

Se a programadora quiser saber apenas se um certo X faz parte de uma certa lista Xs, ela pode usar o corte, ``cortando'' as demais soluções:

\lstinputlisting[caption=Member]{../Exemplos/Cap4/prog2_member2.pl}

O que o corte ``diz'' essencialmente é: ``Tudo bem, agora que chegamos a este ponto, não olhe para trás''. O uso desse novo programa para \textit{member} é o mesmo do anterior, exceto que, agora, ao gerar membros da lista, só é gerado um membro e, ao checar se outro elemento pertence a lista, assim que se chega ao sucesso, o processo para.

Mencionamos anteriormente que o corte é usado para fins de eficiência. Quando ele é usado somente para esse fim, ou seja, se ele só serve para fazer o programa rodar melhor, não interferindo no seu significado, dizemos que o corte é verde\marginpar{\textbf{Corte verde}}.
Quando o corte não é verde, dizemos que ele é vermelho. O que acabamos de ver foi um corte vermelho\marginpar{\textbf{Corte Vermelho}}: goals como {\tt member(b,[a,b,c])?} e {\tt member(c,[a,b,c])?} não estão mais no significado. Perceba que o corte não é um predicado lógico, mas sim operacional: ele nos diz algo sobre ``como'' rodar um programa mas, a princípio, não sobre ``o que'' ele é.

O corte é, provavelmente, o predicado não lógico mais importante e polêmico do Prolog. O que o torna polêmico é justamente o fato de ser não lógico. Dissemos anteriormente que a ideia da programação lógica era lidar com a parte da lógica (o ``o que fazer''), abstraindo a parte procedural (o ``como fazer''). O corte é um exemplo em que isso não foi obtido.

Apesar disso, se deixarmos de lado nosso preconceito, o corte pode contribuir de maneiras interessantes para a lógica do programa.

Por exemplo, o not/1 pode ser implementado de maneira similar à seguinte:

\lstinputlisting[caption=Not]{../Exemplos/Cap4/prog3_not.pl}

Onde o predicado fail/0 é um predicado do prolog que falha sempre.

Similarmente, podemos, a partir do corte, gerar outras formas de controle, familiares a programadores de linguagens procedurais:

\lstinputlisting[caption=Se entao se nao]{../Exemplos/Cap4/prog4_ifthenelse.pl}

\lstinputlisting[caption=Ou]{../Exemplos/Cap4/prog5_or.pl}

Perceba que, diferentemente do que ocorre em programas puramente lógicos\footnote{Você pode se perguntar se algum do programa nesse texto é puramente lógico. Alguns dos mais simples, como o \textit{Natural}, do capítulo 2, podem ser, mas a situação geral é que os programas que veremos são só ``meio que'' lógicos.}, a ordem das cláusulas e dos termos em cada cláusula podem ser fundamentais. Pode ser que, por exemplo, na cláusula {\tt p(a) :- b, c.}, b resulte em falha e c em um
processo interminável. Neste caso, a mudança de ordem seria fatal.

Os programas acima demonstram uma possível forma de se definir, respectivamente, o ``se,então, se não'' e o ``Ou''\footnote{Na verdade, poderíamos pensar como o processo de backtracking como o nosso ``ou'' natural, mas às vezes pode ser conveniente usar um ``ou'' em uma cláusula, seja por efeito de legibilidade ou para evitar o backtracking.}, mas essas construções já existem no Prolog por padrão, como:

\begin{itemize}
  \item se, então, se não: {\tt A $->$ B ; R.} (lê-se: se A, então B, se não R);
  \item ou: {\tt A ; B.} (lê-se: A ou B).
\end{itemize}


  \begin{thebibliography}{2}


  \end{thebibliography}

\end{document}
