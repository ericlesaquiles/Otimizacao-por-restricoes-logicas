\documentclass{article}

\usepackage[utf8]{inputenc}
\usepackage{placeins}
\usepackage{graphicx}
\usepackage[ruled,vlined,linesnumbered]{algorithm2e}
\usepackage{listings}
\usepackage{amsthm}
\usepackage{amsmath}
\usepackage{caption}
\usepackage{epigraph}

\captionsetup[figure]{font=scriptsize}

\newtheorem{definition}{Definição}[section]
\theoremstyle{remark}
\newtheorem*{remark}{Observações}
\theoremstyle{theorem}
\newtheorem{theorem}{Teorema}[section]

\setlength{\parskip}{.5em}

\renewcommand\refname{Leituras adicionais}
\renewcommand\figurename{Figura}
\renewcommand{\lstlistingname}{Código}

\newcommand{\eclipse}{$ECL^iPS^e$}
\newcommand{\definicao}[1]{\textbf{#1}\marginpar{\small \textbf{#1}}}

\lstdefinestyle{prosty}{
  language=Prolog,
  basicstyle=\small
}



\begin{document}

\section{Propagação de restrições em domínios finitos}

Nesta seção trabalharemos primariamente com restrições em domínios finitos.
Domínios finitos são importantes porque costumam ser bons para modelar
decisões, o que é algo com que gostaríamos que o computador ajudasse.

Um exemplo simples e bem conhecido é o problema de coloração de um mapa: dado um conjunto finito (digamos, igual a 5) de cores, precisamos colorir um mapa (digamos, o mapa do Cazaquistão) de modo que nenhuma das regiões do mapa receba a mesma cor que outra região com que faça fronteira\footnote{Acontece que esse problema é essencialmente o mesmo que as companhias de aviação tem para alocar seu tráfico aéreo.}.
Outro exemplo bem conhecido é o do ``casamento a moda antiga'' (menos popularmente conhecido como o ``problema da correspondência bipartida'').
Nesse problema, temos um conjunto de homens, um de mulheres a relação de restrição {\tt gosta/2}, que existe quando um indivíduo $i$ gosta de outro $j$.
O problema é separar esses grupos de homens e mulheres em casais que se gostam.

Esses dois exemplos tem a particularidade de terem restrições primitivas binárias e, por isso, são chamados de CSPs binários. Um ponto interessante em CSPs binários é que sempre podem ser representados como um grafo não direcionado:
  cada variável (cada indivíduo no segundo exemplo ou cada região do Cazaquistão no primeiro) é representada como um nó e cada restrição como um arco entre suas variáveis\footnote{No geral, restrições CSPs em restrições n-árias podem ser representados como um multi-grafo.}.

  Em particular, problemas como os de roteamento e criação de cronogramas
  costumam ser facilmente expressos como CPs em domínio finito, o que indica a sua importância comercial.

  Essa classe de problemas foi estudada por diferentes comunidades científicas.
  A comunidade de Inteligênica Artificial desenvolvou técnicas de consistência
por arco e por nó, para CSPs, a comunidade de programação por restrições desenvolveu  técnicas de propagação de limites e a comunidade de pesquisas
operacionais desenvolveu técnicas de programação inteira.

\subsection{Consistência por nó e por arco}

Resolução de CSPs por consistência por nó e por arco acontece em tempo polinomial
(mas, possívelmente, de forma incompleta)\footnote{Mas ela, assim como as demais formas de consistência vistas aqui
 pode ser usada em conjunto com \textit{backtracking}, gerando um resolvedor completo.}.
A ideia aqui é diminuir os domínios das variáveis,
transformando o problema em outro equivalente (com as mesmas soluções).
Se o domínio de alguma variável for vazio, é o fim do CSP.

Essa forma de resolução é dita ser baseada em consistência, porque ele
funciona propagando informações dos domínios de cada variável para
os demais, tornando-os ``consistentes'' entre si.

\begin{definition}
  Uma restrição $r/n$ é dita \textbf{consistente por nó}\marginpar{\textbf{Consistente por nó}} com um domínio D
  se $n > 1$ ou, $X$ sendo for uma variável de $r$, se para cada $d$ no domínio
  de $X$ {\tt  X=d,r(X).} resulta em sucesso.
  Uma restrição composta é dita consistente por nó se cada uma de suas restrições
  primitivas o é.
\end{definition}

\begin{definition}
  Uma restrição $r/n$ é dita \textbf{consistente por arco}\marginpar{\textbf{Consistente por arco}} se
  $n \neq 2$ ou, se $r$ é uma restrição nas variáveis $X$ e $Y$ e se $D_x$ é o domínio de $X$ e $D_y$ o
  de $Y$, então $x \in D_x \Rightarrow \exists y \in D_y : X=x, Y=y, r(X,Y).$ resulta em sucesso.
  Uma restrição composta é dita consistente por arco se cada uma de suas restrições primitivas o é.
\end{definition}

Essas noções de consistência são noções fracas no sentido de que um CSP pode não ser satisfazível
e ainda manter consistência por arco e por nó.

Não é difícil escrever um código para manter consistêncai por arco e por nós. A seguir segue um exemplo. Ele é para fins
demonstrativos: para algoritmos mais eficientes veja \cite{tsang}. O {\tt apply/2} usado foi definido no Capítulo 6.

\lstinputlisting[caption=Consistência por nó, style=prosty]{../Exemplos/Cap9/node_consistency.pl}\label{lst:no_consistency}
\vspace{2cm}

\lstinputlisting[caption=Consistência por arco, style=prosty]{../Exemplos/Cap9/arc_consistency.pl}\label{lst:arc_consistency}

\subsection{Consistência por limites\footnote{Mais conhecido como \textit{bounds consistency}}}

As noções de consistência desenvolvidas acima funcionam bem para restrições em uma ou duas variáveis,
mas se quisermos usar algo do tipo para mais variáveis, precisaremos generalizar um pouco:

\begin{definition}
  Uma restrição $r/n$ nas variáveis $X_1$, ..., $X_n$ é dita \textbf{consistente por hiper-arco}\marginpar{\textbf{Consistente por hiper-arco}}
  se para cada $x_i$ no domínio de $X_i$, existem $x_j$ nos domínios de $X_j$ tal que
  $X_i = x_i, X_j = x_j$  $\forall j: 1 \geq j \leq n$, $j \neq i$.
  Uma restrição composta é dita consistente por hiper-arco se cada uma de suas restrições o é.
\end{definition}

Infelizmente, manter a consistência por hiper-arco é algo caro demais para se fazer em um problema geral. Para
encontrarmos uma nova checagem de consistência realmente útil, precisaremos restringir o domínio com que lidamos.

Dizemos que um CSP é aritmético\marginpar{\textbf{CSP aritmético}} se o domínio de cada variável é uma união
finita de intervalos finitos de números inteiros e se as restrições são aritméticas. Muitos CSPs podem ser modelados como aritméticos de forma natural, e muitos outros podem ser transformados em CSPs aritméticos por uma mudança de variáveis.
Por exemplo, se o problema tem a ver com escolhas, uma mudança de variáveis natural é denotar cada escolha por um número. No problema da
coloração do mapa do Cazaquistão (mencionado acima), ao invés de denotar as cores como ``vermelho'', ``azul'', etc., podemos denotá-las como ``1'', ``2'', etc., obtendo resultados equivalentes.

Lidando com CSPs aritméticos, podemos definir a noção de \textbf{consistência por limites}. A ideia é limitar o domínio
de uma variável por limites inferiores e superiores. As sequintes convenções de notação serão convenientes:

\begin{itemize}
  \item $min_D(X)$ := $x: y \geq x \forall y \in D$;
  \item $max_D(X)$ := $x: y \leq x \forall y \in D$.
\end{itemize}

\begin{definition}
  Uma restrição $r/n$ nas variáveis $X_1$, ..., $X_n$ é dita consistente por limites\marginpar{\textbf{Consistente por limites}}
  se
  \begin{itemize}
    \item Para cada $x_i$ variável de $r/n$, existem valores reais $x_1$, ..., $x_n$ tal que $min_D(x_j) \geq x_j \leq max_D(x_j)$
          e $X_i = min_D(X_i), X_j = x_j$ $\forall j \neq i$ é uma solução de r e
    \item Outros valores reais $x_1$, ..., $x_n$ tal que $min_D(x_j) \geq x_j \leq max_D(x_j)$
          e $X_i = max_D(X_i), X_j = x_j$ $\forall j \neq i$ é uma solução de r e
  \end{itemize}
  Uma restrição composta é dita consistente por limites se cada uma de suas restrições o é.
\end{definition}

Um método eficiente, que é uma \textbf{regra de propagação} pode ser eleborado a partir
de uma constatação ilustrada no seguinte exemplo:

Considere a restrição {\tt X = Z + Y}. Ela pode ser escrita nas formas
\[
  X = Z + Y \text{, } Y = X - Y \text{, } Z = X - Y
\]

Podemos ver que:
\begin{gather}
  X \geq min_D(Y)+min_D(Z),  X \leq max_D(Y) + max_D(Z)\\
  Y \geq min_D(Y)+min_D(Z),  Y \leq max_D(Y) + max_D(Z)\\
  Z \geq min_D(Y)+min_D(Z),  Z \leq max_D(Y) + max_D(Z)
\end{gather}

Podemos usar essa observação para tentar diminuir os domínios de X, Y e Z. Com essa ideia, obtemos o seguinte programa:

\lstinputlisting[caption=Busca, style=prosty]{../Exemplos/Cap9/bounds_consistency.pl}\label{lst:bounds}

Observações semelhantes podem ser feitas para outros tipos de restrições aritméticas. Para restrições do tipo $X \neq Z$ e $X  \neq min(Z, Y)$, isso é especialmente simples de ser feito. Para restrições  não lineares do tipo $X < Z\times Y$, isso pode ser
custoso se $Z$ e $Y$ podem assumir valores positivos e negativos, mas ainda pode ser feito.

Como é meio chato escrever uma regra para cada caso de restrição dessas para a manutenção de consistência por limites, o que é
mais usual é que um sistema que ofereça esse tipo de consistência suporte apenas uma quantidade reduzida dessas restrições,
sendo as demais transformadas em versões equivalentes às quais essas restrições se apliquem, o que não é difícil de se fazer.
Isso está sujeito ao potencial incoveniente de que restrições equivalentes mas escritas de formas diferentes podem oferecer
oportunidades diferentes para a diminuição de domínio de cada restrição e a reescrita pode um domínio que poderia originalmente ser
grandemente simplificado, sofra apenas uma pequena alteração.

Apesar disso, a aplicação de consistência por limites frequentemente é útil. Um programa que realiza essa aplicação é simples
de se fazer: ele toma cada restrição primitiva e os domínios de suas respectivas variáveis e aplica um algo como o mostrado no
código \ref{lst:bounds}. Assim, temos um mecanismo de busca incompleto. Torná-lo um mecanismo completo é simples e pode ser feito
com a adição do backtracking.


\subsection{Consistência generalizada e complexa}

Consistência por limites também podem ser aplicadas a restrições em duas ou em uma variável, mas, nesse caso, consistência por
arco ou por nó costumam resultar em diminuição maior nos domínios. Um problema geral, no entanto, pode ser composto por uma
combinação de restrições de diferente aridade, tornando vantajosa a aplicação de diferentes noções de consistência.

No entanto, um potencial problema com abordagens baseadas em consistência é que elas consideram apenas uma ou um pequeno número de
restrições e variáveis por vez, enquanto que frequentemente muitas restrições e variáveis oferecem informações sobre as demais.

Tome por exemplo a seguinte restrição {\tt X $\neq$ Y, Y $\neq$ Z}, que é equivalente a dizer que as variáveis X, Y e Z são todas
diferentes. Dos métodos de consistência que vimos, o mais indicado a essa restrição é o de consistência em arco, já que $\neq$ é
de aridade 2. Mas esse método é muito fraco para restrições de desigualdade, o que significa que, na prática, resolver isso
por \textit{backtraking}, que tem um crescimento assintótico exponencial.% TODO adicionar exemplo

Essa dificuldade decorre de não usarmos que o domínio de cada variável, nesse caso, oferece informações sobre os domínios das
demais. Essa restrição é tão usada que recebeu um nome especial e em vários sistemas CLP é denotada {\tt all\_diferent/1}.
Restrições que fazem uso da informação no domínio de muitas variáveis para realizar a atualização de cada domínio são
chamadas \textbf{restrições complexas}. No cado do \textit{alldifferent/1}, podemos notar que essa restrição é equivalente ao
supra-mencionado problema de correspondência bipartida e que existem algoritmos eficientes para lidar com ele.






  \begin{thebibliography}{1}

    \bibitem{tsang}
      E. Tsang (1930), ``Foundations of Constraint Satisfaction'', Academic Press.

  \end{thebibliography}




\end{document}
