%\documentclass{article}

%\usepackage[utf8]{inputenc}
\usepackage{placeins}
\usepackage{graphicx}
\usepackage[ruled,vlined,linesnumbered]{algorithm2e}
\usepackage{listings}
\usepackage{amsthm}
\usepackage{amsmath}
\usepackage{caption}
\usepackage{epigraph}

\captionsetup[figure]{font=scriptsize}

\newtheorem{definition}{Definição}[section]
\theoremstyle{remark}
\newtheorem*{remark}{Observações}
\theoremstyle{theorem}
\newtheorem{theorem}{Teorema}[section]

\setlength{\parskip}{.5em}

\renewcommand\refname{Leituras adicionais}
\renewcommand\figurename{Figura}
\renewcommand{\lstlistingname}{Código}

\newcommand{\eclipse}{$ECL^iPS^e$}
\newcommand{\definicao}[1]{\textbf{#1}\marginpar{\small \textbf{#1}}}

\lstdefinestyle{prosty}{
  language=Prolog,
  basicstyle=\small
}



%\begin{document}

\section{Propagação de restrições em domínios finitos}

Nesta seção trabalharemos primariamente com restrições em domínios
finitos.  Domínios finitos são importantes porque costumam ser bons
para modelar decisões, o que é algo com que gostaríamos que o
computador ajudasse.

Um exemplo simples e bem conhecido é o problema de coloração de um
mapa: dado um conjunto finito de cores (de tamanho 15, por exemplo),
precisamos colorir um mapa (digamos, o do Cazaquistão) de modo que
nenhuma de suas regiões receba a mesma cor que outra com que faça
fronteira\footnote{Incidentalmente, acontece que esse problema é
  essencialmente o mesmo que as companhias de aviação tem para alocar
  seu tráfico aéreo.}.  Outro exemplo bem conhecido é o do ``casamento
a moda antiga'' (menos popularmente conhecido como o ``problema da
correspondência bipartida'').  Nesse problema, temos um conjunto de
homens, um de mulheres a relação \codigo{gosta/2}, que existe quando
um indivíduo $i$ gosta de outro $j$.  O problema é separar esses
grupos de homens e mulheres em casais que se gostam.

Esses dois exemplos tem a particularidade de terem restrições
primitivas binárias e, por isso, são chamados de CSPs binários. Um
ponto interessante em CSPs binários é que sempre podem ser
representados como um grafo não direcionado: cada variável (cada
indivíduo no segundo exemplo ou cada região do Cazaquistão, no
primeiro) é representada como um nó e cada restrição como um arco
entre suas variáveis. Mais em geral, restrições n-árias CSPs podem ser
representados como um multigrafo (isto é, um grafo em que podem
existir mais de uma aresta entre dois vértices).

Em particular, problemas como os de roteamento e criação de
cronogramas costumam ser facilmente expressos como CPs em domínio
finito, o que indica sua importância comercial.

Sendo de uso tão amplo, essa classe de problemas (a de CPs em
domínios finitos) foi estudada por diferentes comunidades
científicas. A comunidade de Inteligência Artificial desenvolveu
técnicas de consistência por arco e por nó, para CSPs, a comunidade
de programação por restrições desenvolveu técnicas de propagação de
limites e a comunidade de pesquisas operacionais desenvolveu
técnicas de programação inteira. Daremos uma olhada em cada uma
dessas abordagens a seguir.

\subsection{Consistência por nó e por arco}

Resolução de CSPs por consistência por nó e por arco acontece em tempo
polinomial (mas, possivelmente, de forma incompleta)\footnote{Mas ela,
  assim como as demais formas de consistência vistas aqui pode ser
  usada em conjunto com \technical{backtracking}, gerando um
  resolvedor completo, mas não mais em tempo polinomial.}.  A ideia
aqui é diminuir os domínios das variáveis, transformando o problema em
outro equivalente (com as mesmas soluções).  Se o domínio de alguma
variável for vazio, é o fim do CSP.

Essa forma de resolução é dita baseada em consistência porque ele
funciona propagando informações dos domínios de cada variável para os
demais, tornando-os ``consistentes'' entre si.

\begin{definition}
  Uma restrição $r/n$ é dita \definicao{consistente por nó} se $n > 1$
  ou, $X$ sendo for uma variável em $r$, se para cada $d$ no domínio
  de \var{X}, \codigo{ X=d,r(X).} resulta em sucesso.  Uma restrição
  composta é dita consistente por nó se cada uma de suas restrições
  primitivas o é.
\end{definition}

\begin{definition}
  Uma restrição $r/n$ é dita \definicao{consistente por arco} se $n
  \neq 2$ ou, se \enphasis{r} é uma restrição nas variáveis \var{X} e
  \var{Y} e se $D_x$ é o domínio de \var{X} e $D_y$ o de \var{Y},
  então $x \in D_x$ implica que existe $y \in D_y$ tal que
  \codigo{X=x, Y=y, r(X,Y).} resulta em sucesso.  Uma restrição
  composta é dita consistente por arco se cada uma de suas restrições
  primitivas o é.
\end{definition}

Vale notar que essas noções de consistência são noções fracas no
sentido de que um CSP pode não ser satisfazível e ainda manter
consistência por arco e por nó.

Não é difícil escrever um código para manter consistência por arco e
por nós. A seguir segue um exemplo. Ele é para fins demonstrativos:
para algoritmos mais eficientes veja \cite{tsang}. O funtor
\codigo{apply/2} é o definido no Capítulo 6. %TODO adicionar link

\begin{listing}[H]
  \inputminted{prolog}{../Exemplos/Cap9/node_consistency.pl}\label{lst:no_consistency}
  \caption{Consistência por nó}
\end{listing}
\vspace{2cm}

\begin{listing}[H]
  \inputminted{prolog}{../Exemplos/Cap9/arc_consistency.pl}\label{lst:arc_consistency}
  \caption{Consistência por arco}
\end{listing}

O \codigo{consistente\_por\_no/2} recebe uma lista de pares
``Domínio-Restrição''. Se a restrição for unária, ela checa cada valor
do domínio.  Os valores que resultam em falha são retirados do
domínio. \codigo{consistente\_por\_arco/2} tem alguns casos a mais,
mas é essencialmente a mesma coisa.

\subsection{Consistncia por limite}

As noções de consistência desenvolvidas a cima funcionam bem para
restrições em uma ou duas variáveis, mas se quisermos usar algo do
tipo para mais variáveis, precisaremos generalizar um pouco:

\begin{definition}
  Uma restrição \codigo{r/n} nas variáveis \var{$X_1$}, ...,
  \var{$X_n$} é dita \definicao{consistente por hiper-arco} se para
  cada $x_i$ no domínio de \var{$X_i$}, existem $x_j$ nos domínios de
  \var{$X_j$} tais que para todo $j \neq i$ entre 1 e \enphasis{n},
  \codigo{$X_i$ = $x_i$, $X_j$ = $x_j$.} resulta em sucesso. Uma
  restrição composta é dita consistente por hiper-arco se cada uma de
  suas restrições o é.
\end{definition}

Infelizmente, manter consistência por hiper-arco é algo caro demais
para se fazer em um problema geral. Para encontrarmos uma nova
checagem de consistência realmente útil, precisaremos restringir o
domínio com que lidamos.

Dizemos que um \definicao{CSP é aritmético} se o domínio de cada
variável é uma união finita de intervalos finitos de números inteiros
e se as restrições são aritméticas. Muitos CSPs podem ser modelados
como aritméticos de forma natural e muitos outros podem ser
transformados em CSPs aritméticos por uma mudança de variáveis
apropriada.  Por exemplo, se o problema tem a ver com escolhas, uma
mudança de variáveis natural é denotar cada escolha por um número. No
problema da coloração do mapa do Cazaquistão, por exemplo, ao invés de
denotar as cores como ``vermelho'', ``azul'', etc., podemos denotá-las
como ``1'', ``2'', etc., obtendo resultados equivalentes.

Lidando com CSPs aritméticos, podemos definir a noção de
\definicao{consistência por limites}. A ideia é limitar o domínio de
uma variável por limitantes inferiores e superiores. As seguintes
convenções de notação serão convenientes:

\begin{itemize}
\item $min_D(\var{X})$ := algum $x$ tal que para todo $y \in D$, $y
  \geq x$;
\item $max_D(\var{X})$ := algum $x$ tal que para todo $y \in D$, $y
  \leq x$.
\end{itemize}

Dadas essas convenções, podemos construir uma definição de
consistência por limites. Por incrível que pareça, existem três noções
de consistência por limites, uma incompatível com a outra. Não
obstante, os pesquisadores frequentemente confundem uma noção com a
outra (os motivos de tal confusão serão explicados a seguir).

As três noções de consistência por limites serão denotadas \boundd,
\boundz{} e \boundr.  Intuitivamente, para \boundd, dentro de cada
limitante no domínio de uma variável há ``suporte inteiro'' para os
valores dos domínios das outras variáveis ocorrendo na mesma
restrição.  Para \boundz, o ``suporte inteiro'' precisa ser apenas
dentro do intervalo definido pelos limitantes inferiores e superiores
das outras variáveis.  Para \boundr, os ``suportes'' podem assumir
valores reais no intervalo definido pelos limitantes inferiores e
superiores das outras variáveis.

Mais formalmente, temos as seguintes definições:

\begin{definition}
  Uma restrição \funtor{r/n} nas variáveis \var{$X_1$}, \dots,
  \var{$X_n$} é dita \definicao{\boundd{} consistente} se, para cada
  \var{$X_i$}, existem \enphasisb{valores inteiros} $x_j$ no domínio
  de \var{$X_j$}, para $j \neq i$, tais que \{\var{$X_k$} = $x_k$ : $1
  \leq k \leq n$ \} é uma solução de \funtor{r/n}.

  %$min_D(X_j) \leq x_j \leq max_D(X_j)$ e $\{X_i = min_D(X_i), X_j =
  %x_j\}$ é uma solução de r.  E outros valores reais $x_1$, ...,
  %$x_n$ tais que para todo $j \neq i $, $min_D(x_j) \leq x_j \leq
  %max_D(x_j)$ e $\{X_i = max_D(X_i)$, $X_j = x_j\}$ é uma solução de
  %r.

  Uma restrição composta é dita \boundd{} consistente se cada uma de
  suas restrições o é.
\end{definition}

\begin{definition}
  Uma restrição \funtor{r/n} nas variáveis \var{$X_1$}, \dots,
  \var{$X_n$} é dita \definicao{\boundz{} consistente} se, para cada
  \var{$X_i$}, existem \enphasisb{valores inteiros} $x_j$, para $j
  \neq i$, satisfazendo $min_D(\var{$X_j$}) \leq x_j \leq
  max_D(\var{$X_j$})$, tais que \{\var{$X_k$} = $x_k$ : $1 \leq k \leq
  n$ \} é uma solução inteira de \funtor{r/n}.

  Uma restrição composta é dita \boundz{} consistente se cada uma de
  suas restrições o é.
\end{definition}

\begin{definition}
  Uma restrição \funtor{r/n} nas variáveis \var{$X_1$}, \dots,
  \var{$X_n$} é dita \definicao{\boundr{} consistente} se, para cada
  \var{$X_i$}, existem \enphasisb{valores reais} $x_j$, para $j \neq
  i$, satisfazendo $min_D(\var{$X_j$}) \leq x_j \leq
  max_D(\var{$X_j$})$, tais que \{\var{$X_k$} = $x_k$ : $1 \leq k \leq
  n$ \} é uma solução real de \funtor{r/n}.

  Uma restrição composta é dita \boundr{} consistente se cada uma de
  suas restrições o é.
\end{definition}

É claro pelas definições que \boundd{} $\Rightarrow$ \boundz{}
$\Rightarrow$ \boundr. Existem, no entanto, problemas práticos com
\boundd{} e \boundz, nomeadamente que, para restrições lineares, por
exemplo, manter consistêncai por \boundd{} ou por \boundz{} é um
problema NP-completo, enquanto que manter consistência por \boundr{}
pode ser feito em tempo linear.

Também vale notar que um conjunto de domínios D é consistente em
\boundz{} ou \boundr{} para uma restrição \funtor{r/n} se, e só se,
range(D) é consistente em \boundz{} ou em \boundr{} para \funtor{r/n},
em que range(D) é definido como $range(D) := \{[min_{D_{x_i(X)}} \ldots
  max_{D_{x_{i(X_i)}}}]\}$, $D_{x_i}$ o domínio da variável de \funtor{r/n}
$X_i$. Essa propriedade é muito usada para não reexecutar propagadores
de limites sem necessidade, e não vale para \boundd{}.

Um problema com consistência em \boundr{} é que pode não ser claro
como interpretar uma restrição inteira nos reais.

Com tantas diferenças entre essas noções, alguém poderia nos perguntar
``Como alguém poderia confundi-las?''. O fato é que, para muitas
restrições, elas são equivalentes. Essas restrições são ditas
monótonas. Para definir uma restrição monótona, será conveniente ter
uma definição mais refinada do que é um domínio. Daqui para frente, um
\definicao{domínio} D de uma restrição \funtor{r/n} será tido como uma
função do conjunto de variáveis de \funtor{r/n} para os conjuntos de
valores que essa variável pode receber. Por exemplo, para
\codigo{r(X,Y) :- X = Y.}, em que \var{X} pode assumir os valores 1 e
2 e, \var{Y}, os valores 2 e 3, D(X) = \{1, 2\} e D(Y) = \{2, 3\}. Por
abuso de notação, diremos que $\Gamma \in D$ se $\Gamma$ é outra
função domínio tal que $\Gamma(X_i) \subseteq D(X_i)$ para $1 \leq i
\leq n$.

\begin{definition}
  Uma restrição \funtor{r/n} é dita monótona com respeito às suas
  variáveis \var{$X_i$} se, e só se, existe uma ordem
  total\footnote{Lembrete: uma ordem total em um conjunto S é uma
    relação $\preceq$ que satisfaz, para todo $a, b, c \in S$: $a
    \preceq a$, $a \preceq b \wedge b \preceq a \Rightarrow a = b$,
    $a\preceq b \wedge b \preceq c \Rightarrow a \preceq c$ e
    $a\preceq b \vee b\preceq a$.}  $\prec_i$\footnote{A ordem pode
    mudar segundo a variável.}  no domínio de $X_i$, $D(X_i)$,
  tal que, se para todo i, $\Gamma \in D(X_i)$ é uma solução de
  \funtor{r/n}, então também o é qualquer $\Gamma$' tal que, para $i
  \neq j$, $\Gamma$'$(X_j) = \Gamma(X_j)$ e $\Gamma$'$(X_i) \preceq_i
  \Gamma(X_i)$.
\end{definition}

Restrições na forma de desigualdades lineares\footnote{Exceto
  desigualdades da forma $x \neq y$, mas essas são muito fáceis de
  lidar, além de serem equivalentes para \boundd, \boundz{} e
  \boundr.} e de desigualdades do tipo $x_1 \times x_2 \leq x_3$, com
$min_D(x_i) \geq 0$, são monótonas, por exemplo.

Mais detalhes sobre consistência por limites podem ser vistos em
\cite{choi}. A seguir é elaborado um exemplo de um método de
propagação por limites.

Considere a restrição \codigo{X = Z + Y}. Ela pode ser escrita nas
formas
\[
X = Z + Y \text{, } Y = X - Y \text{, } Z = X - Y
\]

Podemos ver que:
\begin{gather}
  X \geq min_D(Y)+min_D(Z), X \leq max_D(Y) + max_D(Z)\\ Y \geq
  min_D(Y)+min_D(Z), Y \leq max_D(Y) + max_D(Z)\\ Z \geq
  min_D(Y)+min_D(Z), Z \leq max_D(Y) + max_D(Z)
\end{gather}

Podemos usar essa observação para tentar diminuir os domínios de X, Y
e Z. Com essa ideia, obtemos o seguinte programa:

\begin{listing}[H]
  \inputminted{prolog}{../Exemplos/Cap9/bounds_consistency.pl}
  \caption{Consistência por limites}\label{lst:bounds}
\end{listing}

Observações semelhantes podem ser feitas para outros tipos de
restrições aritméticas. Para restrições do tipo $X \neq Z$ e $X \neq
min(Z, Y)$, isso é especialmente simples de ser feito. Para restrições
não lineares do tipo $X < Z\times Y$, isso pode ser especialmente
complicado, ainda mais se $Z$ e $Y$ puderem assumir valores positivos
e negativos, mas ainda pode ser feito.

Como é meio chato escrever uma regra para cada caso de restrição
dessas para a manutenção de consistência por limites, o que é mais
usual em um sistema que ofereça esse tipo de consistência é que
suporte apenas uma quantidade reduzida dessas restrições, sendo as
demais transformadas em versões equivalentes às quais essas restrições
se apliquem, o que não é difícil de se fazer.  Isso, no entanto, está
sujeito ao potencial inconveniente de que restrições equivalentes mas
escritas de formas diferentes podem oferecer oportunidades diferentes
para a diminuição de domínio de cada restrição e a reescrita pode
tornar um domínio que poderia originalmente ser grandemente
simplificado, em um que sofra apenas uma pequena alteração (e o
usuário pode ficar frustrado ao ver que uma diminuição de domínio
``óbvia'' não foi feita).

Apesar disso, a aplicação de consistência por limites frequentemente é
útil. Um programa que realiza essa aplicação é simples de se fazer:
ele toma cada restrição primitiva e os domínios de suas respectivas
variáveis e aplica um algo como o mostrado no código
\ref{lst:bounds}. Assim, temos um mecanismo de busca
incompleto. Torná-lo um mecanismo completo é simples e pode ser feito
com a adição do \technical{backtracking}.


\subsection{Consistência global}

Consistência por limites também podem ser aplicadas a restrições em
duas ou em uma variável, mas, nesse caso, consistência por arco ou por
nó costumam resultar em diminuição maior nos domínios. Um problema
geral, no entanto, pode ser composto por uma combinação de restrições
de diferente aridade, tornando vantajosa a aplicação de diferentes
noções de consistência.

No entanto, um potencial problema com abordagens baseadas em
consistência é que elas consideram apenas uma ou um pequeno número de
restrições e variáveis por vez, enquanto que frequentemente muitas
restrições e variáveis oferecem informações sobre as demais.

Tome, por exemplo, a restrição \codigo{X $\neq$ Y, Y $\neq$ Z, X
  $\neq$ Z}, equivalente a dizer que as variáveis X, Y e Z são todas
diferentes. Dos métodos de consistência que vimos, o mais indicado a
essa restrição é o de consistência em arco, já que $\neq$ é de aridade
2. Mas esse método é muito fraco para restrições de desigualdade, o
que significa que, na prática, resolveríamos isso por
\technical{backtraking}, que tem um crescimento assintótico
exponencial.% TODO adicionar exemplo

Restrições desse tipo são tão usadas que receberam o nome especial de
\funtor{alldiferent/1}\footnote{Na verdade, a aridade dessa restrição
  pode ser arbitrária, mas, por simplicidade, assumimos que as
  variáveis que devem ser diferentes entre si estão organizadas em uma
  lista, tornando a aridade igual a um.} em vários sistemas CLP e é,
talvez, a restrição mais estudada.  Restrições que fazem uso da
informação no domínio de muitas variáveis para realizar a atualização
de cada domínio são chamadas restrições globais (como notado
anteriormente). No caso do \funtor{alldifferent/1}, podemos notar que
essa restrição é equivalente ao supramencionado problema de
correspondência bipartida e que existem algoritmos eficientes para
lidar com ele.

Assim como nas restrições vistas anteriormente, os domínios das
variáveis em \funtor{alldifferent/1} são importantes na decisão de
como resolvê-la. Em particular, se as variáveis são inteiras um
algoritmo de propagação baseado em consistência por limites atinge um
bom desempenho. Se as variáveis não são inteiras, no entanto, uma
algoritmos baseados em consistência por hiper-arco podem ser usados. A
seguir é apresentada os fundamentos de um tal algoritmo. Esse material
é baseado em \cite{basileos}.

\subsubsection{Alldifferent}

Precisaremos das seguintes definições:

\begin{definition}
  Um \definicao{grafo} é uma tupla $G = (V,A)$, de vértices e arestas
  (V é um conjunto de vértices e A de arestas).  Em um grafo não
  orientado, uma aresta é uma tupla de vértices.
\end{definition}


Dado um grafo $G = (V,E)$, um \definicao{pareamento}\footnote{Também
  comumente conhecido como {\it matching}.} M em G é um conjunto de
arestas cujos vértices aparecem em apenas uma aresta de M (em outras
palavras, M é um pareamento em G se os vértices em (V,M) tem no máximo
grau 1).

\begin{definition}
  Um pareamento M em G é dito máximo se para todos os pareamentos P de
  G, $|P| \leq |M|$.
\end{definition}

A teoria de pareamento é relevante para nosso problema porque, para
qualquer restrição $r/n$ com variáveis $X_j$, de domínios $D_j$, a
informação de que $X_j \in D_j$ pode ser expressa por um grafo
bipartido $(\cup_{j=1}^nX_i\cup(\cup_{i=1}^nD_i),E)$, onde $(X_i,d)
\in E \Leftrightarrow d \in D_i$. Esse grafo é conhecido como
\definicao{grafo valor} e pode ser construído em tempo polinomial.

É fácil ver que a restrição \funtor{alldifferent/n} tem solução se e
somente se existe um pareamento máximo de tamanho n em seu grafo
valor. Incidentalmente, existe um algoritmo que, dado um grafo,
encontra um pareamento máximo em $O(\sqrt[2]{|X|}\times|E|)$.

Ademais, dado um pareamento máximo, existem algoritmos eficientes que
tornam \funtor{alldifferent/n} hiper arco consistente (cada aresta em
um pareamento máximo corresponde a uma atribuição de valor a uma
variável da restrição).

Para desenvolvermos essas afirmações um pouco melhor, as seguintes
definições serão uteis.

\begin{definition}
  Dado um pareamento P em G, um vértice $e$ em G é dito pareado se
  $e\in P$, ou livre caso contrário.
\end{definition}

\begin{definition}
  Um pareamento P em G é dito uma \definicao{cobertura} para os
  vértices do G se todo $v$ vértice de G pertence a P.
\end{definition}

\begin{definition}
  Dado um pareamento P em G, um \definicao{caminho (ou ciclo)
    alternante} de G é um caminho (ou ciclo) cujos vértices são
  alternadamente pareadas e livres.
\end{definition}

\begin{theorem}
  Um vértice pertence a algum pareamento máximo P de tamanho n se, e
  somente se, para qualquer pareamento máximo P', ele ou pertence a P'
  ou a um caminho alternante em P' de comprimento par que começa em um
  vértice livre em P', ou a um ciclo alternante em P' de comprimento
  par.\footnote{Se esse teorema não soa intuitivo, você pode fazer uns
    desenhos e se convencer de sua veracidade (a prova real não vai
    ser muito diferente dos desenhos que fizer).}
\end{theorem}

Disso segue que quando uma aresta não pertencer a algum circuito ou
caminho alternante, podemos extraí-la do grafo original.  Para
sabermos se tal ocasião acontece, podemos transformar o grafo em um
grafo direcionado bipartido G' da seguinte forma: dado G=(V,A), nas
arestas em A que são pareadas com P a aresta é orientada das variáveis
para os valores e, nas demais, dos valores ara as variáveis.

Assim, podemos fazer uso do seguinte:

\begin{theorem}
  Todo circuito direcionado de G' tem comprimento par e corresponde a
  um circuito alternante par de G. Além disso, todo caminho em G' que
  for ímpar pode ser estendido em um caminho par, que corresponde a um
  caminho alternante par de G começando em um vértice livre.
\end{theorem}

Para achar os caminhos procurados, podemos, então determinar os
componentes fortemente conectados de G', assim como os caminhos
direcionados em G' começando num vértice livre.

Um resumo do algoritmo é como se segue\footnote{Tem algumas sutilezas
  nele não comentadas aqui. Para saber mais, veja a bibliografia (em
  particular, \cite{basileos} e, em \cite{cristo}, o Capítulo
  \foreign{Algorithms for matching})}:

\begin{enumerate}
\item É dada a restrição \codigo{alldifferent(X)}, onde $X = [X_1,$
  ..., $X_n]$;
\item Construímos então o grafo valor $G =
  (\cup_{j=1}^nX_i\cup(\cup_{i=1}^nD_i),E)$ ;
\item Computamos algum pareamento máximo de G, P;
\item Se $|P| < |X|$, falha;
\item Caso contrário, construa G';
\item Marque as arestas de G' que correspondem a arestas de G
  pertencentes a P como consistentes;
\item Encontre os componentes fortemente conectados de G' e marque
  as arestas nesses componentes como consistentes;
\item Encontre os caminhos direcionados que começam em um vértice
  livre e marque suas arestas como consistentes;
\item Para cada aresta de G' não marcada como consistente, remova a
  aresta correspondente em G.
\end{enumerate}



\subsection{Indexicals}

Como deve ter dado para notar, uma operação chave em CLP(FD) é a
propagação de restrições.

Em sistemas CLP(FD) iniciais, como o CHIP, as restrições eram primeiro
transformadas em termos de forma canônica e, então, executados em um
interpretador que, entre outras coisas, realiza a propagação de
restrições (veja \cite{zhou}).  Devido ao sucesso na compilação de
programas Prolog para a \enphasis{máquina abstrata de Warren} (ou WAM,
da sigla em inglês), que é o tipo de máquina abstrata mais usada na
compilação de programas em Prolog, foram feitos esforços para a
compilação de restrições em CLP(FD) nos mesmos moldes (sendo, na
prática, uma extensão da WAM).

Nesse modelo, as restrições são traduzidas para códigos de baixo nível
de modo que formas de propagação especializadas sejam usadas para cada
tipo de restrição. Isso foi chamado de ``modelo caixa-preta'', já que
o programador não sabe, a princípio, que tipo de propagação seria
realizada em suas restrições. Na prática, esse modelo não se provou
flexível o suficiente e foi abandonado.

Uma construção chamada \definicao{indexicals}, mais flexível do que a
anterior, foi então criada para auxiliar na compilação de restrições
em CLP(FD). O modelo com \technical{indexicals} é dito de ``caixa de
vidro'', denotando seu caráter intermediário entre ``o programador
dita como as restrições são tratadas'' e o seu contrário. No nosso
contexto, um \technical{indexical} é algo da forma \codigo{X in S},
onde X é uma variável de domínio finito e S é uma expressão.

Por exemplo, uma regra de propagação em consistência por limites para
a restrição \codigo{X = Y + Z.}  em \technical{indexical} é:

$X$ in $min(Y)$ + $min(Z)..max(Y)$ + $max(Z)$

A ideia é descrever o domínio de cada variável como uma função do
domínio inicial por meio de construções como \technical{in} e
\technical{min(Z)..max(Y)}.

\technical{Indexicals} não são usados no \eclipse, mas são usados em
outros sistemas e algumas outras formas de propagação fazem uso de
métodos que tomam \technical{indexicals} por base, por isso o
discutimos brevemente por nesta ocasião.

Incidentalmente, os \technical{indexicals} usados aqui provém de um
conceito mais geral de \technical{indexical} proveniente da filosofia
da linguagem.  Para entender esse conceito um pouco melhor, considere
a seguinte situação (retirada de \cite{raymond}):

Digamos que existem dois irmãos gêmeos, um que só diz a verdade e
outro que só diz a mentira. Além disso, o primeiro irmão, que só diz
a, verdade também tem uma crença perfeita do que é ou não verdade
(isto é, todas as proposições que são verdadeiras ele crê serem
verdade e analogamente o contrário). Em contrapartida, o outro irmão
tem uma crença completamente imperfeita do que é verdade (o que é
verdade ele crê não o ser e analogamente o contrário). Note que, ao
serem feitos a mesma pergunta, os dois irmãos dariam a mesma
resposta. Considerando essa situação e tendo em mente que o irmão
mentiroso o deve dinheiro, um logicista pergunta a outro logicista
``Você acha que é possível, fazendo perguntas de sim ou não, descobrir
se um irmão é o mentiroso ou o verdadeiro?'', ao que ele responde
``Claramente não, já que eles dariam as mesmas respostas às mesmas
questões''.  Você acha que o segundo logicista estava correto?\\ Na
verdade, não estava: Considere a pergunta ``Você é o que diz a
verdade?''. Se ele o for, dirá que é. Se não o for, crerá que o é e
dirá que não o é. Apesar disso, o segundo logicista estava correto ao
dizer que eles dariam as mesmas respostas às mesmas questões: o
``você'' em ``Você é o que diz a verdade?'' se refere a ``coisas''
diferentes se usados com pessoas diferentes, então a pergunta feita ao
mentiroso seria fundamentalmente diferente da feita ao não mentiroso.
Aqui, ``você'' é um \technical{indexical}.

O ponto é que um mesmo \technical{indexical} pode significar coisas
diferentes em contextos diferentes (na verdade, do ponto de vista
filosófico, é isso que o define como sendo um
\technical{indexical}). No exemplo de \technical{indexical} acima, os
domínios iniciais de X, Y e Z poderiam mudar significativamente o que
aquele \technical{indexical} significa.


\begin{thebibliography}{1}

\bibitem{basileos} Basileos Anastasatos ``Propagation Algorithms
  for the Alldifferent Constraint''

\bibitem{choi} C.W. Choi, W. Harvey, J.H.M. Lee, and P.J. Stuckey,
  ``Finite Domain Bounds Consistency Revisited''

\bibitem{cristo} Christos Papadimitriou (1998), ``Combinatorial
  Optimization, Algorithms and Complexity'', Dover Publications

\bibitem{raymond} Smullyan, Raymond, ``5000 B.C. and Other
  Philosophical Fantasies''

\bibitem{tsang} E. Tsang (1930), ``Foundations of Constraint
  Satisfaction'', Academic Press.

\bibitem{zhou} Neng-Fa Zhou (2006), ``Programming Finite-Domain
  Constraint Propagator in Action Rules'', Theory and Practice of
  Logic Programming, Vol. 6, No. 5, pp.483-508, 2006

\end{thebibliography}


%\end{document}
