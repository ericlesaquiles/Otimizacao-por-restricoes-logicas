%\documentclass{article}

%\usepackage[utf8]{inputenc}
\usepackage{placeins}
\usepackage{graphicx}
\usepackage[ruled,vlined,linesnumbered]{algorithm2e}
\usepackage{listings}
\usepackage{amsthm}
\usepackage{amsmath}
\usepackage{caption}
\usepackage{epigraph}

\captionsetup[figure]{font=scriptsize}

\newtheorem{definition}{Definição}[section]
\theoremstyle{remark}
\newtheorem*{remark}{Observações}
\theoremstyle{theorem}
\newtheorem{theorem}{Teorema}[section]

\setlength{\parskip}{.5em}

\renewcommand\refname{Leituras adicionais}
\renewcommand\figurename{Figura}
\renewcommand{\lstlistingname}{Código}

\newcommand{\eclipse}{$ECL^iPS^e$}
\newcommand{\definicao}[1]{\textbf{#1}\marginpar{\small \textbf{#1}}}

\lstdefinestyle{prosty}{
  language=Prolog,
  basicstyle=\small
}



%\begin{document}

\section{Próximos passos}

Para finalizar, vamos explorar aqui alguns tópicos que não puderam ser
explorados no resto do texto, na tentativa de mostrar algumas direções
em que é possível caminhar, além das já apresentadas. Começamos com
\technical{CLP(SMT)}.

Antes, convém começarmos com problemas \technical{SAT} \footnote{SAT
  vem de \foreign{satisfiability}.}. SAT o problema de
determinar se existe uma interpretação (atribuição de
valores \technical{booleanos} a variáveis) que
satisfaça uma fórmula \technical{booleana}.Foi o primeiro
problema a ser provado NP-completo mas, apesar disso, tem se mostrado
de grande utilidade para diversos problemas práticos. Como é de se
esperar da formulação, SAT é um problema central na ciência da
computação. Assim, tem recebido muita atenção e conta com
implementações eficientes para grandes classes de problemas
específicos.

Uma instância de SAT é o de obter atribuições para as variáveis $A, B$
e $C$ tal que a fórmula $A \wedge B \vee (\neg C \vee (A \wedge \neg B))$.

\technical{SMT}\footnote{SMT vem de
  \foreign{satisfiability modulo theories}.} é uma generalização do
problema SAT para lidar com outras teorias, de forma mais geral, como
sobre inteiros, listas, vetores, dentre outros. Semelhante ao caso já
visto com problemas em CLP, um \foreign{solver} SMT terá, no caso
geral, algorítmos especializados para lidar com teorias diferentes
(lembre-se que uma das vantagens de CLP é facilitar essa unificação de
diferentes métodos). Devido em grande parte a sua generalidade e
eficiência, o uso de \foreign{solvers} SMT tem ganho grande
popularidade para se lidar com problemas de diversas áreas, como de
análise de programas\cite{zheng}, síntese de programas\cite{beyene},
verificação de hardware\cite{kroenig}, automação de design
eletrônico\cite{kroenig}, segurança de computadores\cite{vanegue}, IA,
pesquisas operacionais (MAXSAT)\cite{li} e biologia\cite{yordanov}.

Existem atualmente vários \foreign{solvers} de SMT, dos quais os mais
populares são Z3, criado pela Microsoft e CVC4\footnote{De
  \foreign{cooperating validity checking}.}, mantido por um grupo de
pesquisadores (em grande parte, da Universidade de Stanford). Neste
texto, focaremos no Z3\footnote{Explicaremos apenas o básico para que
  seja compreensível. Para mais informações, cheque, por exemplo,
  \cite{z3}.}, um \foreign{solver} considerado
estado-da-arte, que se tornou um projeto de código aberto em 2012
(veja \cite{leo} para detalhes).

Para que se tenha alguma ideia de como se usa um \foreign{solver} SMT,
segue um exemplo (retirado de \cite{tuto}) de código para
Z3\footnote{Vale notar, ele está expresso em \technical{Sexps}, mas
  não em Scheme.}:

\begin{lstlisting}
; declare a mutually recursive parametric datatype
(declare-datatypes (T) ((Tree leaf (node (value T)
                                         (children TreeList)))
                        (TreeList nil (cons (car Tree)
                                         (cdr TreeList)))))
(declare-const t1 (Tree Int))
(declare-const t2 (Tree Bool))
; we must use the 'as' construct to distinguish the leaf
; (Tree Int) from leaf (Tree Bool)
(assert (not (= t1 (as leaf (Tree Int)))))
(assert (> (value t1) 20))
(assert (not (is-leaf t2)))
(assert (not (value t2)))
(check-sat)
(get-model)
\end{lstlisting}

\noindent que avalia para:

\begin{lstlisting}
sat
(model
  (define-fun t2 () (Tree Bool)
    (node false (as nil (TreeList Bool))))
  (define-fun t1 () (Tree Int)
    (node 21 (as nil (TreeList Int)))))
\end{lstlisting}

O \enphasis{sat} é de \foreign{satisfyable}, a resposta ao
\codigo{(check-sat)} (seria \enphasis{unsat}, se o modelo posto não
fosse satisfazível).

\subsection{Breve Introdução a SMT e sobre o problema da estratégia}

\foreign{Solvers} SMT se baseiam fortemente em \foreign{solvers} SAT,
baseados em DPLL \footnote{O algorítmo de
  Davis–Putnam–Logemann–Loveland é um algorítmo de busca completo,
  baseado em \foreign{backtracking}, para decidir a se dadas fórmulas
  de lógica proposicional em forma normal conjuntiva podem ser
  satisfeitas, introduzido em 1962, que ainda forma a base de
  \foreign{solvers} eficientes para SAT.}. Para SMT, temos o DPLL(T),
, que é um formalismo para descrever como \foreign{solvers} da teoria
T devem ser integrados com os \foreign{solvers} SAT.

Uma dificuldade é que, para um \foreign{solver} SMT de alta
performance, parte importante da implementação não está no esquema
formal do DPLL(T), mas sim na forma de heurísticas. Essas heurísticas
são frequentemente projetadas para funcionar muito bem para algumas
classe de problemas, tendendo a funcionar mal para outra classes. À
medida que \foreign{solvers} SMT ganham a atenção de cientistas e
engenheiros passou a se tornar claro que isso é um problema, porque há
uma necessidade de grande controle sobre o \foreign{solver} para que
ele se comporte de forma eficiente, o que 
significa expor até centenas de parâmetros para que usuários
decidam quais heurísticas usar e como. Esse é um problema posto por
Leonardo de Moura, um criador do Z3, no artigo \cite{moura}. Nesse
artigo, ele define o problema da estratégia como o de prover ao
usuário meios adequados de dirigir a busca. Mais em geral, ele define
estratégia (neste contexto) como ``\foreign{adaptations of general
  search mechanisms which reduce the search space by tailoring its
  exploration to a particular class of problems}''\cite{moura} e, assim, o
problema da estratégia se traduziria como o de prover uma linguagem
adequada para o usuário realizar sua busca.

Existe mais de uma forma de buscar resolver esse problema. A abordagem
que tomaremos aqui é a adotada por Nada Amin e William Byrd, de juntar
SMT com CLP, em um CLP(SMT), usando miniKanren como base. Para checar
esse trabalho, cheque \cite{namin} ou \cite{namim} (esse último, na
linguagem Clojure). Vale notar que, no momento da escrita deste texto
(em Outubro de 2018), isso é trabalho em progresso.

A ideia é pensar no \foreign{solver} SMT como um implementador de
restrições de baixo nível, com o qual o usuário interaje com o
miniKanren, de forma mais abstrata.



% \subsection{Conclusão}

% Com essa última seção, esperamos ter conseguido passar uma ideia de
% como o \foreign{framework} CLP pode ser extendido positivamente e de
% como podemos usar as ideias desenvolvidas até aqui para criar algo
% que, em certo sentido, se torna uma linguagem que se molda ao
% problema. 


Se você prestou atenção no caminho até aqui, deve ter adquirido algum
conhecimento básico sobre programação por restrições e, assim
esperamos, quando e se precisar de fazer uso de algumas das 


\begin{thebibliography}{99}

\bibitem{beyene} Beyene, Tewodros A., Swarat Chaudhuri, Corneliu
  Popeea, e Andrey Rybalchenko
  ``Recursive games for compositional program synthesis.''
  Working Conference on Verified Software: Theories, Tools, and
  Experiments, pp. 19-39. Springer, Cham, 2015.

\bibitem{moura} De Moura, Leonardo, and Grant Olney Passmore.
  ``The strategy challenge in SMT solving.''
  Automated Reasoning and Mathematics. Springer, Berlin, Heidelberg,
  2013. 15-44.
  
\bibitem{li} Li, Yi, Aws Albarghouthi, Zachary Kincaid, Arie
  Gurfinkel,  Marsha Chechik.
  ``Symbolic optimization with SMT solvers.''
  In ACM SIGPLAN Notices, vol. 49, no. 1, pp. 607-618. ACM, 2014.

\bibitem{kroenig} Mukherjee, Rajdeep, Daniel Kroening, and Tom
  Melham
  ``Hardware verification using software analyzers.'' In VLSI
  (ISVLSI), 2015 IEEE Computer Society Annual Symposium on,
  pp. 7-12. IEEE, 2015.

\bibitem{namin} CLP(SMT) miniKanren:
  \url{https://github.com/namin/clpsmt-miniKanren}

\bibitem{namim} Explorations in logic programming:
  \url{https://github.com/namin/logically}
  
\bibitem{trindade} Trindade, Alessandro B., Lucas C. Cordeiro.
  ``Applying SMT-based verification to hardware/software partitioning
  in embedded systems.''
  Design Automation for Embedded Systems 20, no. 1 (2016): 1-19.

\bibitem{tuto} Z3 Tutorial: \url{https://rise4fun.com/z3/tutorial}
  
\bibitem{vanegue} Vanegue, Julien, Sean Heelan, Rolf Rolles.
  ``SMT Solvers in Software Security.'' WOOT 12 (2012): 9-22.
  
\bibitem{yordanov} Yordanov, Boyan, Christoph M. Wintersteiger,
  Youssef Hamadi, Hillel Kugler.
  ``Z34Bio: An SMT-based framework for analyzing biological
  computation.''. SMT’13 (2013).

\bibitem{zheng} Zheng, Yunhui, Xiangyu Zhang, e Vijay Ganesh.
  ``Z3-str: a z3-based string solver for web application analysis.'',
  Proceedings of the 2013 9th Joint Meeting on Foundations of
  Software Engineering, pp. 114-124. ACM, 2013. 

\bibitem{leo} Z3 se torna \foreign{open source}:
  \url{http://leodemoura.github.io/blog/2012/10/02/open-z3.html}
  
\bibitem{z3} Z3 wiki: \url{https://github.com/Z3Prover/z3/wiki}
  
\end{thebibliography}

%\end{document}
