\documentclass{article}

\usepackage[utf8]{inputenc}
\usepackage{placeins}
\usepackage{graphicx}
\usepackage[ruled,vlined,linesnumbered]{algorithm2e}
\usepackage{listings}
\usepackage{amsthm}
\usepackage{amsmath}
\usepackage{caption}
\usepackage{epigraph}

\captionsetup[figure]{font=scriptsize}

\newtheorem{definition}{Definição}[section]
\theoremstyle{remark}
\newtheorem*{remark}{Observações}
\theoremstyle{theorem}
\newtheorem{theorem}{Teorema}[section]

\setlength{\parskip}{.5em}

\renewcommand\refname{Leituras adicionais}
\renewcommand\figurename{Figura}
\renewcommand{\lstlistingname}{Código}

\newcommand{\eclipse}{$ECL^iPS^e$}
\newcommand{\definicao}[1]{\textbf{#1}\marginpar{\small \textbf{#1}}}

\lstdefinestyle{prosty}{
  language=Prolog,
  basicstyle=\small
}



\begin{document}

\section{Próximos passos}

Para podermos melhorar nosso campo de atuação, é importante sabermos
usar as ferramentas de que dispomos para estender outras de modo
adequado, abrindo espaço para novas aplicações. Como conclusão deste
texto, buscaremos mostrar isso com um breve exemplo, de extensão de
\foreign{solvers} para SMT por CLP.

Antes, convém conhecer o problema \technical{SAT} \footnote{SAT
  vem de \foreign{satisfiability}.}. SAT é o problema de
determinar se existe uma interpretação (atribuição de
valores \technical{booleanos} a variáveis) que
satisfaça uma fórmula booleana. Foi o primeiro
problema a ser provado NP-completo\cite{schaefer} mas, apesar disso,
tem se mostrado
de grande utilidade para diversos problemas práticos. Como é de se
esperar da formulação, SAT é um problema central em ciência da
computação. Assim, tem recebido muita atenção e conta com
implementações de \foreign{solvers} eficientes para grandes classes de
sub-problemas mais específicos.

Uma instância de SAT é o de obter valores booleanos para
$A, B$ e $C$ tal que a fórmula $A \wedge B \vee (\neg C \vee (A \wedge \neg B))$ seja
satisfeita.

\technical{SMT}\footnote{SMT vem de
  \foreign{satisfiability modulo theories}.} é uma generalização do
problema SAT para lidar com outras teorias e de forma mais
geral. Exemplos de teorias abrangidas por SMT são sobre inteiros,
listas e vetores\footnote{Existem várias outras, mas as teorias
  específicas utilizadas variam de \foreign{solver}.}. Uma instância
de problema SMT é como uma de problema SAT, exceto que algumas (ou
todas) as variáveis booleanas são substituídas por predicados sob
variáveis não-binárias, como em $(3 + X \leq 2) \wedge (4Y - \pi \geq X)$.

Semelhante ao caso já visto com problemas em CLP, um \foreign{solver}
para SMT terá, no caso geral, algoritmos especializados para lidar com
teorias diferentes (lembre-se que uma das vantagens de CLP é facilitar
essa unificação de diferentes métodos). Devido em grande parte à sua
generalidade e à eficiência desenvolvida (depois de anos de pesquisa),
o uso de \foreign{solvers} para SMT tem ganho grande
popularidade para lidar com problemas de diversas áreas, como de
análise de programas\cite{zheng}, síntese de programas\cite{beyene},
verificação de hardware\cite{kroenig}, automação de design
eletrônico\cite{kroenig}, segurança de computadores\cite{vanegue}, IA,
pesquisa operacional\cite{li} e biologia\cite{yordanov}.

Existem atualmente vários \foreign{solvers} para SMT, dos quais os mais
populares são Z3 da Microsoft e CVC4\footnote{De
  \foreign{cooperating validity checking}.}, mantido por um grupo de
pesquisadores (em grande parte, da Universidade de Stanford). Neste
texto, focaremos no Z3\footnote{Explicaremos apenas o básico para que
  seja compreensível. Para mais informações, cheque, por exemplo,
  \cite{z3}.}, um \foreign{solver} considerado
estado-da-arte, que se tornou um projeto de código aberto em 2012
(veja \cite{leo} para detalhes).

Para que se tenha alguma ideia de como se usa um \foreign{solver} para SMT,
segue um exemplo (retirado de \cite{tuto}) de código para
Z3\footnote{Vale notar, ele está expresso em \technical{Sexps}, mas
  não em Scheme.}:

\begin{lstlisting}
; declare a mutually recursive parametric datatype
(declare-datatypes (T) ((Tree leaf (node (value T)
                                         (children TreeList)))
                        (TreeList nil (cons (car Tree)
                                         (cdr TreeList)))))
(declare-const t1 (Tree Int))
(declare-const t2 (Tree Bool))
; we must use the 'as' construct to distinguish the
; leaf (Tree Int) from leaf (Tree Bool)
(assert (not (= t1 (as leaf (Tree Int)))))
(assert (> (value t1) 20))
(assert (not (is-leaf t2)))
(assert (not (value t2)))
(check-sat)
(get-model)
\end{lstlisting}

\noindent que avalia para:

\begin{lstlisting}
sat
(model
  (define-fun t2 () (Tree Bool)
    (node false (as nil (TreeList Bool))))
  (define-fun t1 () (Tree Int)
    (node 21 (as nil (TreeList Int)))))
\end{lstlisting}

O \enphasis{sat} é de \foreign{satisfyable}, a resposta ao
\codigo{(check-sat)} (seria \enphasis{unsat}, se o modelo posto não
fosse satisfazível). SMT é um problema mais geral do que SAT (note
que SAT é um caso particular de SMT), que é
NP-completo. Assim, como é de se esperar, existem vários problemas
para os quais ele não funciona muito bem. Então, parte do desafio ao
(buscar) resolver um problema SMT está em expressá-lo de forma que
fique na ``parte fácil'' para o \foreign{solver}.

\subsection{Breve introdução a SMT e sobre o problema da estratégia}

\foreign{Solvers} para SMT se baseiam fortemente em \foreign{solvers}
para SAT, baseados em DPLL \footnote{O algoritmo de
  Davis–Putnam–Logemann–Loveland é um algoritmo de busca completo,
  baseado em \foreign{backtracking}, para decidir a se dadas fórmulas
  de lógica proposicional em forma normal conjuntiva podem ser
  satisfeitas, introduzido em 1962, que ainda forma a base de
  \foreign{solvers} eficientes para SAT.}. Para SMT, temos o DPLL(T),
que é um formalismo para descrever como \foreign{solvers} da teoria
T devem ser integrados com os \foreign{solvers} SAT.

Uma dificuldade é que, para um \foreign{solver} para SMT de alta
performance, parte importante da implementação não está no esquema
formal do DPLL(T), mas sim em heurísticas. Essas heurísticas
são frequentemente projetadas para funcionar muito bem para algumas
classe de problemas, tendendo a funcionar mal para outras. À
medida que \foreign{solvers} para SMT ganham a atenção de cientistas e
engenheiros, passou a se tornar claro que isso é um problema, porque há
uma necessidade de grande controle sobre o \foreign{solver} para que
ele se comporte de forma eficiente, o que 
significa expor até centenas de parâmetros para que usuários
decidam quais heurísticas usar e como. Esse é um problema posto por
Leonardo de Moura \foreign{et. al}, um criador do Z3, no artigo
\cite{moura}. Nesse artigo, ele define o problema da estratégia como o
de prover ao usuário meios adequados de dirigir a busca (já que, de
uma forma ou de outra, a resolução de uma instância do problema SMT
envolve busca). Mais em geral, ele define estratégia (neste contexto)
como ``\foreign{adaptations of general search mechanisms which reduce
  the search space by tailoring its exploration to a particular class
  of problems}''\cite{moura} e, assim, o problema da estratégia se
traduziria como o de prover uma linguagem adequada que o usuário possa
usar para realizar sua busca.

Existe mais de uma forma de buscar resolver esse problema. A abordagem
que tomaremos (de forma resumida) aqui é a adotada por Nada Amin e
William Byrd, de juntar SMT com CLP, em um CLP(SMT), usando \technical{miniKanren}
como base\footnote{Vale notar, o que implementamos no capítulo passado
  não lida com restrições no sentido usual, mas ele pode ser
  facilmente estendido com relações de CLP. Veja \cite{alvis}.}. Para
checar esse trabalho, veja \cite{namin} ou
\cite{namim} (esse último, na linguagem Clojure). Vale notar que, no
momento da escrita deste texto (em outubro de 2018), isto é trabalho
em progresso.

A ideia é pensar no \foreign{solver} para SMT como um implementador de
restrições de baixo nível, com o qual o usuário interage com o
\technical{miniKanren}, de forma mais abstrata (assim como fizemos com outras
técnicas para otimização e para busca de satisfação de
restrições). Isso pode ser útil de várias formas. Primeiro, nota-se
que \foreign{solvers} para SMT não lidam bem com recursões em geral, o que
não é problema para linguagens de programação lógica ou de programação
por restrições. Assim, por exemplo, se o seguinte for submetido ao Z3,
depois de algum tempo, ele retorna (no momento da escrita deste texto)
\foreign{unknown}:

\begin{lstlisting}
  (declare-fun fact (Int) Int)
  (assert (= (fact 0) 1))
  (assert (forall ((n Int))
    (=> (> n 0) (= (fact n) (* n (fact (- n 1)))))))
  (declare-const r6 Int)
  (check-sat)
\end{lstlisting}

Mas, usando Z3 a partir do \technical{miniKanren}\footnote{O que, já que as
  entradas e saídas são em \technical{Sexps}, não é difícil.} (com o
software em \cite{namin}, por exemplo), podemos fazer o seguinte:\\

\begin{lstlisting}[escapeinside={[}{]}]
(define faco
  ([$\lambda$] (n out)
    ([$cond^e$] ((z/assert `(= ,n 0))
            (z/assert `(= ,out 1)))
           ((z/assert `(> ,n 0))
            (fresh (n-1 r)
              (z/assert `(= (- ,n 1) ,n-1))
              (z/assert `(= (* ,n ,r) ,out))
              (faco n-1 r))))))

(test "faco-7"
  (run 7 (q)
    (fresh (n out)
      (faco n out)
      (== q `(,n ,out))))
  '((0 1) (1 1) (2 2) (3 6) (4 24) (5 120) (6 720)))

\end{lstlisting}

\noindent em que \codigo{(con$d^e$ ($a_1$ ... $a_i$) ... ($f_1$
  ... $f_j$))} pode ser lido logicamente como $(a_1 \wedge ... \wedge a_i) \vee
... \vee (f_1 ... f_j)$ e \codigo{z/assert} faz a interface com o
\foreign{solver}, tomando uma expressão booleana com variáveis (por
padrão, inteiras). Além de \codigo{z/assert}, existem \codigo{z/},
\codigo{z/check} e \codigo{z/purge}, para interagir com o Z3 de forma
diferente.

Ainda não é certo, para qual ou quais tipos de problemas esse tipo de
abordagem é apropriado, ou mesmo se para algum, mas é um exemplo de
tentativa de uso da ideia de CLP para algo diferente, de forma não
óbvia.
           
\subsection{Conclusão}

% Com essa última seção, esperamos ter conseguido passar uma ideia de
% como o \foreign{framework} CLP pode ser extendido positivamente e de
% como podemos usar as ideias desenvolvidas até aqui para criar algo
% que, em certo sentido, se torna uma linguagem que se molda ao
% problema. 


Se você prestou atenção no caminho até aqui, deve ter adquirido algum
conhecimento básico sobre programação por restrições e, assim
esperamos, quando e se precisar de fazer uso de algumas das ideias
desenvolvidas, será capaz de avaliar como fazê-lo. Mais importante
ainda, esperamos que seja capaz de avaliar quando e como pode ``ser
preciso'' fazer uso das ideias aqui desenvolvidas.


\begin{thebibliography}{99}

\bibitem{alvis} Alvis, Claire E., Jeremiah J. Willcock, Kyle
  M. Carter, William E. Byrd, Daniel P. Friedman.
  ``cKanren miniKanren with constraints.'' (2011).
  
\bibitem{beyene} Beyene, Tewodros A., Swarat Chaudhuri, Corneliu
  Popeea, e Andrey Rybalchenko
  ``Recursive games for compositional program synthesis.''
  Working Conference on Verified Software: Theories, Tools, and
  Experiments, pp. 19-39. Springer, Cham, 2015.

\bibitem{moura} De Moura, Leonardo, and Grant Olney Passmore.
  ``The strategy challenge in SMT solving.''
  Automated Reasoning and Mathematics. Springer, Berlin, Heidelberg,
  2013. 15-44.
  
\bibitem{li} Li, Yi, Aws Albarghouthi, Zachary Kincaid, Arie
  Gurfinkel,  Marsha Chechik.
  ``Symbolic optimization with SMT solvers.''
  In ACM SIGPLAN Notices, vol. 49, no. 1, pp. 607-618. ACM, 2014.

\bibitem{kroenig} Mukherjee, Rajdeep, Daniel Kroening, and Tom
  Melham
  ``Hardware verification using software analyzers.'' In VLSI
  (ISVLSI), 2015 IEEE Computer Society Annual Symposium on,
  pp. 7-12. IEEE, 2015.

\bibitem{namin} CLP(SMT) miniKanren:
  \url{https://github.com/namin/clpsmt-miniKanren}

\bibitem{namim} Explorations in logic programming:
  \url{https://github.com/namin/logically}

\bibitem{schaefer} Schaefer, Thomas J.
  ``The complexity of satisfiability problems''
  In: Proceedings of the tenth annual ACM symposium on Theory of
  computing. ACM, 1978. p. 216-226.
  
\bibitem{trindade} Trindade, Alessandro B., Lucas C. Cordeiro.
  ``Applying SMT-based verification to hardware/software partitioning
  in embedded systems.''
  Design Automation for Embedded Systems 20, no. 1 (2016): 1-19.

\bibitem{vanegue} Vanegue, Julien, Sean Heelan, Rolf Rolles.
  ``SMT Solvers in Software Security.'' WOOT 12 (2012): 9-22.
  
\bibitem{yordanov} Yordanov, Boyan, Christoph M. Wintersteiger,
  Youssef Hamadi, Hillel Kugler.
  ``Z34Bio: An SMT-based framework for analyzing biological
  computation.''. SMT’13 (2013).

\bibitem{zheng} Zheng, Yunhui, Xiangyu Zhang, e Vijay Ganesh.
  ``Z3-str: a z3-based string solver for web application analysis.'',
  Proceedings of the 2013 9th Joint Meeting on Foundations of
  Software Engineering, pp. 114-124. ACM, 2013. 

\bibitem{leo} Z3 se torna \foreign{open source}:
  \url{http://leodemoura.github.io/blog/2012/10/02/open-z3.html}
  
\bibitem{z3} Z3 wiki: \url{https://github.com/Z3Prover/z3/wiki}

\bibitem{tuto} Z3 Tutorial: \url{https://rise4fun.com/z3/tutorial}

\end{thebibliography}

\end{document}
