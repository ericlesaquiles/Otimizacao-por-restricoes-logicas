\documentclass{article}

\usepackage[utf8]{inputenc}
\usepackage{placeins}
\usepackage{graphicx}
\usepackage[ruled,vlined,linesnumbered]{algorithm2e}
\usepackage{listings}
\usepackage{amsthm}
\usepackage{amsmath}
\usepackage{caption}
\usepackage{epigraph}

\captionsetup[figure]{font=scriptsize}

\newtheorem{definition}{Definição}[section]
\theoremstyle{remark}
\newtheorem*{remark}{Observações}
\theoremstyle{theorem}
\newtheorem{theorem}{Teorema}[section]

\setlength{\parskip}{.5em}

\renewcommand\refname{Leituras adicionais}
\renewcommand\figurename{Figura}
\renewcommand{\lstlistingname}{Código}

\newcommand{\eclipse}{$ECL^iPS^e$}
\newcommand{\definicao}[1]{\textbf{#1}\marginpar{\small \textbf{#1}}}

\lstdefinestyle{prosty}{
  language=Prolog,
  basicstyle=\small
}



%
% TODO: revisar esta seção!
%

\begin{document}

\section{Restrições passivas e implementações no Eclipse}

Antes de sairmos por aí resolvendo CSPs e COPs, será útil termos a distinção entre restrições ativas
e passivas. Resumidamente, restrições ativas podem alterar o estado das variáveis, enquanto que
restrições passivas por si só não podem e, assim, são mais usadas para fins de testes. Por exemplo,
\codigo{r(a,X) = r(Y,b).} é uma restrição ativa: \var{X} precisa tomar o valor de b, e \var{Y} de a. Mas
\codigo{4*X < Y+2} é, uma restrição passiva, já que \var{X} e \var{Y} precisam estar instanciadas
para a restrição ser usada.

Por enquanto, nos preocuparemos mais com restrições passivas e veremos exemplos de sua utilização no
sistema \eclipse, que é uma expansão suficientemente completa do Prolog para lidar melhor com restrições. Não
cabe darmos aqui uma detalhada mostra do que ele adiciona, mas convém falarmos brevemente
sobre como alguns de seus iteradores funcionam, já que os usaremos daqui em diante.

Nossa exposição é baseada primariamente em \cite{joachim} e em \cite{schimpf}. Na realidade,
apesar de existirem muitos iteradores diferentes no \eclipse, todos são feitos com base na mesma
construção, chamada \funtor{do/2}. Ademais, apenas uma das especificações de iteradores é mesmo
fundamental. Uma chamada \codigo{( fromto(From, In, Out, To) do Body ).} é traduzida como:

    \begin{listing}
\inputminted{prolog}{../Exemplos/Cap8/prog8_fromto.pl}
\caption{Fromto}
    \end{listing}

\noindent É importante notar que, com \technical{fromtos} aninhados, cada um mapeia a um acumulador
diferente.

A partir desse iterador, podemos criar outros (que, na verdade, são abreviações). Por exemplo,
\codigo{( foreach(X,Lista) do Body ).} é uma abreviação de \codigo{( fromto(Lista, [X|Xs], Xs, []) do
  Body ).}, e \codigo{(foreacharg(X,S) do Body).} é uma abreviação para \codigo{N1 is arity(S) + 1,
  ( fromto(1,I,I1,N1), param(S) do arg(I,S,X), I1 is I+1, Body ).}


Enquanto em Prolog o único método iterativo é a recursão, no \eclipse\ dispomos de algumas opções a mais. Em particular, temos:

\begin{itemize}
  \item \codigo{foreach(El, Lista) do busca(El).}
    \\ Itera Busca(El) ordenadamente sobre cada elemento El de Lista;
  \item \codigo{fromto(Prim, In, Out, Ult) do busca(In, Out).}
    \\ Itera Busca(In, Out) de In = Prim até Out = Ult.
\end{itemize}

Existem diversos outros iteradores para propósitos diferentes, todos eles seguindo o padrão
\enphasis{(iterador \codigo{do} busca)}. Iteradores podem ser postos em conjunto como \enphasis{(iterador1, iterador2, ..., iteradorn do busca)}. Ao fazer isso, todos os iteradores dão o passo junto, por assim dizer, e o conjunto de iteradores para quando qualquer um deles chegar ao fim. Também podemos aninhar iteradores (como se colocássemos um \technical{for} dentro de outro em uma linguagem
convencional) da seguinte forma: \enphasis{(iterador1 \codigo{do} (iterador2 \codigo{do} ... (iteradorn
  \codigo{do} busca)))}.

\subsection{Backtracking no Prolog/Eclipse}

Já discutimos rapidamente a busca por \technical{backtracking} antes, agora veremos como implementá-la. Para tanto, precisamos decidir qual será o método de ramificação usado, qual a ordenação das variáveis e qual a ordenação dos valores de cada variável.

O método de ramificação usado aqui será o \technical{labelling}. O próximo passo é decidir a ordem das variáveis. Isso pode ter um grande impacto na busca, apesar de a quantidade de folhas na árvore de busca continuar sendo a mesma para qualquer ordem: a diferença está na quantidade de nós internos na árvore. Por exemplo, para um CP nas variáveis X e Y, em que X pode tomar dois valores e Y pode tomar 3 valores diferentes, a quantidade de folhas na árvore de busca é 3$\times$4=12. Fazendo o
\technical{labelling}
do X antes do Y, temos dois nós internos, enquanto que, fazendo o \technical{labelling} do Y antes do X, temos três nós internos. A presença de maior quantidade de nós internos na árvore de busca torna a busca mais difícil, sendo razão razoável para que busquemos fazer antes o \technical{labelling} das variáveis com menor domínio. Veremos depois que, na presença de restrições ativas, a ordenação das variáveis pode ter grande influência no desempenho do algoritmo.

Mencionamos que uma forma de descrever a escolha de valores de forma mais geral é por meio de alocação de crédito. Uma parte de um programa que implementa essa ideia é o seguinte, que assume que os valores de cada domínio já estão ordenados segundo uma preferência:

    \begin{listing}[H]
\inputminted{prolog}{../Exemplos/Cap8/prog2_busca2.pl}
\caption{Busca 0}
    \end{listing}

Ele assume que Lista é uma lista de pares variável-domínio e precisa ser completada pelas escolhas
de \funtor{escolhe\_var/3} e \funtor{compartilha\_credito/3}. O seguinte exemplo de \funtor{compartilha\_credito/3} corresponde à escolha dos N primeiros valores, se N for menor
que o tamanho do domínio; ou do domínio todo,
caso contrário:

    \begin{listing}[H]
\inputminted{prolog}{../Exemplos/Cap8/prog3_partilha.pl}
\caption{Partilha 0}
    \end{listing}

Essa escolha ocorre atribuindo aos primeiros N valores do domínio o mesmo crédito, de N. Outra escolha de compartilha\_crédito, possivelmente mais natural, é a que envolve a atribuição de N créditos ao primeiro valor, N/2 ao segundo, e assim por diante:

    \begin{listing}[H]
\inputminted{prolog}{../Exemplos/Cap8/prog4_partilha2.pl}
\caption{Partilha 1}
    \end{listing}

Nesse código, o \codigo{fix(ceiling(AtuCredito/2))} retorna o maior inteiro menor ou igual que AtuCredito/2.

\subsection{Variáveis não-lógicas}

  Ocasionalmente será útil, como uma medida da eficiência de um programa, quantificar coisas como a quantidade de sucessos em uma computação ou a quantidade de \technical{backtrackings}. Para isso, o \eclipse\ permite a utilização de variáveis não-lógicas e oferece quatro meios de lidar com elas:

  \begin{itemize}
    \item \funtor{setval/2};
    \item \funtor{incval/1};
    \item \funtor{getval/1};
    \item \funtor{decval/1};
  \end{itemize}

  O que define uma variável como não-lógica é que seu valor não muda com o \technical{backtracking}.
  Além disso, variáveis não-lógicas não são capitalizadas e a única forma de mudar ou acessar o
  valor delas é por meio de um dos predicados acima. Segue uma implementação de nosso programa de
  busca que conta a quantidade de \technical{backtrackings}\footnote{Esse \funtor{once/1}, usado no programa, definido como \codigo{once(Goal) :- Goal, !.}}:

    \begin{listing}[H]
\inputminted{prolog}{../Exemplos/Cap8/prog5_busca3.pl}
\caption{Busca 1}
    \end{listing}

Esse programa explora a forma como é feito o \technical{backtracking} e, por isso, a ordem em que foi posta é crucial. Vale notar que ele conta todos os backtrackings que ocorrem na busca, possibilitando contar a quantidade de nós na árvore.

Frequentemente, no entanto, pode ocorrer um \technical{backtracking} entre mais de um nível. Isso
ocorre quando, logo depois de realizar um, é realizado outro \technical{backtracking}. Uma medida
melhor de eficiência pode ser uma contagem de \technical{backtrackings} que conta uma sequência
ininterrupta como sendo apenas um. Um \funtor{conta\_backtracking/0} que faz isso é dado a seguir:

    \begin{listing}[H]
\inputminted{prolog}{../Exemplos/Cap8/prog6_conta_backtrack.pl}
\caption{Backtracking}
    \end{listing}

\subsection{A biblioteca suspend}

Voltando a resoluções de CPs aritméticos e booleanos, introduzimos a biblioteca \eclipse\ \technical{suspend}. A biblioteca \technical{suspend} lida com restrições aritméticas suspendendo a avaliação delas até que as variáveis tenham sido instanciadas e possam ser avaliadas. Caso elas não se tornem instanciadas até o fim da busca, o resultado é uma restrição, que é o que queríamos.

No \eclipse\ existem duas formas de se usar uma biblioteca: pode-se colocar um \codigo{:-library(nome\_da\_biblioteca).} no início do arquivo utilizado ou, ao usar um predicado da biblioteca nome\_da\_biblioteca, colocar \codigo{nome\_da\_biblioteca:( ... ).}. Um exemplo de uso de \technical{suspend} é: \codigo{suspend:(2 < Y + 4), Y = 3.}, que resultaria em erro em Prolog puro. Caso a biblioteca \technical{suspend} já tenha sido carregada, essa restrição pode ser reescrita como \codigo{2 \$< Y + 4, Y = 3.},
em que o \$ indica que a restrição é usada tal como na biblioteca \technical{suspend}.

Essa biblioteca também lida com restrições booleanas (para as quais os valores de variáveis são 0 ou
1 e os símbolos de restrições são tais como \funtor{or/2}, \funtor{and/2}, \funtor{neg/1} e
\funtor{=>/2}, de implicação) e permite a declaração de variáveis de formas distintas.

Uma delas é por meio do \technical{range}: \codigo{suspend(X :: 2..10).}, ou \codigo{suspend(X \#:: 2..10).}
que gera uma variável X cujo valor é restrito ao intervalo de inteiros entre 2 e 10. Se a biblioteca já estiver carregada, que é o que assumiremos daqui para frente, essa restrição pode ser escrita como \codigo{X :: 2..10.}.
Se quisermos usar intervalos reais no lugar de intervalos de inteiros (assumidos como o padrão),
podemos usar um \$ no lugar de \#, como em \codigo{X \$:: 2..10} (vale repetir que o uso de
intervalos inteiros é o padrão, o que significa que, na falta de uma símbolo como  \# ou \$, o
intervalo é entendido como sendo em inteiros).
Alternativamente, pode-se usar a restrição \funtor{integers/1} ou \funtor{reals/1} para restringir a variável ou lista de variáveis a assumir valores nos inteiros ou reais, respectivamente.



A biblioteca \technical{suspend} também permite a criação de suspensões arbitrárias pelo usuário a
partir de \funtor{suspend/3}. O primeiro argumento de \funtor{suspend/3} indica a restrição a ser
suspensa, o segundo indica a prioridade da suspensão (o que nos dá a ordem de execução de restrições
que deixam a suspensão juntas) e o terceiro a condição de saída da suspensão, escrito como
\codigo{Termo -> Condicao} (dizendo que na ocorrência da \var{Condicao}, em relação a Termo, a restrição deixa a suspensão), em que
``var{Condicao}'' geralmente é \technical{inst}, indicando que o Termo é instanciado.
Um exemplo é \codigo{suspend(X =:= 21, 2, X -> inst)}, indicando que a restrição de que X =:= 21 está em estado de suspensão até que X seja instanciada.

Talvez se lembre do programa Ou Exclusivo do Capítulo 6. Com o uso de \funtor{suspend/3}, podemos reimplementá-lo sem recorrer ao \technical{backtracking}:
%\vspace{3cm}

    \begin{listing}[H]
\inputminted{prolog}{../Exemplos/Cap8/prog7_xor.pl}
\caption{XOR}
    \end{listing}

Note que, agora, nosso \funtor{ou\_exclusivo/2} não é mais uma operação aritmética, agindo como um predicado relacional como os demais.

\subsection{Outras Bibliotecas}

Como lidamos, nesta seção, com a biblioteca \technical{suspend}, vale a pena fazermos um breve
comentário sobre as demais bibliotecas. Isso é útil não só pelos motivos práticos (no uso das
bibliotecas), mas também como um resumo das restrições que veremos mais para frente.

\begin{itemize}
  \item A biblioteca \technical{ic} (de \foreign{interval constraint})\footnote{Esse nome vem do
      fato de que essas restrições atuam sobre intervalos, e as operações aritméticas realizadas sob elas são
      operações de intervalos. Veja \cite{hickey} para mais detalhes.} provê um resolvedor de restrições misto inteiro/real.
  \item A biblioteca \technical{branch and bound} provê um \technical{framework} para resolver
    problemas por  \foreign{branch and bound} muito customizável.
  \item A biblioteca \technical{eplex} provê otimização para problemas de LP e MIP (\foreign{linear
    programming} e \foreign{mixed integer programming}, respectivamente). Existem outras bibliotecas
  nomeadas \technical{eplex\_x}, para usar o resolvedor \technical{x} específico (exemplos para
  valores de \technical{x} são ``cplex'' e ``gurobi'').
  \item A biblioteca \technical{ic\_global} provê restrições globais sobre listas de inteiros.
  \item A biblioteca \technical{ic\_symbolic} provê um resolvedor para restrições sobre domínios
    simbólicos ordenados.
  \item A biblioteca \technical{fd} provê um resolvedor para domínios finitos no geral.
\end{itemize}

Citamos algumas das que consideramos importantes. Mais detalhes sobre essas bibliotecas (e sobre as
demais, não citadas aqui) podem ser encontrados em
\url{http://eclipseclp.org/doc/bips/lib/fd/index.html}\footnote{Acessado em 22/02/2018.}

A biblioteca \technical{ic\_global} merece umas palavras a mais. Foi anteriormente mencionada a
distinção entre restrições ativas e passivas, mas existe outra distinção, que às vezes pode ser mais
significativa:

\begin{itemize}
  \item \definicao{Restrições elementares} são as que agem sobre uma quantidade predeterminada de
    variáveis. Elas estão disponíveis, por exemplo, nas bibliotecas \technical{ic} e
    \technical{branch\_and\_bound};
  \item \definicao{Restrições globais} são as que agem sobre uma quantidade indeterminada de variáveis. Elas estão disponíveis, por
    exemplo, na biblioteca \technical{ic\_global} (e também em outras, não citadas aqui).
\end{itemize}

Um exemplo de restrição global é o \funtor{alldiferent/n}, que será vista adiante.

  \begin{thebibliography}{1}

    \bibitem{joachim}
      Schimpf, Joachim,
      \url{https://groups.google.com/forum/?hl=en#!msg/comp.lang.prolog/UYfgRxUWbGo/0KuxfWPEDqoJ}

      \bibitem{hickey}
      T. Hickey and Q. Ju and M. H. van Emden,
      ``Interval Arithmetic: from Principles to Implementation'',
      Journal of the ACM

      \bibitem{schimpf}
        Schimpf, Joachim,
        ``Logical Loops'', IC-Parc, Imperial College, London

        \bibitem{shen}
        Schimpf, Joachim e Shen, Kish,
        ``\eclipse - From LP to CLP'',
        Theory and Practice of Logic Programming


  \end{thebibliography}


%COMMENT: em um lugar voce usou "nao-logica" e nos outros "nao logica". Nao sei qual a regra, para parece fazer mais sentido usar "nao-logica" (ja que isso seria uma palavra soh). Entao mudei em todos os lugares para ficar com o tracinho. Se descobrir que o certo eh sem, tem de tirar de todos os lugares.


\end{document}
