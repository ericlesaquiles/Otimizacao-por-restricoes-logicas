\documentclass{article}
\usepackage[utf8]{inputenc}
\usepackage{placeins}

\usepackage[ruled,vlined,linesnumbered]{algorithm2e}
\usepackage{listings}
\lstset{language=Prolog}
\renewcommand{\lstlistingname}{Código}

\usepackage{amsthm}

%\theoremstyle{definition}
\newtheorem{definition}{Definição}[section]

\theoremstyle{remark}
\newtheorem*{remark}{Remark}

\theoremstyle{theorem}
\newtheorem{theorem}{Teorema}[section]

\setlength{\parskip}{.5em}




%
% TODO: revisar esta seção!
%

\begin{document}
\section{Modelo Computacional de Programas Lógicos}

Considere o seguinte programa:


\lstinputlisting[caption=Circuito]{../Exemplos/Cap1/prog1_circuito.pl}

Qual será o resultado do goal {\tt porta\_and(Entrada1, Entrada2, Saida)?}, a leitora pode se perguntar. Mais do que isso, ela pode se perguntar ``Será que, dado um programa qualquer e um goal qualquer dá para ``calcular'' o resultado do goal?''. Te convido a refletir por alguns momentos sobre essa questão.

%Note que, como o goal consiste contém variáveis (lembre-se que o que for capitalizado em programas lógicos denota uma variável), um resultado não é só \textit{sucesso} ou \textit{falha}, mas
%também, possivelmente, uma substituição para valores das variáveis (isto é, valores para as variáveis tal que o goal tenha sucesso).

A leitora pode imaginar que, se houvesse muitos programas com goals de resultados incalculáveis, programação lógica não seria tão útil e dificilmente teria sido feito um material como este (mais difícil ainda é o material ter sido feito e a leitora estar lendo), então esse não deve ser o caso.

Se o goal estiver expresso no programa apenas como um fato base, prová-lo é fácil: só precisamos checar se algum dos fatos é igual ao goal. Mas, se o goal contiver alguma variável ou só puder ser provado através de alguma regra, que é o caso geral, a situação fica mais complicada.

Se o goal contiver variáveis, para prová-lo o que precisamos é encontrar uma substituição para cada uma delas de forma que cada um dos termos do goal seja logicamente consistente com o programa. Aqui o que queremos dizer com ``substituição'' é que a variável toma o valor de um outro termo. Uma forma de pensar sobre isso é que, antes da substituição, a variável ``tem uma vida só sua'' (ou, é irrestrita) e que, depois, sua vida é, na verdade, ``a vida de
outro'' (ou, é restrita). Mais precisamente, temos:\\

\begin{definition}{Substituição}\\
  Dado um termo {\it p($a_1$, ..., $a_n$)}, onde os $a_j$, para $j \in J$, $J$ algum conjunto indexador, são variáveis, uma substituição é um conjunto $\iota$ de \textit{unificações}, escritas como $a_i = k_j$, onde $k_j$ é uma variável ou um termo atômico e ``='' denota que $a_i$ é idêntica a $k_j$ e dizemos que $a_i$ é unificado com $k_j$.
  Uma substituição $\iota$ sobre um programa P é escrita P$\iota$.
\end{definition}

Convém fazer algumas observações a respeito do que foi dito:

\begin{enumerate}
  \item A relação ``$A = B$'' deve ser entendida como usado em álgebra (isto é, como denotando uma relação simétrica de igualdade entre $A$ e $B$) e não como geralmente usado em programação, como um operador de atribuição assimétrico (onde $A = b$ não é o mesmo que $b = A$);
  \item O símbolo ``='' expressa a relação de dois termos serem idênticos;
  \item Essa relação é transitiva: se A, B e C são variáveis e se $A = B$ e $B = C$, então $A = C$ (se A é idêntico a B e B é idêntico a C, então A é idêntico a C);
  \item Pelo item (2), não podemos fazer $A = 1$ e $A = 2$: isso resulta em falha, por inconsistência.
\end{enumerate}

Se temos que existe alguma substituição $\iota$ (possívelmente vazia) para que $p(a_1, ..., a_n) = q(b_1, ..., b_n)$, dizemos que  $p(a_1, ..., a_n)$ é unificável com $q(b_1, ..., b_n)$. ``='' é o \textbf{símbolo de unificação}\footnote{Ou, dependendo do contexto, de substituição, mas isso não deve alterar a compreensão do texto.}.

Todas as substituições são iguais, mas algumas são mais iguais que outras. Em particular, dado um programa P e substituições $\iota$ e $\nu$, se existe alguma substituição $\eta$ tal que $(P\iota)\eta = P \nu$, dizemos que $\iota$ é uma substituição mais geral do que $\nu$. A substituição mais geral será de especial importância logo mais. %QUESTION: o que significa $P\iota\eta$?

Agora podemos expressar nosso objetivo de provar o goal a partir do programa como o de achar uma substituição tal que cada termo do goal seja unificável com alguma cláusula do programa. Mais precisamente, um goal é provado a partir do programa se é possível unificar cada termo do goal com alguma cláusula do programa de forma a preservar a consistência das regras. O processo pelo qual esse objetivo é realizado é chamado \textbf{processo de resolução}.

Unificação exerce um papel fundamental na programação lógica. Na prática, ele resume processos de atribuição de valores, gerenciamento de memória, invocação de funções e passagem de valores, entre outros. O primeiro estudo formal sobre unificação é devido a John A. Robinson, que depois de provar que existe um algoritmo de unificação, gerou o primeiro de que temos conhecimento. %QUESTION: nao tem uma referencia?

O algoritmo dele é um tanto ineficiente e não será estudado aqui. Usaremos um mais prático no lugar. Antes, vale lembrar que um programa lógico é um conjunto de regras que recebe um goal (ou uma busca) e retorna {\it sucesso} (ou, sim, ou verdadeiro, dependendo da preferência pessoal) se a busca tem sucesso ou {\it falha} (ou, não, ou falso...) se não.

Como discutido acima, para provar um goal a partir de um programa é suficiente que tenhamos um algoritmo de unificação. Esse algoritmo recebe uma equação do tipo $T_1 = T_2$, e devolve uma substituição mais geral\marginpar{É importante que seja a mais geral, para não perdermos possíveis soluções} %COMMENT: como essa "nota lateral" nao deixa uma marca na coisa a que se refere, como no caso da nota de rodape, ficou estranho. Neste caso, acho que ficaria melhor a nota de rodape mesmo. Estas notas na margem ficam melhores quando voce faz nelas comentarios mais auto-contidos, que tem a ver com o que foi dito no texto principal que esta ao lado dela.
para as variáveis presentes, caso tal substituição exista, ou falha, caso contrário. O algoritmo que utilizaremos faz uso de uma pilha para armazenar as equações a serem resolvidas e de um espaço
$\Gamma$ para armazenar a substituições:

\begin{itemize}
    \item[(a)] Primeiro faça o \textit{push} da equação na pilha;
    \item[(b)] Se a pilha estiver vazia, devolva sucesso. Se não, faça o \textit{pop} de um elemento (uma equação) $T_1 = T_2$ da pilha. Realize uma das ações a seguir, segundo a equação retirada:
      \begin{enumerate}
        \item Se $T_1$ e $T_2$ forem termos unários idênticos, nada precisa ser feito;
        \item Se $T_1$ é uma variável e $T_2$ um termo não contendo $T_1$, realize uma busca na pilha pelas ocorrências de $T_1$ e troque $T_1$ por $T_2$ (o mesmo é feito em $\Gamma$); %QUESTION: que \Gamma eh este? E depois disso? Repete o passo b?
        \item Se $T_2$ for uma variável e $T_1$ for um termo não contendo $T_2$, a ação tomada é análoga ao do passo anterior; %QUESTION: e o que faz neste caso?
        \item Se $T_1$ e $T_2$ forem termos compostos de mesmo funtor principal e aridade, $f(a_1, ..., a_n)$ e $p(b_1, ..., b_n)$, adicione as equações $a_i = b_i$ na pilha;
        \item Em outro caso, devolva falha.
      \end{enumerate}
    \item[(c)] Retorne ao passo (b).
\end{itemize}






  %\begin{enumerate}
      %\item $f(k_1, ..., k_m) = p_i(t_1, ..., t_n)$, onde $m = n$, $f = p_i$: buscue as ocorrências   troque as variáveis $t_l$ por $s_l$;
      %\item $f(k_1, ..., k_m) = p_i(t_1, ..., t_n)$, onde $f \neq p_i$: delete essa equação e pare, retorne com falha;
      %\item $x = x$: delete a equação;
      %\item $x = y$, onde $x$ é uma variável, mas $y$ não: troque $x$ por $y$ em todas as proposições consideradas;
      %\item $x = y$, onde $x$ é variável de $y$ (isto é, $y$ é algo como $y(a_1, ..., x, ..., a_n))$: pare e retorne com falha.
  %\end{enumerate}
  %\item[(c)] Se existe mais algum $p_i$ a ser escolhido, volte ao item (a), se não retorne com sucesso.
%\end{itemize}

Intuitivamente, esse algoritmo tenta provar a equação de forma construtiva: isto é, tenta construir uma solução por meio de substituições e, se não chegar a uma contradição, termina com sucesso, ``devolvendo'' (não no sentido de uma função que devolve um valor, mas no de ``mostrar'' ao usuário do programa) a substituição realizada.

Para provar um goal G, escolhemos não-deterministicamente\footnote{No geral, podem existir várias escolhas possíveis e pode ser que, por algumas sequências de escolhas de cláusulas, nunca chegemos a uma prova do goal, apesar de ele ser deduzível a partir de outras cláusulas. Quando dizemos que a escolha é não-determinística, queremos dizer que, se existem conjuntos de escolhas que provam o goal, um desses conjuntos é escolhido (a escolha é feita entre as cláusulas que podem provar o
  goal, o que significa que, se ele é provável, ele é provado). Na prática, isso pode ser implementado apenas aproximadamente, mas, ainda assim, é uma abstração importante e leva a aplicações interessantes, como as da assim chamada \textit{programação não-determinística}.} a cabeça de uma cláusula T do programa, construímos uma equação do tipo $G = T$ e aplicamos o algoritmo acima. Caso ele devolva sucesso, fazemos o mesmo com cada termo do corpo da cláusula. Caso devolva falha, seleciona-se outra cláusula e é realizado o mesmo processo, até que não haja mais cláusulas a serem selecionadas, quando o goal retorna falha.

O passo 2.b do algoritmo merece uma explicação um pouco mais detalhada. Ela diz implicitamente que x não é unificável com algum $y(a_1, ...,x, ...a_n)$, isto é, com algum funtor que tome x como argumento. Pode parecer estranho a princípio, mas a estranheza some se se lembrar que funtor não é função: um funtor exerce uma função primariamente estrutural e simbólica. Sem isso, se x = y($a_1$, ..., x, ..., $a_n$), então x = y($a_1$, ...,y($a_1$, ..., x, ..., $a_n$) , ...,
$a_n$) = y($a_1$, ...,y($a_1$, ...,y($a_1$, ...,y($a_1$,
...,x, ..., $a_n$) , ..., $a_n$) , ..., $a_n$) , ..., $a_n$) em um ciclo sem fim. Com um processo desses, não dá para provar um goal e, portanto, é devolvida falha.

Para entender melhor, tome o exemplo do código Circuito, no início deste capítulo, e suponha que àquele código é submetido o goal {\tt resistor(energia, n1)?}. O algoritmo é aplicado como se segue:

  \begin{enumerate}
    \item Tentaremos a unificação do goal com a cláusula na primeira linha do programa: a equação resistor(energia, n1) = resistor(energia, n1) é posta na pilha;
    \item Uma equação é retirada da pilha: a equação resistor(energia, n1) = resistor(energia, n1);
    \item A equação é formada por dois funtores termos compostos de mesmo funtor principal e mesma aridade: as equações energia  = energia e n1 = n1 são postas na pilha;
    \item É retirada uma equação da pilha: a equação energia = energia. Como os dois lados da equação são idênticos, não há mais o que fazer;
    \item É retirada outra equação da pilha: a equação n1 = n1. Como os dois lados da equação são idênticos, não há mais o que fazer;
    \item A pilha está vazia: o programa devolve sucesso, com a substituição $\Gamma = \{\}$ (substituição vazia).
  \end{enumerate}


\end{document}
