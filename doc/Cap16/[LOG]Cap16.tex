\documentclass{article}

\usepackage[utf8]{inputenc}
\usepackage{placeins}
\usepackage{graphicx}
\usepackage[ruled,vlined,linesnumbered]{algorithm2e}
\usepackage{listings}
\usepackage{amsthm}
\usepackage{amsmath}
\usepackage{caption}
\usepackage{epigraph}

\captionsetup[figure]{font=scriptsize}

\newtheorem{definition}{Definição}[section]
\theoremstyle{remark}
\newtheorem*{remark}{Observações}
\theoremstyle{theorem}
\newtheorem{theorem}{Teorema}[section]

\setlength{\parskip}{.5em}

\renewcommand\refname{Leituras adicionais}
\renewcommand\figurename{Figura}
\renewcommand{\lstlistingname}{Código}

\newcommand{\eclipse}{$ECL^iPS^e$}
\newcommand{\definicao}[1]{\textbf{#1}\marginpar{\small \textbf{#1}}}

\lstdefinestyle{prosty}{
  language=Prolog,
  basicstyle=\small
}



\begin{document}

\section{Programação Relacional em Scheme}

\epigraph{\cit{Uma pessoa só tem certeza daquilo que
    constrói}}{Giambattista Vico - \\Historiador italiano, século XVIII}


Neste capítulo, veremos uma linguagem de programação lógica com um
sabor diferente do Prolog que vimos no início, chamada
\technical{miniKanren}, e veremos como implementá-la, por meio da
linguagem \technical{Scheme}.

Usaremos \technical{Scheme} porque é uma linguagem pequena, o
que significa que sua apresentação será curta, e que é, ao mesmo
tempo, poderosa, o que significa que não precisaremos de muito código
para fazer o que nos propomos.

Esse capítulo tem um sabor diferente dos demais. A diferença pode ser vista
rapidamente, pela cara do código\footnote{Em particular, nota-se que
  usamos aqui, para os códigos, fontes e cores diferentes das usadas nos
  capítulos anteriores. Isso é, principalmente, porque entendemos que, aqui, a
  utilização de cores e artifícios como o {\bf negrito} tenderiam a
  atribuir atenção especial para elementos que não precisam de atenção
  especial.} para , mas também de várias outras
formas. Redefiniremos alguns termos usados anteriormente. Essas
redefinições terão semelhanças e diferenças às definições originais,
mas escolhemos não explicitar essas diferenças aqui, por acreditarmos serem
claras o suficiente sem um comentário a mais. Outro ponto que vale nota é que, aqui, buscamos
apenas um maior entendimento e, para tanto, tentamos deixar o código
como ``implementado em primeiros princípios'', isto é, de forma mais
simples. Isso significa evitar construções que poderiam
deixá-lo mais eficiente, mas que requeririam uma discussão maior,
o que seria às custas de uma divergência de atenção aos pontos
principais que queremos passar. Para uma implementação que tenta ser
eficiente, veja, por exemplo \cite{fast}.

\subsection{Introdução ao Scheme}

\technical{Scheme} é um \technical{Lisp}\footnote{De
  \foreign{LISt Processing}.}. O termo \technical{Lisp} é às vezes
usado para se referir a uma linguagem de programação, mas o mais
correto seria ser usado para se referir a uma família de linguagens (de fato,
dezenas de linguagens), todas com algumas características em comum,
em particular:

\begin{itemize}
\item São linguagens multi-paradigma, mas como um
  foco no paradigma de programação funcional, o que significa, entre
  outras coisas, que  funções são ``cidadãos de primeira classe'';
\item Todo código \technical{Lisp} (que não tenha erro de sintaxe) é avaliado para
  algum valor no momento de execução;
\item Programas são expressos em ``notação polonesa'' (notação prefixada), em formato de
  listas\footnote{Também conhecidas como \technical{S-Expressions}, ou
    \technical{Sexps}.}(listas em \technical{Lisps} são delimitadas por
  parenteses)\footnote{Usaremos daqui em diante a notação
    \enphasisb{codigo \seta\ valor} para denotar que o código
    \enphasisb{código} avalia para o valor \enphasisb{valor}.}, por exemplo:
  \\
  \codigo{(+ 1 2)} \seta\ 3;
\item O que nos leva a outro ponto: não existe diferença
  sintática entre a forma como programas \technical{Lisp} são escritos
  e a forma como suas estruturas de dados são representadas. Diz-se,
  assim, que \technical{Lisps} são homoicônicas (vale dizer, Prolog
  também é uma língua homoicônica), o que significa que a diferença
  entre dados e programa é ``borrada'', e programas podem, e
  frequentemente são, manipulados livremente.
\end{itemize}

Agora, vamos rapidamente introduzir a sintaxe principal de
\technical{Scheme}, com alguns exemplos\footnote{Só
  introduziremos a parte da linguagem que nos será relevante, o que
  não é a linguagem inteira.}:

\begin{itemize}
\item Listas são representadas como \codigo{($el_1\ el_2\ ...\
    el_n$)}, em que $el_i$ é o $i$-ésimo elemento da lista. No entanto, se
  escrevêssemos uma lista assim, ela seria
  confundida com uma aplicação de função (aplicação da função $el_1$
  aos argumentos $el_2$ a $el_n$), então, para fins de
  desambiguação, é usada uma aspa simples, e a lista é escrita como
  '($el_1 el_2 ... el_n$)\footnote{Note que, por ser fechada por
    parênteses, apenas uma aspa é o suficiente.}, que é equivalente a \codigo{(list $'el_1\
    ...\ 'el_n$)} . Essa aspa simples também pode ser usada para
  ``evitar que um objeto seja avaliado''\footnote{Uma colocação mais
    correta seria ``tornar objetos auto-avaliantes'', mas não
    precisamos entrar em muitos detalhes de como isso funciona. Para
    nós é suficiente dizer que 'a é um símbolo.}:\\
  A linha\\
  \codigo{a}\\
  resulta em erro se \codigo{a} não for uma variável, já que
  o executor do código tentará avaliá-la (gerar um valor a partir
  dela), mas \codigo{a} não tem valor associado. Mas a linha\\
  \codigo{'a} $\Rightarrow $ \codigo{'a}\\
  é avaliada, ``como ela mesma''.\\

\item Uma estrutura de dados mais geral do que lista em
  \technical{Scheme} é o que é chamado \technical{cons pair} (que nós
  chamaremos daqui para frente simplesmente de ``par''). A lista
  \codigo{'(a b c d)} é equivalente a \codigo{(cons a (cons b (cons c
    (cons d '()))))}, em que \codigo{'()} é a lista
  vazia
  % \footnote{Note que a estrutura \codigo{(cons a (cons b (cons c  d)))} não.
    Assim \technical{cons} constrói uma estrutura de
    dados formada por um par. Para obter o primeiro elemento do par,
    usa-se \technical{car} e, para obter o segundo,
    \technical{cdr}\footnote{Esses nomes têm uma origem histórica:
      eram nomes de registradores quando os primeiros
      \technical{Lisps} estavam sendo criados (vale notar que o
      primeiro Lisp foi também uma das primeiras linguagens de
      alto-nível ainda em uso, tendo surgido pouco depois do Fortran).}. Temos, por exemplo,\\
    \codigo{(car (cons 1 (cons 2 3)))} \seta\ 1\\
    \codigo{(cdr (cons 1 (cons 2 3)))} \seta\ \codigo{(cons 2 3)} =
    \codigo{'(2 . 3)}\\
    \codigo{(car '(1 2 3 4))} \seta\ 1\\
    \codigo{(cdr '(1 2 3 4))} \seta\ \codigo{'(2 3 4)}\\
    Com isso, podemos definir um lista indutivamente como sendo ou a
    lista vazia, \codigo{'()}, ou um par, cujo
    \technical{cdr} é uma lista\footnote{Note que \codigo{(cons 2 3)},
      por exemplo, não é uma lista. Esse tipo de estrutura é chamada
      \foreign{dotted list}, porque, para distingui-la de uma lista, é
      costumeiramente impressa como \codigo{'(2 . 3)}, mas, assim como
      com Prolog, é uma estrutura de dados diferente, que tem o nome
      \foreign{dotted \enphasisb{list}} devido a uma aparência como que acidental.}.
\item Para definir funções, use \enphasisb{lambda}, ou
  \enphasisb{$\lambda$}\footnote{Editores de texto atuais podem aceitar os dois tipos de
    entrada, mas optamos por usar $\lambda$. Este uso do símbolo tem a
    seguinte origem: Bertrand Russel e Alfred Whitehead buscaram, no
    início do século XX, lançar as bases lógicas da matemática em seu
    trabalho \foreign{Principia Mathematica}. Lá, para denotar que uma
    variável é livre, ela recebia um chapéu, como em \technical{â(a +
      y)}. Mais tarde, ainda trabalhando nos fundamentos da
    matemática, Alonzo Church achou que seria mais conveniente
    ter fórmulas crescendo linearmente na horizontal (note que o
    ``chapéu'' faz com que a fórmula cresça para cima), então decidiu
    mover o chapéu para o lado, obtendo \technical{\^{}a(a + y)}.
    Mas o chapéu flutuando parece engraçado, então Church o trocou
    pelo  o símbolo não usado mais próximo que tinha, um $\Lambda$, como em
    \technical{$\Lambda a$(a + y)}. Mas
    $\Lambda$ tem uma grafia muito parecida com outra letra comum, o que foi
    percebido como um incoveniente, então ele acabou eventualmente
    trocando para $\lambda$ em sua teoria, que acabou se chamando
    \enphasis{cálculo $\lambda$} \cite{norvig}.}:\\
  \codigo{(($\lambda$ (a b) (/ a b)) 1 3)} \seta\ 1/3n

\item Para definir constantes, use \enphasisb{define}:\\
  \codigo{(define divide ($\lambda$ (a b) (/ a b)))}\footnote{Um
    \technical{açucar sintático} para essa construção é
    \codigo{(define (divide a b) (/ a b))}.}\\
  \codigo{(divide 1 3)} \seta\ 1/3\\
  \codigo{(define c (divide 1 3))}\\
  \codigo{(divide (divide 1 9) c)} \seta\ 1/3\\
  \codigo{(define (divide3 (divide3 a) (divide a 3)))}\\
  \codigo{(divide3 9)} \seta 3\footnote{Esse tipo de uso é chamado
    \foreign{currying}, e é possível porque \technical{Scheme} tem
    fecho léxico.},;

\item  Um ponto importante, que usaremos muito logo mais é que, se
  quisermos criar listas com os valores das variáveis, no lugar de
  nomes simbólicos, podemos usar, no lugar da aspa simples, a crase e
  preceder o nome da variável com uma vírgula:\\
  \codigo{(define x 10)}\\
  \codigo{`(1 2 'x ,x)} \seta\ \codigo{'(1 2 'x 10)}

\item Para realizar execuções condicionais, use \enphasisb{cond}:\\
  \begin{lstlisting}
    (cond
      ((< 1 0) (+ 3 4))
      ((< 0 1) (- 3 4))
      (else 0))
  \end{lstlisting}
  \hspace{1cm} \seta\ -1\\
  
  Podem ser adicionadas quantas cláusulas do tipo
  \codigo{((condicao)(efeito))} se quiser (vale notar, elas são
  avaliadas sequencialmente), sendo que a última pode
  opcionalmente ser como \codigo{(else (efeito))}, ou \codigo{(\#t
    (efeito))}.

\item Para adicionar variáveis locais, use \technical{let}:
  \begin{lstlisting}
    (let ((a (+ 3 4))
          (b (cons 1 2)))
      (+ a (car b)))
  \end{lstlisting}
  \hspace{1cm} \seta\ 8\\
  O \technical{let} tem duas partes, a de definições, da forma
  \codigo{((variavel valor)(variavel valor) ... (variavel
    valor))}\footnote{Podem ser adicionadas quantas variáveis se
    quiser. As atribuições são feitas ``paralelamente'' (o que
    significa que a atribuição de valor a uma variável não influencia no
    da outra, o que pode ser feito de forma paralela, no sentido usual, ou não).}
  e, em seguida, a parte de valor, que nos dá o valor que \technical{let}
  assume.
\end{itemize}

  % Como exemplo, o seguinte programa calcula o comprimento de uma lista

  % \begin{lstlisting}[escapeinside={[}{]}]
  %   (define length
  %     ([$\lambda$] (l)
  %       (cond
  %         ((null? l)0)
  %         (else
  %           (+ 1 (length (cdr l)))))))
  % \end{lstlisting}
  

  Dada essa introdução, faremos uso dessas e outras construções da
  linguagem sem maiores comentários (exceto quando uma construção
  especialmente difícil ou complexa o justificar). Para uma introdução
  mais compreensiva à linguagem, veja \cite{kent}.

  \subsection{A linguagem miniKanren}

  Nosso objetivo aqui é implementar miniKanren, uma linguagem de
  programação relacional. No lugar de descrever toda a linguagem e
  depois implementá-la, seguimos pelo caminho de mostrar um pequeno
  exemplo do que esperamos conseguir e, então, implementamos a
  linguagem. A esperança é que essa abordagem ofereça maior
  entendimento dos conceitos explorados.

  % Antes de prosseguir, vale uma pequena explicação sobre os usos
  % não-usuais da
  O tipo de coisa que queremos poder fazer com miniKanren é como o
  seguinte\footnote{Usaremos as convenções da literatura de usar
    subescritos e sobrescritos e símbolos matemáticos, como $\equiv $, para
    representar as relações, na esperança de que isso clarifique a
    notação e deixe o texto menos pesado. Em particular, para
    diferenciar um objeto relacional de um funcional, o relacional
    terá um ``o'' sobrescrito (como em $relacao^o$). Ao escrever o
    programa para o computador ler, os sobrescritos e subescritos que
    forem alfa-numéricos ou ``*'' podem ser escritos normalmente, na
    frente do termo (como em $relacaoo$, ou $run*$). O símbolo $\equiv$ é
    escrito ``$==$'' e, termos como $termo^\infty$,
    ``$termo-inf$''. Ademais, \#u e \#s devem ser trocados por
    \enphasis{fail} e \enphasis{succeed}, respectivamente.}:
  \\

  \begin{lstlisting}[escapeinside={[}{]}]
    (defrel ([$teacup^o$] t)
      ([$disj_2$] ([$\equiv $] 'tea t) ([$\equiv $] 'cup t)))
    ([$run^*$] x
      ([$teacup^o$] x))
  \end{lstlisting}
  \hspace{1cm} \seta\ \codigo{'(tea cup)}\\

  %% TODO

  %% \begin{lstlisting}[escapeinside={[}{]}]
  %%   ([$run^*$] x
  %%     (fresh c u d a)
  %%       (([$\equiv$] c 'pea)([$\equiv$] x c)))
  %% \end{lstlisting}
  %% \hspace{1cm} \seta\ \codigo{'(tea cup)}\\

  Veremos mais exemplos quando o construirmos. A construção que se segue
  é em grande parte baseada em \cite{will}. Para conferir detalhes de
  implementações completas, veja \cite{kanren}.

  Como visto no primeiro exemplo, não seguimos, como em Prolog, uma
  convenção de nomeação de variáveis (em Prolog, a convenção era de
  que variáveis são  capitalizadas). Assim, precisamos de algo para
  discerni-las e, para tal fim, o que usamos é o \technical{fresh},
  informando que a variável é ``fresca''.
  
  Lembre-se que uma variável relacional não é a mesma
  coisa que uma variável em uma linguagem de programação tradicional
  (não relacional). Para definirmos uma variável única, vamos precisar
  de\footnote{Usamos \technical{vector} para que a unicidade da
    variável seja definida por sua posição de memória. Outra opção
    seria distingui-las por valor, se nos assegurássemos de que seu valor
    é único.}
  
  \begin{lstlisting}
    (define (var name)(vector name))
  \end{lstlisting}

  \noindent Usaremos também a seguinte definição:\footnote{Símbolos como ``?'' podem ser
    usados no meio do código da mesma forma que outros, tais como ``a'' ou ``b''.}

  \begin{lstlisting}
    (define (var? name)(vector? name))
  \end{lstlisting}

  Para evitar problemas como os de colisão de variáveis, as variáveis
  são locais, assim como em \technical{Scheme}. Precisamos, então, de
  uma forma de modelar isso (note que a definição acima não reflete
  isso) e, o que usamos é o seguinte:

  \begin{lstlisting}
    (define (call/fresh name f)
      (f (var name)))
  \end{lstlisting}

  Essa função\footnote{``Método'', ou ``\foreign{procedure}'', como é
    mais popularmente conhecido na comunidade Scheme.},  espera, como
  segundo argumento, uma expressão $\lambda$, que
  recebe como argumento uma variável e produz como valor um goal, o qual
  tem acesso à variável criada.
  
  Precisamos, agora, saber como associar um valor a uma
  variável. Diremos que o par \codigo{`(,z . a)} é uma associação de
  \enphasis{a} à variável \enphasis{z}. Mais em geral, um par é uma
  associação quando o seu \technical{car} for uma variável.

  Uma lista de associações será chamada \enphasis{substituição}.

  Na substituição \codigo{`((,x . ,z))}, a variável \enphasis{x} é
  ``fundida'' (ou, na nossa linguagem anterior, unificada) à variável
  \enphasis{z}. A substituição vazia é simplesmente uma lista vazia:
  \codigo{(define empty-s '())}. Nem toda lista de associações é uma
  substituição, no entanto. Isso porque, não aceitamos, em nossas
  substituições, associações com o mesmo \technical{car}. Então, o
  seguinte não é uma substituição: \codigo{`((,z . a)(,x . c)(,z
    . b))}.

  Precisamos de dois procedimentos importantes para lidar com
  substituições: um para estendê-las e um para obter o valor de uma
  variável presente nela.

  Para obter o valor associado a \enphasis{x}, usamos \enphasis{walk},
  que deve se comportar como o seguinte:
  \begin{lstlisting}
    (walk y
      `((,z . a)(,x . ,w)(,y . ,z)))
  \end{lstlisting}
  \hspace{1cm} \seta\ \enphasis{a}\\
  porque \enphasis{y} está fundido a \enphasis{z}, que está associado
  a \enphasis{a}. O \technical{walk} é como se
  segue\footnote{Lembre-se que \codigo{(and a b)} \seta\ \enphasis{b},
    se \enphasis{a} $\neq \#f$.}:

  \begin{lstlisting}
    (define (walk v s)
      (let ((a (and (var? v)(assv v s))))
        (cond
          ((pair? a)(walk (cdr a) s))
          (else v))))
  \end{lstlisting}

  \noindent Esse código faz uso de \enphasis{assv}, que ou produz
  \enphasis{\#f}, se não há associação cujo \technical{car} seja
  \enphasis{v} na substituição \enphasis{s}, ou produz o
  \technical{cdr} de tal associação. Perceba que, se \enphasis{walk}
  produz uma variável como valor, ela é necessariamente fresca (isto
  é, que não foi associada).

  Para estender uma substituição, usamos \technical{ext-s}, que faz
  uso de \technical{occurs?}:

  \begin{lstlisting}
    (define (ext-s x v s)
      (cond
        ((occurs? x v s) #f)
        (else (cons '(,x . ,v) s))))

    (define (occurs? x v s)
      (let ((v (walk v s)))
        (cond
          ((var? v) (eqv? v x))
          ((pair? v)
            (or (occurs? x (car v) s)
                (occurs? x (cdr v) s)))
          (else #f))))
  \end{lstlisting}

  Esse \technical{occurs?} realiza o ``teste de ocorrência'' (aquele
  que, como mencionamos anteriormente, não é feito por padrão no
  Prolog por razões de eficiência, e que faz com que substituições do
  tipo \codigo{`((,y . (,x))(,x . (a . ,y))} sejam inválidas).

  Com isso em mãos, podemos definir nosso unificador:
  \vspace{1.5cm}
  
  \begin{lstlisting}

    (define (unify u v s)
      (let ((u (walk u s))(v (walk v s)))
        (cond
          ((eqv? u v) s)
          ((var? u) (ext-s u v s))
          ((var? v) (ext-s v u s))
          ((and (pair? u) (pair? v))
           (let ((s (unify (car u)(car v) s)))
             (and s
               (unify (cdr u)(cdr v) s))))
          (else #f))))

  \end{lstlisting}

  \subsection{Streams}

  Antes de continuarmos, precisamos tocar no modelo de avaliação de
  Scheme. Scheme é ``uma linguagem de ordem aplicativa'', o que
  significa que, quando uma função é avaliada, seus argumentos são
  todos avaliados no momento de aplicação. Por esse motivo, os
  \technical{and} e \technical{or} usuais, por exemplo, não podem ser
  funções em Scheme\footnote{Se fossem, \codigo{(or \#t a)}, por
    exemplo, poderia gerar erro quando \codigo{a} não for uma
    variável. Como \technical{or} não é uma função, o \technical{a}
    nessa linha não chega a ser avaliado, e temos \codigo{(or \#t a)}
    \seta \#t.}. Uma alternativa à ordem aplicativa é a ``ordem
  normal'', que outras linguagens de programação funcional
  adotaram. Linguagens de ordem normal só avaliam o argumento de uma
  função quando esse argumento for usado, ``atrasando'' a avaliação do
  mesmo (no que é chamado ``avaliação tardia'', ou ``avaliação
  preguiçosa'').

  Avaliação preguiçosa é conveniente em diversas ocasiões e pode
  ser emulada em linguagens de programação funcional de ordem
  aplicativa\footnote{Com fecho léxico.} pelo uso de
  \definicao{streams}. \technical{Streams} são definidos indutivamente
  como sendo ou a lista vazia, ou um par cujo
  \technical{cdr} é um \technical{stream}, ou uma suspenção. Uma
  \definicao{suspenção} é uma função do tipo \codigo{($\lambda$ ()
    corpo)}, em que \codigo{(($\lambda$ () corpo))} é uma
  \technical{stream}. Agora, se fizermos

  \begin{lstlisting}[escapeinside={[}{]}]
    (define ([$\equiv $] u v)
      ([$\lambda$] (s)
        (let ((s (unify u v s)))
          (if s `(,s) '()))))
  \end{lstlisting}
        
  \noindent temos que $\equiv $ produz um goal. Dois outros goals, \enphasis{sucesso}
  e \enphasis{falha}, são denotados \#s e \#u:\\

  \begin{lstlisting}[escapeinside={[}{]}]
    (define #s
      ([$\lambda$] (s)
        `(,s)))

    (define #u
      ([$\lambda$] (s)
        '()))
  \end{lstlisting}

  Definimos, neste contexto, um goal como uma função que recebe uma
  substituição e que, se retorna, retorna uma \technical{stream} de
  substituições.

  Como um exemplo, temos que \codigo{(($\equiv $ x y) empty-s)} \seta\
  \codigo{`(((,x . ,y)))}, uma lista com uma substituição (com uma
  associação).

  Ao lidar com \technical{Streams}, precisamos de funções especiais,
  já que não são ``simples listas''. \technical{Streams} são uteis
  (entre outras coisas) para a representação de estruturas de dados
  infinitas, então, por isso, funções e variáveis para lidar com elas
  terão um $\infty$ sobrescrito, para diferenciá-las das funções para
  listas comuns. Podemos, então, definir $append^\infty$:

  \begin{lstlisting}[escapeinside={[}{]}]

    (define ([$append^\infty$] [$s^\infty$] [$t^\infty$])
      (cond
        ((null? [$s^\infty$]) [$t^\infty$])
        ((pair? [$s^\infty$])
          (cons (car [$s^\infty$]
            ([$append^\infty$] (cdr [$s^\infty$]) [$t^\infty$]))))
        (else ([$\lambda$] ()
                ([$append^\infty$] [$t^\infty$] ([$s^\infty$]))))))
        
  \end{lstlisting}

  \noindent Note que, na suspensção, a ordem dos argumentos é
  trocada\footnote{No que é chamado de \technical{trampolim binário} \cite{tramp}.}
  Com essa função, podemos fazer

  \begin{lstlisting}[escapeinside={[}{]}]

    (define ([$disj_2\ g_1\ g_2$])
      ([$\lambda$] (s)
        ([$append^\infty$] ([$g_1$] s) ([$g_2$] s))))

  \end{lstlisting}
      
  \noindent em que $disj_2$ é uma disjunção (como um \enphasis{ou}
  lógico). 

  Veja agora a seguinte definição:

  \begin{lstlisting}[escapeinside={[}{]}]

    (define ([$never^o$])
      ([$\lambda$] (s)
        ([$\lambda$] ()
          (([$never^o$]) s))))

  \end{lstlisting}

  \noindent Esse é um goal que não sucede nem falha. Para entender
  porque a ordem dos argumentos é trocada na suspenção de $append^\infty$,
  compare o valor de $s^\infty$ em
  \vspace{.7cm}

  \begin{lstlisting}[escapeinside={[}{]}]

    (let (([$s^\infty$] (([$disj_2$]
                      ([$never^o$])
                      ([$\equiv $ 'olive x]))
                      empty-s)))
      [$s^\infty$])
                      
  \end{lstlisting}

  \noindent com o valor de $s^\infty$ em

  \begin{lstlisting}[escapeinside={[}{]}]

    (let (([$s^\infty$] (([$disj_2$]
                      ([$\equiv $ 'olive x]))
                      ([$never^o$])
                      empty-s)))
      [$s^\infty$])
                      
  \end{lstlisting}

  Em contraste com $never^o$, aqui está $always^o$, que sempre sucede:

  \begin{lstlisting}[escapeinside={[}{]}]

    (define ([$always^o$])
      ([$\lambda$] (s)
        ([$\lambda$] ()
          (([$disj_2$] #s ([$always^o$])) s))))

  \end{lstlisting}

  Antes de continuar, será útil conhecer a função \technical{map}:\\
  \codigo{(map f '($el_1\ ...\ el_n$))} \seta\ \codigo{'((f $e_1$) ... (f $el_n$))}

  \noindent A lista construída por \technical{map} é constrúida por
  \technical{cons}. Mas existe também \technical{map-append}, análoga
  a \technical{map}, mas em que a lista resultante é construída por
  \technical{append}. Usaremos um \technical{append-map}, mas para
  \technical{streams}, isto é, um \technical{append-ma$p^\infty$}:

  \begin{lstlisting}[escapeinside={[}{]}]

    (define ([$append-map^\infty$] g [$s^\infty$])
      (cond
        ((null? [$s^\infty$]) '())
        ((pair? [$s^\infty$])
          ([$append^\infty$] (g (car [$s^\infty$]))
            ([$append-map^\infty$] g (cdr [$s^\infty$]))))
      (else ([$\lambda$] ()
              ([$append-map^\infty$] g ([$s^\infty$]))))))

  \end{lstlisting}

  Assim como definimos a disjunção de dois goals, com isso podemos
  definir também a conjunção:

  \begin{lstlisting}[escapeinside={[}{]}]

    (define ([$conj_2$] [$g_1$] [$g_2$])
      ([$\lambda$] (s)
        ([$append-map^\infty$] [$g_2$] ([$g_1$] s))))

  \end{lstlisting}


  \subsubsection{Voltando ao problema das variáveis}

  No \technical{miniKanren}, assim como em linguagens de programação
  relacional no geral, as variáveis são lógicas. Mas, como a
  implementação está sendo feita em \technical{Scheme}, precisamos,
  eventualmente, representar variáveis em termos de
  \technical{Scheme}, num processo de reificação (lembre-se que
  reificação tem a ver com ``concretização''). Em particular, um
  termo \technical{miniKanren} reificado não pode conter variáveis
  lógicas. Fazemos isso associando variáveis lógicas a símbolos do
  tipo $\__i$. Para realizar essa operação, precisamos
  primeiro do \technical{reify-name}\footnote{$string\rightarrow symbol$ é
    escrito \technical{$string->symbol$}.}:

  \begin{lstlisting}[escapeinside={[}{]}]

    (define (reify-name n)
      ([$string\rightarrow symbol$]
        (string-append ``_''
          ([$number\rightarrow string$] n))))

  \end{lstlisting}

  Com \technical{reify-name}, podemos criar o \technical{reify-s}, que
  recebe uma variável e uma substituição, inicialmente vazia, r:

  \begin{lstlisting}[escapeinside={[}{]}]

    (define (reify-s v r)
      (let ((v (walk v r)))
        (cond
          ((var? v)
           (let ((n (length r)))
             (let ((rn (reify-name n)))
               (cons `(,v . ,rn) r))))
          ((pair? v)
           (let ((r (reify-s (car v) r)))
             (reify-s (cdr v) r)))
          (else r))))

  \end{lstlisting}

  \noindent Vale notar, aqui \technical{length} produz um número único
  em cada uso de \technical{reify-name}.

  Para continuar com nosso esquema de reificação, vamos precisar de
  uma versão ligeiramente diferente do \technical{walk}, o qual
  chamaremos \technical{$walk^*$}\footnote{Leia \enphasis{walk
      star}.}. Veja a definição:
  
  \begin{lstlisting}[escapeinside={[}{]}]
    (define ([$walk^*$] v s)
      (let (v (walk v s))
        (cond
          ((var? v) v)
          ((pair? v)
            (cons
              ([$walk^*$] (car v) s)
              ([$walk^*$] (cdr v) s)))
          (else v))))
  \end{lstlisting}

  Note que \technical{walk} e \technical{$walk^*$} só diferem se
  \technical{$walk^*$} \enphasis{caminhar} a um par com alguma
  variável com associação na substituição \enphasis{s} (algo como
  \codigo{(,z . (1 . ,x))}). Além disso,
  note que, se \technical{$walk^*$} produz um valor \enphasis{v} ao
  \enphasis{caminhar} por uma substituição \enphasis{s}, temos
  garantia de que as variáveis em \enphasis{v} (se existirem) são frescas.

  Com isso em mãos, podemos substituir cada variável fresca por sua
  reificação, com

  \begin{lstlisting}[escapeinside={[}{]}]

    (define (reify s)
      ([$\lambda$] (s)
        (let ((v ([$walk^*$] v s)))
          (let ((r (reify-s v empty-s)))
            ([$walk^*$] v r)))))

  \end{lstlisting}

  % Exemplo:

  % \begin{lstlisting}[escapeinside={[}{]}]
  % Insert example code
  % \end{lstlisting}

  \subsection{Finalizando}

  Na seção anterior, definimos a ``coluna dorsal'' do
  \technical{miniKanren}. Para terminarmos\footnote{Mais precisamente,
    terminarmos o \enphasis{início}, já que \technical{miniKanren} vai
    além do que vemos aqui.} precisaremos usar as macros do
  \foreign{Scheme}. A palavra
  ``macro'', no geral, é usada (neste contexto de programação) para se
  referir a código que ``escreve código'', isto é, que realiza
  transformações no código a ser compilado ou interpretado. Várias
  linguagens têm macros de tipos diferentes, mas
  poucas são tão poderosas como as macros que (geralmente) estão
  presentes
  em linguagens Lisp\footnote{Provavelmente isso é devido ao aspecto
    homoicônico da linguagem, que torna transformações desse tipo mais
    simples de se realizar do ponto de vista do compilador ou
    interpretador em relação a outras linguagens.}. \technical{Scheme}
  não tem na própria linguagem
  mecanismos de iteração, por exemplo (como um laço \technical{for},
  ou \technical{while}), mas esses (assim como vários outros
  mecanismos de iteração) podem ser facilmente implementados por meio
  de macros.

  Para começar, notamos que temos a disjunção e a conjunção, mas para
  apenas dois argumentos, na forma de  $disj_2$ e $conj_2$. Gostaríamos
  de realizar disjunções e conjunções com $n$ goals, em que $n$ pode
  ser diferente de 2. A disjunção de $n$ termos é definida
  indutivamente (a conjunção é análoga):

  \begin{align*}
    &(disj) \Rightarrow \#u \\
    &(disj\ g) \Rightarrow g \\
    &(disj\ g_0\ g\ ...) \Rightarrow (disj_2\ g_0\ (disj\ g\ ...))
  \end{align*}

  \noindent que se traduz em código como

  \begin{lstlisting}[escapeinside={[}{]}]

    (define-syntax disj
      (syntax-rules ()
        ((disj) #u)
        ((disj g) g)
        ((disj [$g_0$] g ...) ([$disj_2$] [$g_0$] (disj g ...)))))

  \end{lstlisting}

  Cada \technical{defrel} vai definir uma nova função:

  \begin{lstlisting}[escapeinside={[}{]}]

    (define-syntax defrel
      (syntax-rules ()
        ((defrel (name x  ...) g ...)
         (define (name x ...)
           ([$\lambda$] (s)
             ([$\lambda$] ()
               ((conj g ...) s)))))))

  \end{lstlisting}
             
  Váriaveis frescas são criadas com \technical{fresh}:

  \begin{lstlisting}[escapeinside={[}{]}]

    (define-syntax fresh
      (syntax-rules ()
        ((fresh () g ...)(conj g ...))
        ((fresh ([$x_0$] x ...) g ...)
          (call/fresh '[$x_0$]
            ([$\lambda$] ([$x_0$])
              (fresh (x ...) g ...))))))

  \end{lstlisting}
  
  Para executar o goal, definimos \technical{$run^*$}, que recebe uma
  lista de variáveis e um goal e, se terminar de executar, assume como
  valor uma lista com os valores de associção a tais variáveis de modo
  que o goal tenha sucesso (vale lembrar, tal valor pode ser uma
  variável reificada). Definimos também \technical{$run$}, que recebe
  um número natural $n$, uma lista de variáveis e um goal e, se
  terminar de executar, assume como valor os $n$ primeiros elementos
  de \technical{$run^*$}.

  Para termos essas definições, usaremos \technical{$take^\infty$},
  que, quando dado um número $n$ e uma stream \technical{$s^\infty$}, se
  algo, produz uma lista de, no máximo, $n$ valores:

  \begin{lstlisting}[escapeinside={[}{]}]

    (define ([$take^\infty$] n [$s^\infty$])
      (cond
        ((and n (zero? n)) '())
        ((null? [$s^\infty$]) '())
        ((pair? [$s^\infty$])
         (cons (car [$s^\infty$])
           ([$take^\infty$] (and n (sub1 n))
             (cdr [$s^\infty$]))))
        (else ([$take^\infty$] n ([$s^\infty$])))))

  \end{lstlisting}

  \noindent note que, se $n = false$, \technical{$take^\infty$}, se
  retornar, produz uma lista de \enphasis{todos} os valores (pergunta:
  valores de que?).

  Agora podemos definir

  \begin{lstlisting}[escapeinside={[}{]}]

    (define (run-goal n g)
      ([$take^\infty$] n (g empty-s)))

  \end{lstlisting}

  \noindent e

  \begin{lstlisting}[escapeinside={[}{]}]

    (define-syntax run
      (syntax-rules ()
        ((run n ([$x_0$] x ...) g ...)
         (run n q (fresh ([$x_0$] x ...)
                    ([$\equiv$] `(,[$x_0$] ,x ...) q) g ...)))
        ((run n q g ...)
         (let ((q (var 'q)))
           (map (reify q)
             (run-goal n (conj g ...)))))))

     (define-syntax [$run^*$]
       (syntax-rules ()
         (([$run^*$] q g ...)(run #f q g ...))))

  \end{lstlisting}

  Com isso, temos um implementação mínima de
  \enphasis{miniKanren}. Algumas construções importantes foram
  deixadas de lado em favor da brevidade, como os
  \technical{$cond^e$}, \technical{$cond^a$} e \technical{$cond^u$},
  A leitora interessada é convidada a checar \cite{will} ou
  \cite{kanren} para mais detalhes.
  
  
  %% \codigo{(cons (f $el_1$) (cons ... (cons (f $el_n$) '())))}

    
  \begin{thebibliography}{1}

  \bibitem{fast} Ballantyne Michael,
    ``A fast implementation of miniKanren with disequality and absento, compatible with Racket and Chez.''
    \url{https://github.com/michaelballantyne/faster-miniKanren}

  \bibitem{will} Daniel P. Friedman, William E. Byrd, Oleg Kiselyov,
    Jason Hemann,
    ``The Reasoned Schemer - Second Edition'',
    The MIT Press, 2018.

  \bibitem{tramp} Ganz, Steven E., Daniel P. Friedman, and Mitchell Wand.
    ``Trampolined style.'',
    In ACM SIGPLAN Notices, vol. 34, no. 9, pp. 18-27. ACM, 1999.
    
  \bibitem{kanren} Site do miniKanren \url{http://minikanren.org/}.

  \bibitem{kent} R. Kent Dybvyg,
    ``The Scheme Programming Language - Fourth Edition'',
    disponível em \url{https://www.scheme.com/tspl4/}, acesso em
    Setembro de 2018.
    
  \bibitem{norvig} Peter Norvig,
    ``Paradigms of Artificial Inteligence Programming - Case Studies
    in Common Lisp'',
    Morgan Kauffman Publishers, 1992.

  \end{thebibliography}

\end{document}
