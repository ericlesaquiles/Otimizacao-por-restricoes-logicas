%\documentclass{article}

%\usepackage[utf8]{inputenc}
\usepackage{placeins}
\usepackage{graphicx}
\usepackage[ruled,vlined,linesnumbered]{algorithm2e}
\usepackage{listings}
\usepackage{amsthm}
\usepackage{amsmath}
\usepackage{caption}
\usepackage{epigraph}

\captionsetup[figure]{font=scriptsize}

\newtheorem{definition}{Definição}[section]
\theoremstyle{remark}
\newtheorem*{remark}{Observações}
\theoremstyle{theorem}
\newtheorem{theorem}{Teorema}[section]

\setlength{\parskip}{.5em}

\renewcommand\refname{Leituras adicionais}
\renewcommand\figurename{Figura}
\renewcommand{\lstlistingname}{Código}

\newcommand{\eclipse}{$ECL^iPS^e$}
\newcommand{\definicao}[1]{\textbf{#1}\marginpar{\small \textbf{#1}}}

\lstdefinestyle{prosty}{
  language=Prolog,
  basicstyle=\small
}



%\begin{document}
\section{Predicados de Inspeção de Estrutura}

Predicados de inspeção de estrutura nos passam informações sobre um
termo específico. Por exemplo, será um dado termo atômico, numérico,
constante, variável? Ou será um funtor composto? Se for, qual será seu
funtor principal, qual sua aridade e quais seus argumentos? Os
predicados de inspeção de estrutura respondem esse tipo de questão.

\subsection{Predicados de tipos}

Alguns dos, assim chamados, predicados de tipo são:

\begin{itemize}
\item \funtor{integer/1};
\item \funtor{real/1};
\item \funtor{atom/1};
\item \funtor{compound/1};
\end{itemize}

Cada um deles pode ser interpretado como uma lista infinita de
átomos. Por exemplo, \funtor{integer/1} pode ser interpretado como:

\codigo{integer(0). integer(1). integer(2). integer(3), ...}.

\noindent A partir desses predicados podemos criar outros. Por
exemplo, podemos fazer

\codigo{numero(X) :- integer(X);real(X).}

\noindent ou

\codigo{constante(X) :- numero(X); atom(X).}

\subsection{Acesso de termos compostos}

Queremos ser capaz, além de lidar com tipos, lidar com funtores. Para
acessar o funtor principal, temos o predicado \funtor{functor/3}. O
goal \codigo{functor(Termo, F, Aridade)?} tem sucesso se o funtor
principal de \var{Termo} tem aridade \var{Aridade} e nome
\var{F}. Assim, por exemplo, \codigo{functor(f($X_1$, ..., $X_n$), f,
  n)}, onde os \var{$X_i$} são variáveis, tem sucesso, já que o funtor
principal é \funtor{f/n}.

Pode-se usar esse predicado para, entre outras coisas, realizar a
decomposição e criação de termos:
\begin{enumerate}
\item\codigo{functor(tio(a,b), X, Y)?} tem a solução \{\var{X} =
  tio, \var{Y} = 2\};
\item\codigo{functor(F, tio, 2)} tem a solução \{\var{F} = tio\}.
\end{enumerate}

Note que, no item 2 acima, se o goal fosse \codigo{functor(F, tio,
  N)?}, teríamos erro, já que o interpretador não consegue adivinhar a
aridade de um funtor a partir do nome. Analogamente,
\codigo{functor(F, tio(a, b), N)?} resultaria em erro, já que
\codigo{functor} espera um átomo como segundo argumento, não um funtor
composto. Mas, \codigo{functor(tio, X, N)?} teria sucesso, com \{X =
tio, N = 0\}.

Similar a \funtor{functor/3} é o \funtor{arg/3}: \codigo{arg(N,
  F($X_1$, ..., $X_n$), Q)?} tem sucesso se o Enésimo argumento de
\var{F} é o \var{Q}. Assim como functor, \funtor{arg/3} é comumente
usado para decomposição e criação de termos:
\begin{itemize}
\item Para decompor um termo, \funtor{arg/3} acha um argumento
  particular de um termo composto;
\item Para criar um termo, ele instancia um argumento variável de um
  termo.
\end{itemize}

Por exemplo, \codigo{arg(1, tio(a,b), X)?} tem sucesso com \{\var{X} =
a\}, enquanto que \codigo{arg(1, tio(X,b), a)?} tem sucesso com
\{\var{X} = a\}.

Exemplos um pouco mais interessantes são os seguintes:

\begin{listing}
  \inputminted{prolog}{../Exemplos/Cap6/prog1_subtermo.pl}
  \caption{Subtermo}
\end{listing}

\begin{listing}
  \inputminted{prolog}{../Exemplos/Cap6/prog2_substitui.pl}
  \caption{Substituto}
\end{listing}

Outro predicado de inspeção de estrutura é o, assim chamado (por
razões históricas obscuras), \technical{univ}, escrito como
\funtor{=../2}\footnote{Predicados para acesso e construção de termos
  têm origem na família Prolog de Edimburgo (que tem se tornado o
  padrão de fato) e alguns dos nomes usados vêm de lá. Em particular,
  a forma “=..” para \technical{univ} vem do Prolog-10, onde era usado
  “,..” no lugar de “$|$” em listas (\codigo{[a, b,.. Xs]} no lugar de
  \codigo{[a, b|Xs]}).}.  Um exemplo de seu uso é \codigo{Termo
  =.. [f,a,b]?}, que faz \codigo{Termo = f(a, b)}.  O \technical{univ}
pode ser usado de essencialmente duas formas diferentes:
\begin{enumerate}
\item Com um funtor ao lado esquerdo e uma variável no direito: a
  variável é unificada com \codigo{[f|v]}, onde f é o funtor
  principal e V a lista de seus argumentos;
\item Com uma variável ao lado esquerdo e uma lista no direito: se a
  lista é \codigo{[a, $b_1$, ..., $b_n$]}, a variável é unificada
  com \codigo{a($b_1$, ..., $b_n$)}.
\end{enumerate}

O seguinte programa mostra um exemplo da utilidade de
\technical{univ}:

\begin{listing}
  \inputminted{prolog}{../Exemplos/Cap6/prog3_apply.pl}
  \caption{Apply}
\end{listing}

\subsection{Predicados de Meta-programação}

Digamos que você queira fazer um interpretador de Prolog em Prolog
(algumas possíveis aplicações disso são \foreign{debuggers}). O mais
simples possível é dado a seguir:

\codigo{solve(Goal) :- Goal.}

Esse interpretador, sozinho, é de pouca utilidade. Nos dá a
possibilidade de acessar quase nada do programa. Se quisermos fazer
melhor, precisaremos de predicados que nos digam mais sobre predicados
(ou seja, meta-predicados, predicados de meta-programação).

Um predicado de meta-programação básico é o \funtor{clause/2}. O goal
\codigo{clause(Head, Body)?} é verdadeiro se \var{Head} for unificável
com a cabeça de uma cláusula e \var{Body} com seu respectivo corpo (se
\var{Head} for um fato, \var{Body} é unificável com \technical{true}).

Com isso, podemos fazer um meta-interpretador, algo mais elaborado:

\begin{listing}
  \inputminted{prolog}{../Exemplos/Cap6/prog4_meta1.pl}
  \caption{Meta 1}
\end{listing}

Precisamos dos parênteses a mais em \codigo{solve((A,B))} por razões
técnicas (não queremos confundir o compilador). Esse interpretador
teria que ser estendido se quisermos que faça tudo o que é esperado de
um interpretador Prolog (ele não lida apropriadamente com cortes, por
exemplo, ou com entradas pelo teclado). Isso poderia ser corrigido sem
grandes dificuldades.

O Prolog nos possibilita não só facilmente escrever interpretadores
para a própria linguagem (o que pode ser útil, por exemplo, na
construção de \foreign{debuggers}, de sistemas especializados, de
programas autocorretores, de programas que ``se explicam'' em termos
de porquês e comos entre outros), mas também nos possibilita
facilmente escrever interpretadores para outras linguagens e para
dialeto dessas. Por exemplo, alguém poderia argumentar que o Prolog,
sendo uma linguagem de programação lógica, não lida bem com situações
em que a incerteza é uma parte importante. Mas, com ele, podemos
facilmente criar nossa própria linguagem para fazer isso. Para tanto,
suponhamos, por exemplo, que cláusulas com uma certeza (a
probabilidade de estar correta) \var{C} sejam representadas pelo
funtor \funtor{clause\_c/3}, em que \codigo{clause\_c(Head,Body,C)?} é
verdade quando \var{Head} unifica com uma cabeça de cláusula cujo
corpo unifica com \var{Body} e cujo ``fator de certeza'' unifica com
\var{C}, e considere o seguinte:

\begin{listing}
  \inputminted{prolog}{../Exemplos/Cap6/prog5_meta2.pl}
  \caption{Meta 2}
\end{listing}

Um goal \codigo{solve(Goal,C,Limit)?} submetido a esse interpretador
resulta em sucesso se a ``confiança'' \var{C} de \var{Goal} for maior
do que o limite inferior \var{Limit}. Alguns pontos válidos de se
notar nesse programa são que, numa conjunção \codigo{A,B}, a nossa
confiança na conjunção é tomada como o mínimo entre a de \var{A} e a
de \var{B} e que se, para provar um goal \var{Goal}, é preciso passar
por uma cláusula com fator de certeza \var{C1}, a prova do corpo da
cláusula terá um fator de certeza menor, dado por
\codigo{Limit/C1}. Isto porque queremos que nossa ``confiança''
(pensando de forma probabilística) \var{C1} na cláusula vezes a
confiança \var{C2} (pense na probabilidade de eventos independentes)
na resolução do corpo da cláusula seja maior que \var{Limit}.

\subsubsection{Meta-predicados de variáveis}

Outro predicado de meta-programação básico é o \funtor{var/1}: o goal
\codigo{var(T)?} tem sucesso se \var{T} é uma variável não instanciada
e falha caso contrário. Sua irmã, \funtor{nonvar/1}, funciona de
maneira análoga. Por exemplo, \codigo{var(a)?} e \codigo{var([X|Xs])?}
falham, enquanto que \codigo{var(X)?} tem sucesso, se X é uma variável
não instanciada.  Esse predicado nos permite fazer, por exemplo,
programas como o seguinte:

\begin{listing}
  \inputminted{prolog}{../Exemplos/Cap6/prog6_glength.pl}
  \caption{Length mais geral}
\end{listing}

%TODO inserir refs e links
\noindent em que \funtor{length1/2} e \funtor{length2/2} são dados
pelos length no (Capítulo 4).

%TODO inserir bibliografia para solver

%\end{document}
