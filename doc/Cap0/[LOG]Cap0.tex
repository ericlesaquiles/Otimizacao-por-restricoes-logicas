\documentclass{article}

\usepackage[utf8]{inputenc}
\usepackage{placeins}
\usepackage{graphicx}
\usepackage[ruled,vlined,linesnumbered]{algorithm2e}
\usepackage{listings}
\usepackage{amsthm}
\usepackage{amsmath}
\usepackage{caption}
\usepackage{epigraph}

\captionsetup[figure]{font=scriptsize}

\newtheorem{definition}{Definição}[section]
\theoremstyle{remark}
\newtheorem*{remark}{Observações}
\theoremstyle{theorem}
\newtheorem{theorem}{Teorema}[section]

\setlength{\parskip}{.5em}

\renewcommand\refname{Leituras adicionais}
\renewcommand\figurename{Figura}
\renewcommand{\lstlistingname}{Código}

\newcommand{\eclipse}{$ECL^iPS^e$}
\newcommand{\definicao}[1]{\textbf{#1}\marginpar{\small \textbf{#1}}}

\lstdefinestyle{prosty}{
  language=Prolog,
  basicstyle=\small
}




\begin{document}

\section{Dando nome aos bois}

\epigraph{\cit{Foi a influência da filosofia grega que fez da matemática uma ciência. De fato, os primeiros matemáticos gregos estão entre os primeiros filósofos, como é o caso de Tales e Pitágoras. A noção de que um fato matemático pode ser demonstrado é fruto da interação entre matemática e filosofia. Afinal, uma demonstração é essencialmente um argumento para esclarecer como um certo fato é consequência de algo que já conhecemos. E se há alguma coisa que os filósofos gregos gostavam de
    fazer era argumentar.}}{S. C. Coutinho - Números Inteiros e Criptografia RSA}


Estamos para começar uma jornada cujo destino é a resolução e desenvolvimento de problemas de otimização por restrições lógicas.
Antes de sairmos, precisamos reunir algumas ferramentas que serão de grande utilidade. A mais importante dessas ferramentas, que
será a base de muitas outras, é a programação lógica. Na verdade, programação lógica é um paradigma de programação, como o de
programação funcional, procedural e etc., mas, em princípio, lidaremos com ela como uma linguagem: Prolog (que vem de \foreign{``Programmation Logique''}). Isso nos
dá a vantagem de lidar com algo concreto enquanto perdemos muito pouco, se algo, do poder de abstração. Um detalhe que vale nota
é que existem vários diferentes ``sabores'' de Prolog, cada um com suas peculiaridades, na maior parte de natureza técnica.
Ignoraremos essas peculiaridades sempre que possível (este não tem a intenção de ser um texto técnico sobre Prolog) nos focando
no ``padrão de facto'' Edinburgh Prolog.

Como dizia Sussman \cite{sussman}
, quando se quer aprender uma nova linguagem, as perguntas básicas que precisamos perguntar são

\begin{enumerate}
  \item Quais são as ``coisas'' primitivas da linguagem?
  \item Quais são seus meios de combinação com os quais podemos usar as primitivas de maneira estruturada?
  \item Quais são seus meios de abstração com os quais podemos criar programas cada vez maiores?
\end{enumerate}

A resposta à primeira pergunta é a seguinte. O Prolog tem uma primitiva: o \definicao{funtor}. Para uma melhor compreensão sobre o que é um
funtor é útil sabermos a origem do termo. O termo \technical{funtor} foi introduzido por Rudolf Carnap, filósofo alemão que
participou do círculo de Viena, em seu \foreign{Logische Syntax der Sprache}\cite{carnap} e indica uma \enphasis{palavra função}
(não confundir com ``uma função''), que contribui primariamente com a sintaxe de uma sentença (em contraste com \enphasis{palavras conteúdo},
que contribuem primariamente com o significado): resumidamente, em sua concepção original, funtores expressam a estrutura
relacional que palavras tem umas com as outras. O termo atualmente também é usado em outras áreas (além de em linguística e
programação lógica), como em \enphasis{Teoria de Categoria}, com significado semelhante (levando em conta o contexto).

Para nossos propósitos, é uma simplificação razoável dizer que funtores expressam relações. Na linguagem natural, somos restritos a usar os funtores presentes na língua usada (mesmo uma ``licença poética'' não vai muito longe disso).
Na programação lógica, os funtores só precisam ser sintaticamente corretos. Em Prolog, um funtor
\enphasis{f} de aridade \enphasis{n} (aridade é a quantidade de parâmetros, ou ``símbolos relacionados pelo funtor''), dito
\funtor{f/n}, é escrito com uma sintaxe semelhante à de uma função:

\codigo{
  f($a_1$, ..., $a_n$)
}

\noindent em que os argumentos $a_i$ são outros funtores. Em particular, símbolos e números são ditos funtores
de aridade zero, também chamados \definicao{átomos}\footnote{Em Prolog, nem todo símbolo pode ser
  tratado como átomo (embora os usuais sejam). Quando na dúvida, recomendamos testar antes de
  confiar.}. Quando um funtor \funtor{f/n} não é um átomo, ele, na presença de seus argumentos,
é dito um \definicao{termo composto} de \definicao{funtor principal} f. Mais em geral, um \definicao{termo} é um funtor ou variável (o que é uma variável e como se comporta  veremos mais para frente) e, quando corresponder a um átomo ou uma variável, dizemos que ele é atômico.

Dados funtores \funtor{f/3} e \funtor{p/1}, exemplos de seus usos são:

\begin{itemize}
  \item f(a,b,c)
  \item f(1,a,8)
  \item f(p(9),c,e)
\end{itemize}

Os exemplos acima são úteis como primeiro exemplo de ``como escrever um funtor'' mas, como colocados, não têm significado. Isto
porque funtores expressam relações, mas relações arbitrárias entre símbolos e números arbitrários não têm necessidade de existir.
A maneira de dizermos que uma dada relação existe em Prolog é por meio de um \definicao{fato}. Um fato em Prolog é expresso como
um funtor, junto de seus argumentos, seguido de um ponto. Por exemplo

\codigo{
  mais(1,2,3).
}

\noindent diz que ``é verdade que mais(1,2,3)'' ou, mais naturalmente, ``é verdade que existe a relação 'mais' entre 1, 2 e 3, nessa ordem'' (vale notar que a ordem é importante: \codigo{mais(1,2,3)} não é o igual a \codigo{mais(3,2,1)}, em que por ``igual a'' entende-se ``idêntico a'').
Ocasionalmente, é muito mais conveniente que um funtor receba argumentos pela direita e pela esquerda, quase como um operador,
como \codigo{1 f 2}. É possível definir funtores dessa forma, mas nos restringiremos ao modo usual, pelo qual a última relação fica
\codigo{f(1,2)}.

Eventualmente pode ser que, na ocasião de a relação \codigo{p(a,b)} ser verdadeira, outras relações também sejam.
Em particular, pode ser que, em linguagem matemática, $ f(1,l) \Rightarrow p(a,b) $. Esse tipo de relação é escrita em Prolog como:

\codigo{
  p(a,b) :- f(1,l).
}

\noindent ou, se $ (f(1,l)$ e $q(f)) \Rightarrow p(a,b) $, escrevemos

\codigo{
  p(a,b) :- f(1,l), q(f).
}

\noindent na qual as vírgulas entre os termos fazem o papel do ``e'' lógico. Esse tipo de estrutura é chamada \definicao{regra}
e é o mais importante meio de combinação em programação lógica. O que está para a esquerda de ``:-'' em uma regra é dita
``cabeça da regra'', ou só ``cabeça'', se a regra for clara do contexto, e o que está à direita de ``:-'' é dito o
``corpo da regra''. Quando não for necessário distinguir entre fatos e regras, usaremos o termo
\definicao{cláusula}\footnote{Note que usamos ``cláusula'' em um sentido diferente do da lógica
  clássica (em que ``cláusula'' é definida como sendo uma disjunção de proposições). } para
nos referir a qualquer um dos dois.

Um \definicao{programa lógico} consiste em um conjunto de cláusulas. Um exemplo é o seguinte:\\

    \begin{listing}
\inputminted{prolog}{../Exemplos/Cap0/prog1_cafe.pl}
    \caption{Café}\label{lst:cafe}
    \end{listing}

Note que os nomes usados para os funtores são arbitrários. O seguinte programa tem essencialmente o mesmo significado do ponto de vista computacional:\\

%\lstinputlisting[caption=Queijo, style=prosty]{../Exemplos/Cap0/prog2_queijo.pl}
    \begin{listing}
\inputminted{prolog}{../Exemplos/Cap0/prog2_queijo.pl}
\caption{queijo}\label{lst:queijo}
    \end{listing}

%QUESTION: o pao no primeiro codigo (e o queijo, no segundo) aparece primeiro com 1 argumento e depois com 2. Eh isso mesmo?

%COMMENT: seria bom ter uma legenda no codigo, para o leitor saber o que quer dizer com "Codigo 1".

Pelo código \ref{lst:cafe}, por exemplo, podemos dizer que ``é verdade que pao(de, queijo) e que com(cafe)''. De fato, a pergunta mais
geral que podemos fazer a um programa lógico é se existe alguma relação $r$ entre os termos $a_1$, ..., $a_n$.

E essa constitui a maneira primária de se utilizar um programa lógico, que pode ocorrer, por
exemplo, em um \foreign{prompt} no computador. Por meio de questões, ou buscas \footnote{Porque
  busca é um termo apropriado deve ficar cada vez mais claro no decorrer do texto.}, às vezes também
chamadas objetivos ou ``\definicao{goals}'' (daqui para frente preferiremos usar ``goal'', sem
itálico %COMMENT: veja que nao usou em italico antes
, exceto quando for conveniente utilizar ``busca'' ou ``objetivo''). Um goal é escrito como
um fato e é o que se busca ``provar'' a partir do programa. Intuitivamente, pode-se pensar no
programa lógico como um conjunto de axiomas e no goal como uma hipótese que se quer provar a partir desses axiomas.

Assim como dizemos que uma busca foi um sucesso se foi encontrado o que se gostaria de encontrar e uma falha caso contrário,
diremos que um goal relativo a algum programa pode ter o status de sucesso se ele pode ser provado a partir do programa, e de
falha se não. Note que, em Prolog, ``falhar'' não é o mesmo que ``quebrar''. A diferença é
essencialmente a mesma entre receber uma resposta de ``não'' a uma pergunta e receber uma resposta de
``não entendo sua pergunta''.  Se foi um goal mal escrito (se o compilador ``não entende a
pergunta''), pode ocorrer um erro (isto é, o goal gera um erro quando não é ``compreensível'' a
partir da gramática utilizada pelo compilador usado)\footnote{Se um goal que gera erro é de fato um
  goal, é discutível, mas não entraremos nesse mérito.} e o programa pode não ser executado.

Mais especificamente, um goal é uma conjunção\footnote{Isto é, uma sequência de proposições unidas
  por um \enphasis{e} lógico. Algo como \funtor{p\_1} e
  ... e \funtor{p\_n}, em que os \funtor{p\_i} são proposições.}, escrita como \codigo{$p\_1,p\_2,...,p\_n$}, em que $p_i$ é
entendida como uma proposição que pode ser verdadeira ou falsa \footnote{Na verdade, como veremos,
  uma interpretação mais próxima da realidade do programa lógico é uma baseada na lógica
  construtivista, mas, por enquanto, é suficiente pensar em um goal como uma conjunção no sentido
  clássico.} (ou, interpretando como busca, que pode ter sucesso ou falha).

Um ponto interessante sobre goals e também sobre regras é que são como cláusulas de
Horn\footnote{Também, apesar de mais raramente, chamadas \enphasis{cláusulas de McKinsey}. Elas
  foram usadas primeiramente por McKinsey, como notado por Horn \cite{horn}. É importante notar que
  a ``cláusula'' usada aqui não é a mesma ``cláusula'' definida anteriormente, mas a ``cláusula'' da
lógica clássica} (isto é, cláusulas do
tipo C se $A_1$ e ... e $A_n$). Como pode imaginar, a teoria existente sobre cláusulas de Horn é de
alguma importância para programas lógicos. Mais especificamente, dada uma
disjunção\footnote{Proposições unidas por ``ou''.} com ao menos um termo não negado, esta é dita uma
\enphasis{cláusula de Horn}. Se existe exatamente um termo não negado, a seguinte identidade é de
simples constatação:

\[
  X_1 \vee \neg X_2 \hdots \neg X_n \Leftrightarrow [(X_2 \wedge \hdots X_n) \Rightarrow X_1]
\]

\noindent em que o símbolo $\neg$ corresponde à negação lógica usual.

Para melhor diferenciarmos goals de fatos, goals serão, aqui, escritos como:
\codigo{
  p?
}

\noindent apesar de, na natureza\footnote{Isto é, no Prolog padrão.}, não haver essa distinção (neles, goals são, como fatos,
escritos com um ponto final no lugar de com um ponto de interrogação). Na verdade, a falta de distinção entre
goals, fatos e termos de uma cláusula, como deve ficar mais claro nos próximos capítulos, faz
sentido, já que no processo de resolução (que será explicado depois) eles cumprem papel semelhante.



Agora, considere o seguinte código:\\

    \begin{listing}
\inputminted{prolog}{../Exemplos/Cap0/prog3_arvb.pl}
\caption{Árvore Binária}\label{lst:arvb}
    \end{listing}

O goal \codigo{$arvore\_b$(vazio)?} tem sucesso, já que é um dos fatos. O goal \codigo{$arvore\_b$(arvore(raiz, vazio, vazio))?} também tem, já que \codigo{$arvore\_b$(a, B, C)?} tem sucesso se $arvore\_b(A)$ e $arvore\_b(B)$ tiverem, o que ocorre. Perceba que, neste caso, o goal não tem sucesso segundo a primeira cláusula apenas e nem segundo a segunda. Ambas contribuem para a definição de $arvore\_b$.

Utilizamos sempre a convenção de que termos capitalizados (isto é, com inicial maiúscula) denotam
variáveis, enquanto os demais denotam constantes (a leitora descobrirá que, por notável
coincidência, o Prolog faz uso da mesma convenção). Termos que não
contém variáveis são ditos \definicao{termos base}. Variáveis serão explicadas mais a fundo no futuro.

No geral, o que um programa lógico deveria fazer (isto é, o que a programadora tem em mente ao escrevê-lo) pode não ser a princípio claro. Numa tentativa de aliviar isso, usaremos aqui a convenção de usar nomes significativos, assumindo o significado usual (a não ser quando dito o contrário), para os elementos relevantes. Assim, por exemplo, o seguinte trecho:

    \begin{listing}
\inputminted{prolog}{../Exemplos/Cap0/prog4_pai.pl}
\caption{Pai e Filho}\label{lst:pai_filho}
    \end{listing}


Indica que Pai tem a relação \enphasis{pai} com Filho se Filho tem a relação \enphasis{filho} com
Pai (em outras palavras, um Pai é pai de um Filho se Filho é filho de Pai). Mas não é assumido que o
interpretador do programa tenha conhecimento sobre o que é \technical{pai} ou \technical{filho}, ou
seja, essa interpretação é relevante para a leitora do programa apenas (o que também significa que a
leitora do programa pode ler \technical{pai} e \technical{filho} de várias formas e um programa pode, assim,
ser interpretado de diversas maneiras).

Com essa discussão em mente, será ocasionalmente útil ter o \enphasis{significado} de um programa lógico definido de forma algo mais precisa:

  \theoremstyle{definition}
  \begin{definition} \definicao{Significado de um programa lógico} é o conjunto de goals deduzíveis a partir dele (isto é, os goals que obtêm sucesso se aplicados ao programa).
  \end{definition}

  Na verdade, essa é a definição procedural do significado de um programa lógico (apesar de que, como veremos posteriormente, essa distinção é imaterial) e, vale notar, ela é sempre bem definida (isto é, todo programa lógico tem um significado bem definido). Nesse contexto, o significado intencionado pela programadora (isto é, ``o que ela quer dizer com o programa'') é um conjunto de goals que pode ou não estar contido no significado do programa. Assim, podemos nos perguntar ``Será que o
  programa diz tudo que se quer que ele diga?'' (isto é, se o significado intencionado está contido no significado do programa), ou ``Será que tudo o que ele diz é correto?'' (isto é, se o significado do programa está contido no intencionado). No primeiro caso, se o significado intencionado estiver contido no do programa, dizemos que o programa é \definicao{completo} e, no segundo, que ele é \definicao{correto}.

  Por exemplo, digamos que o programa \ref{lst:pai_filho}  seja um trecho de um programa no qual a programadora deseja modelar as relações entre programas e especificações de software atuais e antigos, de forma que A é pai de B se A veio antes de B e B herda características de A, como partes do código ou especificações (como é possível ver, ela apenas começou a escrever o programa). Assim, estão no significado intencionado goals como \codigo{pai(gnu, linux)?} e \codigo{pai(dos, windows)?}, enquanto que o
  significado do programa é \codigo{pai(d,c)?}.

  Fica a pergunta: esse programa é correto? E ele é completo?


  \begin{thebibliography}{1}

    \bibitem{carnap}
     Carnap, Rudolf,
     Logische Syntax der Sprache,
     Wien (Viena): Julius Springer (1968)

     \bibitem{horn}
     Horn, Alfred,
     On Sentences Which are True of Direct Unions of Algebras,
     J. Symbolic Logic 16 (1951),
     no. 1,
     pp 14

     \bibitem{sussman}
     Abelsson, Harold and Sussmann, Gerald J. and Sussman Julie
     Structure and Interpretation of Computer Programs, second edition
     MIT Press

  \end{thebibliography}

\end{document}
