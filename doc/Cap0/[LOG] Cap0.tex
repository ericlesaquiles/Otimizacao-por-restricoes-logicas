\documentclass{article}
\usepackage[utf8]{inputenc}
\usepackage[portuguese]{babel}
\usepackage[utf8]{inputenc}
\usepackage{placeins}

\usepackage{amsthm}

\theoremstyle{definition}
\newtheorem{definition}{Definição}[section]

\theoremstyle{remark}
\newtheorem*{remark}{Remark}


%
% TODO: revisar terminar esta seção!
%
%


\begin{document}
\section{Dando nome aos bois}

Neste capítulo serão explicadas algumas noções mais ou menos simples, assim como definidos termos a serem utilizados posteriormente. Vale notar que alguns dos termos usados aqui também são usados em outros contextos, mas com significado diferente. As definições aqui colocadas são referentes apenas à utilização no presente documento.\par
É uma quantidade relativamente grande de definições que a leitora poderia, possívelmente, reconhecer de capítulos posteriores e está posto aqui para referência. Dependendo de seu estilo, talvez seja considere mais interessante começar do capítulo 1: Continue por sua conta e risco.



  \theoremstyle{definition}
  \begin{definition}{Átomo}
  é algo que seja uma:
    \begin{itemize}
      \item Sequência de caracteres entre aspas simples (\textit{'Qualquer coisa assim'});
      \item Ou uma sequência de caracteres contendo apenas caracteres alfabéticos, numéricos e \textit{underscore} ('\_') começando com um caractere alfabético minúsculo;
      \item Ou uma sequência contínua de caracteres dos símbolos: +, -, *, /, \, \^, >, <, =, ', :, ., ?, @, \#, \$, \&\. (e.g. ***+***+@)
      \item Ou algum dos seguintes átomos especiais: [], {}, ;, !.
    \end{itemize}
  \end{definition}


  \theoremstyle{definition}
  \begin{definition}{Funtor}
    é um átomo seguido de $n$ argumentos fechados por parênteses e separados por vírgula:
    \[
      p(a_1, ..., a_n)
    \]
    Onde $p$ é um átomo e $a_1$ a $a_n$ são quaisquer termos. Quando $n = 0$, não são colocados parênteses e o funtor é apenas um átomo. Quando $n > 0$, o funtor é chamado \textit{termo composto} e $p$ é chamado \textit{funtor principal} (essa definição faz sentido porque cada um dos argumentos pode, por sua vez, ser um outro funtor).
  \end{definition}
Vale notar que os funtores são diferenciados por nome e aridade (quantidade de parâmetros). Assim, os funtores a seguir são distintos:
    \[
      f(a_1, a_2)
    \]
    \[
      f(a_1)
    \]

    Um funtor de aridade \textit{n} e nome \textit{p} será denotado \textit{p/n}.



  \theoremstyle{definition}
  \begin{definition}{Termo}
     é ou um funtor (composto ou não), uma variável (o que são variáveis e como funcionam será explicado adiante), ou um número (em especial, lidaremos com números inteiros ou reais).
  \end{definition}

  Quando um termo for um átomo ou um número, diremos que ele é \textbf{Atômico}.

  \footnote{ O termo \textit{funtor} foi introduzido por Rudolph Carnap, filósofo alemão que participou do círculo de Vienna, em seu \textit{Logische Syntax der Sprache} e indica uma \textit{palavra função}, que contribui primariamente com a sintaxe de uma sentença (em contraste com \textit{palavras conteúdo}, que contribuem primariamente com o significado): resumidamente, em sua concepção original, funtores expressam a estrutura relacional que palavras tem umas com as outras. O termo atualmente
    também é usado em outras áreas (além de em linguística e programação lógica), como em \textit{Teoria de Categoria}, com   significado semelhante (é claro, levando em conta o contexto).  }





\end{document}
