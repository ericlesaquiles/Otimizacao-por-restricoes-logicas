\documentclass{article}

\usepackage[utf8]{inputenc}
\usepackage{placeins}
\usepackage{graphicx}
\usepackage[ruled,vlined,linesnumbered]{algorithm2e}
\usepackage{listings}
\usepackage{amsthm}
\usepackage{amsmath}
\usepackage{caption}
\usepackage{epigraph}

\captionsetup[figure]{font=scriptsize}

\newtheorem{definition}{Definição}[section]
\theoremstyle{remark}
\newtheorem*{remark}{Observações}
\theoremstyle{theorem}
\newtheorem{theorem}{Teorema}[section]

\setlength{\parskip}{.5em}

\renewcommand\refname{Leituras adicionais}
\renewcommand\figurename{Figura}
\renewcommand{\lstlistingname}{Código}

\lstdefinestyle{prosty}{
  language=Prolog,
  basicstyle=\small
}



%\usepackage{pgfgantt}

\begin{document}

\section{O Problema do General Bárbaro}

  Para ver se realmente compreendemos as ideias que vimos anteriormente, é sempre útil ver se
  conseguimos aplicá-las a algo concreto. Para isso, desenvolveremos aqui um problema específico (o
  ``problema do general bárbaro''\footnote{Também conhecido como ``o problema da mochila'', ou
  \foreign{knapsack problem}, em inglês.}). Mas faremos isso não com vistas em colocá-lo diretamente no
\foreign{framework} desenvolvido até agora, mas sim em entendê-lo, pensar em diversos métodos de
resolvê-lo e comparar esses métodos.

\subsection{O Problema}

O general romano Sulla, em suas campanhas pela Grécia, fez (entre outras ocisas) o que qualquer
outro general bárbaro faria em seu lugar, isto é, saqueou tesouros locais para obter vantagem na
guerra (diferente de outros
generais bárbaros, Sulla prometeu devolver parte do saque). Mas isso constituiu um problema,
já que ele tinha uma grande variedade de tesouros para escolher levar ou não, enquanto que a
capacidade de carga de seu exército era muito limitada, umas vez que dispunha de
(relativamente) poucos navios. Mas o que criava a maior dificuldade nesse projeto era não o volume
dos tesouros, mas o seu peso.

O valor de cada peça de tesouro é dado por uma função complicada de interesses internos e externos
(no que influem, entre outros fatores, o preço pelo qual Sulla poderia vender dado item, no valor que cada
soldado atribui ao item, no valor que seus conterrâneos atribuiriam ao item e na reputação que
carregar um item lhe conferiria) e
é sumarizada no que chamamos de \enphasis{valor Sulla}, $\mathscr{S}$, de um item. Depois de calcular
o $\mathscr{S}$ de cada item relevante, Sulla compilou a seguinte tabela preliminar com o peso médio e valor
médio de cada item\footnote{Não há evidências de que tal tabela tenha sido feitas, mas é mantida
  aqui pela conveniência ao exemplo.}:


\begin{center}
  \begin{tabular}{c|c|c|c}
    Item              & Valor  & Peso  & Quantidade\\
    \hline
    Livros            & 10     & 5     & 1000 \\
    Barras de Ouro    & 100    & 500   & 85 \\
    Estátuas de Ouro  & 500    & 4000  & 7 \\
    Jóias             & 50     & 50    & 200 \\
  \end{tabular}
\end{center}

Sem muito sucesso em resolver o problema, Sulla mandou que chamassem um matemático local para
resolver o problema. Theon de Smyrna, estava por perto e foi convocado. No início, se sentiu
incomodado no por ter que deixar de lado seu trabalho com os sólidos de Platão em favor de um
ajudar um bárbaro a resolver um problema desse tipo inferior, como considerava, mas,
notando que pode ser mais difícil para um morto desenvolver trabalhos com
os sólidos de Platão, decidiu ajudar Sulla com seu problema.

Ao ter a situação explicada, Theon deu uma leve risada e disse o que se segue. ``Você me dá um
problema grandioso e difícil, oh grande Sulla! Os Deuses devem ter nos posto juntos, porque eu devo
ser um dos poucos helenos, se não o único, que sabe resolvê-lo! Isto porque vi meu pai o resolvendo
quando era criança. Espero que não se incomode em ouvir por alguns momentos um velho como eu
relembrar a infância, ainda mais que isso te será útil.''

``Continue.'' - disse Sulla.

``Meu pai era um fazendeiro muito forte, mas também tinha o seus limites. Ás vezes, precisava
realizar trabalhos em partes distantes e, para isso, precisava levar suas ferramentas. Cada
ferramenta faria o seu trabalho mais fácil, ou o ajudaria a terminá-lo mais rápido, ou lhe
proporcionaria maior conforto ao realizá-lo. Cada uma, também, tinha o seu peso, e meu pai, forte
como era, não conseguia carregar todas. Então, ele compilou uma tabela parecida com a sua, com um
``valor'' que cada ferramenta lhe proporcionaria, assim como seu peso.''

``Mas, meu caro, esse problema parece muito mais simples do que o meu: seu pai só precisava decidir
entre levar uma ferramenta ou não, enquanto que eu preciso decidir entre quantos de cada tipo de
item levar!'' - disse Sulla.

``É verdade que o problema de meu falecido pai parece muito mais simples do que o seu. Mas toda
pessoa bem educada, como tenho certeza ser o seu caso, oh conquistador de nações!, sabe que não se pode confiar nas aparências.
Neste caso, ver que os problemas são essencialmente os mesmos é muito simples. Toda criança sabe que
qualquer número natural pode ser decomposto como uma soma de potências de dois. Por exemplo, $11 = 2^0
+ 2^1 + 2^3$. Além disso, essa decomposição é única, cada potência aparecendo no máximo uma vez.
Como pode ver, então, o seu problema é um problema de decisão: decidir quais potências de 2 comporão
as quantidades de cada tesouro que levará. Esse problema só tem o posssível incoveniente de fazer
uso de mais variáveis. Mas não muito mais: $\floor{log_2(q_i)}$, em que $q_i$ é a quantidade a ser
levada do item $i$.\footnote{Estamos adicionando, aqui e mais adiante, notação um pouco mais moderna
  do que a que o velho Theon poderia ter usado, por conveniência ao exemplo.}''

``Mas esse problema adiciona complicações extras que não me parecem tão necessárias.'' - disse
Sulla.

``Não tanto assim, mas tudo bem. Não quero tomar muito mais tempo seu, então vou direto ao ponto. Um
modelo mais geral ao problema que deseja resolver é o seguinte:

\begin{align*}
  maximizar \;& \sum\limits_{i=1}^n q_i \mathscr{S}_i,\\
  tal\; que\;& \sum{q_i p_i} \leq C,\\
  0 \leq q_i \leq l_i,\; & q_i \; inteiro
\end{align*}

Em que $q_i$ é a quantidade a ser levada do item $i$, $p_i$ o peso do item $i$ (em
$\mathscr{S})$, $C$ a capacidade máxima a ser levada e $l_i$ limite máximo do item $i$ a ser
levado (não podemos levar mais do que $l_i$ do item i, talvez porque não existam mais do que $l_i$
desses itens).

Esse é um problema de programação inteira. Existem muitas formas de resolver problemas de
programação inteira. Se tiver algum tempo sobrando, eu poderia desenvolver uma outra forma com
você. Será divertido e intrutivo.''

``Vá em frente.'' - disse Sulla.

``Imagine que você sabe resolver esse problema para $C' < C$ e para todo subconjunto dos itens (um
subconjunto de $\{1, ..., n\}$) e tenha em mãos uma solução para ele. Se retirarmos dessa solução um item $i$, de peso $p_i$, deve ser
claro que a solução restante é ótima para o problema cuja capacidade de $C - p_i$ e conjunto de
itens $\{1, ..., n\}-i$. Isso indica que uma solução ótima goza de uma subestrutura ótima. Isto é, o
retiro de itens dessa solução gera uma solução que é ótima para outro problema, de fato um
subproblema, mais simples. Se temos a solução para algum desses subproblemas, só precisamos
expandi-lo, considerando itens que não foram considerados. Para cada um desses itens, sabendo que a
nova solução ótima $op(i,c)$ para uma capacidade $c$, ao considerar a adição de um item $i$, é:

\[
  op(i,c) =
  \begin{cases}
    op(i-1, c),  & se\; p_i > c\\
    max\{op(i-1, c), op(i-1, c-p_i) + \mathscr{S}_i\} & se\; p_i \leq c
  \end{cases}
\]

Fazendo $op(0, C) := 0$ e computando os valores de $op(i, c)$ por essa recursão\footnote{Conhecida
como recursão de Bellman} nos dá o valor ótimo que o meu pai queria. Se quisermos resolver o seu
problema, precisamos fazer apenas uma pequena alteração, como a seguinte:

\[
  op(i,c) =
    max_{q_i}\{op(i-1, c-q_i p_i) + q_i \mathscr{S}_i\; |\; c \geq q_i p_i\}
\]

Não deve ser difícil de ver como isso nos leva ao resultado desejado.''

``Oh, mas Theon! Devo ter me enganado ao achar que era um matemático, pois você me está parecendo
  mais um ator de teatro! Isso porque qualquer ator de teatro me daria uma resposta como essa, mas
  ela é longe de satisfatória. Mas é pior do que um ator de teatro, porque quem quer que fosse
  incubido de realizar a tarefa pelo método que você propôs ficaria um tempo intolerável fazendo
  contas, tempo esse que é, de fato, morto. E você seria o responsável pela morte.'' - disse Sulla.

  Theon, percebendo a gravidade da situação, disse o seguinte. ``Essa formulação, como eu disse
  antes, tinha um caráter mais introdutório e instrutivo. Se quer uma mais séria, existem várias e
  em vários sabores. Uma delas em particular faz uso extenso de técnicas de programação inteira, que
  são desenvolvidas a muitas décadas\footnote{Infelizmente, não há evidências de desenvolvimentos de
  programação inteira na época de Theon, mas a fala é mantida aqui por sua conveniência.}. Fazendo
  uso dessas técnicas, podemos gerar o seguinte programa:

  \begin{listing}
    \inputminted{prolog}{../Exemplos/Cap12/prog1-knapSackEplex.ecl}
    \caption{Conquistador Bárbaro Eplex}\label{lst:knapsackEplex}
  \end{listing}

  Podemos, analogamente, fazer uso da extensa pesquisa em otimização por restrições, com algo como o
  seguinte programa:

  \begin{listing}[H]
    \inputminted{prolog}{../Exemplos/Cap12/prog2-knapSackIc.ecl}
    \caption{Conquistador Bárbaro IC}\label{lst:knapsackEplex}
  \end{listing}

  Para saber qual dos métodos é o mais interessante, podemos simplesmente tomar uma pequena quantidade dos itens
  iniciais, fazer um teste, e ver qual resolvemos mais rápido.''








  \begin{thebibliography}{2}

    \bibitem{kellerer}
    Kellerer, Hans e Pferschy, Ulrich e Pisinger, David,
    ``Knapsack Problems'', Springer

    \bibitem{plutarch}
    Plutarch,
    ``Sulla'', traduzido por John Dryden


  \end{thebibliography}

\end{document}
