\documentclass{article}

\usepackage[utf8]{inputenc}
\usepackage{placeins}
\usepackage{graphicx}
\usepackage[ruled,vlined,linesnumbered]{algorithm2e}
\usepackage{listings}
\usepackage{amsthm}
\usepackage{amsmath}
\usepackage{caption}
\usepackage{epigraph}

\captionsetup[figure]{font=scriptsize}

\newtheorem{definition}{Definição}[section]
\theoremstyle{remark}
\newtheorem*{remark}{Observações}
\theoremstyle{theorem}
\newtheorem{theorem}{Teorema}[section]

\setlength{\parskip}{.5em}

\renewcommand\refname{Leituras adicionais}
\renewcommand\figurename{Figura}
\renewcommand{\lstlistingname}{Código}

\newcommand{\eclipse}{$ECL^iPS^e$}
\newcommand{\definicao}[1]{\textbf{#1}\marginpar{\small \textbf{#1}}}

\lstdefinestyle{prosty}{
  language=Prolog,
  basicstyle=\small
}



%\usepackage{pgfgantt}

\begin{document}

\section{O Problema do General Bárbaro}

Para ver se realmente compreendemos as ideias que vimos
anteriormente, é sempre útil ver se conseguimos aplicá-las a algo
concreto. Para isso, desenvolveremos aqui um problema específico (o
``problema do general bárbaro''\footnote{Também conhecido como ``o
  problema da mochila'', ou \foreign{knapsack problem}, em
  inglês.}). Mas faremos isso não com vistas em colocá-lo
diretamente no \foreign{framework} desenvolvido até agora, mas sim
em entendê-lo, pensar em diversos métodos de resolvê-lo e comparar
esses métodos.

\subsection{O Problema}

O general romano Sulla, em suas campanhas pela Grécia, fez (entre
outras coisas) o que qualquer outro general bárbaro faria em seu
lugar, isto é, saqueou tesouros locais para obter vantagem na guerra
(diferente de outros generais bárbaros, Sulla prometeu devolver parte
do saque). Mas isso constituiu um problema, já que ele tinha uma
grande variedade de tesouros para escolher levar ou não, enquanto que
a capacidade de carga de seu exército era muito limitada, umas vez que
dispunha de (relativamente) poucos navios. Mas o que criava a maior
dificuldade nesse projeto era não o volume dos tesouros, mas o seu
peso.

O valor de cada peça de tesouro é dado por uma função complicada de
interesses internos e externos (no que influenciam, entre outros
fatores, o preço pelo qual Sulla poderia vender dado item, no valor
que cada soldado atribui ao item, no valor que seus conterrâneos
atribuiriam ao item e na reputação que carregar um item lhe
conferiria) e é sumarizada no que chamamos de \enphasis{valor Sulla},
$\mathscr{S}$, de um item. Depois de calcular o $\mathscr{S}$ de cada
item relevante, Sulla compilou a seguinte tabela preliminar com o peso
médio e valor médio de cada item\footnote{Não há evidências de que tal
  tabela tenha sido feita, mas é mantida aqui pela conveniência ao
  exemplo.}:


\begin{center}
  \begin{tabular}{c|c|c|c}
    Item & Valor & Peso & Quantidade\\ \hline Livros & 10 & 5 & 1000
    \\ Barras de Ouro & 100 & 500 & 85 \\ Estátuas de Ouro & 500 &
    4000 & 7 \\ Jóias & 50 & 50 & 200 \\
  \end{tabular}
\end{center}

Sem muito sucesso em resolver o problema, Sulla mandou que chamassem
um matemático local para resolvê-lo. Theon de Smyrna estava por perto
e foi convocado. No início, se sentiu incomodado por ter que deixar de
lado seu trabalho com os sólidos de Platão em favor de ajudar um
bárbaro a resolver um problema desse tipo inferior, como considerava,
mas, notando que pode ser mais difícil para um morto desenvolver
trabalhos com os sólidos de Platão, decidiu ajudar Sulla com seu
problema.

Ao ter a situação explicada, Theon deu uma leve risada e disse o que
se segue. ``Você me dá um problema grandioso e difícil, oh grande
Sulla! Os Deuses devem ter nos posto juntos, porque eu devo ser um dos
poucos helenos, se não o único, que sabe resolvê-lo! Isto porque vi
meu pai o resolvendo quando era criança. Espero que não se incomode em
ouvir por alguns momentos um velho como eu relembrar a infância, ainda
mais que isso te será útil.''

``Continue.'' - disse Sulla.

``Meu pai era um fazendeiro muito forte, mas também tinha o seus
limites. Às vezes, precisava realizar trabalhos em partes distantes e,
para isso, precisava levar suas ferramentas. Cada ferramenta faria o
seu trabalho mais fácil, ou o ajudaria a terminá-lo mais rápido, ou
lhe proporcionaria maior conforto ao realizá-lo. Cada uma, também,
tinha o seu peso, e meu pai, forte como era, não conseguia carregar
todas. Então, ele compilou uma tabela parecida com a sua, com um
``valor'' que cada ferramenta lhe proporcionaria, assim como seu
peso.''

``Mas, meu caro, esse problema parece muito mais simples do que o meu:
seu pai só precisava decidir entre levar uma ferramenta ou não,
enquanto que eu preciso decidir entre quantos de cada tipo de item
levar!'' - disse Sulla.

``É verdade que o problema de meu falecido pai parece muito mais
simples do que o seu. Mas toda pessoa bem educada, como tenho certeza
ser o seu caso, oh conquistador de nações!, sabe que não se pode
confiar nas aparências.  Neste caso, ver que os problemas são
essencialmente os mesmos é muito simples. Toda criança sabe que
qualquer número natural pode ser decomposto como uma soma de potências
de dois. Por exemplo, $11 = 2^0 + 2^1 + 2^3$. Além disso, essa
decomposição é única, cada potência aparecendo no máximo uma vez.
Como pode ver, então, o seu problema é um problema de decisão: decidir
quais potências de 2 comporão as quantidades de cada tesouro que
levará. Esse problema só tem o posssível incoveniente de fazer uso de
mais variáveis. Mas não muito mais: $\floor{log_2(q_i)}$, em que $q_i$
é a quantidade a ser levada do item $i$.\footnote{Estamos adicionando,
  aqui e mais adiante, notação um pouco mais moderna do que a que o
  velho Theon poderia ter usado, por conveniência ao exemplo.}''

``Mas esse problema adiciona complicações extras que não me parecem
tão necessárias.'' - disse Sulla.

``Não tanto assim, mas tudo bem. Não quero tomar muito mais tempo seu,
então vou direto ao ponto. Um modelo mais geral ao problema que deseja
resolver é o seguinte:

\begin{align*}
  maximizar \;& \sum\limits_{i=1}^n q_i \mathscr{S}_i,\\ tal\; que\;&
  \sum{q_i p_i} \leq C,\\ 0 \leq q_i \leq l_i,\; & q_i \; inteiro
\end{align*}

\noindent em que $q_i$ é a quantidade a ser levada do item $i$, $p_i$
o peso do item $i$ (em $\mathscr{S})$, $C$ a capacidade máxima a ser
levada e $l_i$ limite máximo do item $i$ a ser levado (não podemos
levar mais do que $l_i$ do item i, talvez porque não existam mais do
que $l_i$ desses itens).

Esse é um problema de programação inteira. Existem muitas formas de
resolver problemas de programação inteira. Se tiver algum tempo
sobrando, eu poderia desenvolver uma outra forma com você. Será
divertido e intrutivo.''

``Vá em frente.'' - disse Sulla.

``Imagine que você sabe resolver esse problema para $C' < C$ e para
todo subconjunto dos itens (um subconjunto de $\{1, ..., n\}$) e tenha
em mãos uma solução para ele. Se retirarmos dessa solução um item $i$,
de peso $p_i$, deve ser claro que a solução restante é ótima para o
problema cuja capacidade de $C - p_i$ e conjunto de itens $\{1, ...,
n\}-\{i\}$. Isso indica que uma solução ótima goza de uma subestrutura
ótima. Isto é, a remoção de itens dessa solução gera uma solução que é
ótima para outro problema, de fato um subproblema, mais simples. Se
temos a solução para algum desses subproblemas, só precisamos
expandi-lo, considerando itens que não foram considerados. Para cada
um desses itens, sabendo que a nova solução ótima $op(i,c)$ para uma
capacidade $c$, ao considerar a adição de um item $i$, é:

\[
op(i,c) =
\begin{cases}
  op(i-1, c), & se\; p_i > c\\ max\{op(i-1, c), op(i-1, c-p_i) +
  \mathscr{S}_i\} & se\; p_i \leq c
\end{cases}
\]

Fazendo $op(0, C) := 0$ e computando os valores de $op(i, c)$ por essa
recursão\footnote{Conhecida como recursão de Bellman.} temos o valor
ótimo que o meu pai queria. Se quisermos resolver o seu problema,
precisamos fazer apenas uma pequena alteração, como a seguinte:

\[
op(i,c) = max_{q_i}\{op(i-1, c-q_i p_i) + q_i \mathscr{S}_i\; |\; c
\geq q_i p_i\}
\]

Não deve ser difícil de ver como isso nos leva ao resultado
desejado.''

``Oh, mas Theon! Devo ter me enganado ao achar que era um matemático,
pois você me está parecendo mais um ator de teatro! Isso porque
qualquer ator de teatro me daria uma resposta como essa, mas ela é
longe de satisfatória. Mas é pior do que um ator de teatro, porque
quem quer que fosse incumbido de realizar a tarefa pelo método que
você propôs ficaria um tempo intolerável fazendo contas, tempo esse
que é, de fato, morto.  E você seria o responsável pela morte.'' -
disse Sulla.  % QUESTION: o tempo eh morto ou a pessoa morre de tanto tempo que demora?
              % R: o tempo é morto e, com ele, a pessoa, já que o
tempo passado faz parte da nossa morte.

Theon, percebendo a gravidade da situação, disse o seguinte. ``Essa
formulação, como eu disse antes, tinha um caráter mais introdutório
e instrutivo. Se quer uma mais séria, existem várias e em vários
sabores. Uma delas em particular faz uso extenso de técnicas de
programação inteira, que são desenvolvidas há muitas
décadas\footnote{Infelizmente, não há evidências de desenvolvimentos
  de programação inteira na época de Theon, mas a fala é mantida
  aqui por sua conveniência.}. Fazendo uso dessas técnicas, podemos
gerar o seguinte programa:

\begin{listing}
  \inputminted{prolog}{../Exemplos/Cap12/prog1-knapSackEplex.ecl}
  \caption{Conquistador Bárbaro Eplex}\label{lst:knapsackEplex}
\end{listing}

Podemos, analogamente, fazer uso da extensa pesquisa em otimização
por restrições, com algo como o seguinte programa:

\begin{listing}[H]
  \inputminted{prolog}{../Exemplos/Cap12/prog2-knapSackIc.ecl}
  \caption{Conquistador Bárbaro IC}\label{lst:knapsackIC}
\end{listing}

Para saber qual dos métodos é o mais interessante, podemos
simplesmente tomar uma pequena quantidade dos itens iniciais, fazer
um teste, e ver qual resolvemos mais rápido.

``Isso pode ser uma dificuldade, porque, como ainda não alcançamos
todos os lugares de onde os itens serão subtraídos, só temos uma
ideia muito vaga sobre quais itens existem e qual o valor de cada.''
- disse Sulla.

``Um problema facilmente remediado. Podemos fazer testes com valores
aleatórios, como exemplificado aqui.'' e, ao dizer isso, Theon
primeiro explicou a notação que usaria e, depois, mostrou algo
equivalente ao seguinte código a Sulla:

\begin{listing}[H]
  \inputminted{prolog}{../Exemplos/Cap12/test/random.ecl}
  \caption{Problemas Aleatórios}\label{lst:random}
\end{listing}

``Parece que, pelo menos por ora, nosso problema está resolvido.'' -
disse Sulla.

``Sim, o seu problema está.''

\subsection{Os Testes}

Levando em conta a discussão anterior, foram feitos testes para
checar a eficiência dos métodos discutidos. Eles foram realizados em
um computador Inspiron 5447, versão A06, com barramento de 64 bits,
produzido pela Dell. Esse computador conta com um processador Intel
Core i7-4510U, a 2GHz e sistema operaiconal Debian Strech. A versão do
sistema de programação usado é \eclipse 7.0 (de Julho de 2018). 

Os testes foram executados da seguinte forma. O método que faz uso de
programação inteira foi testado 8 vezes, com o teste $i$ contando
com $2^{i+4}$ tipos de itens.

Cada tipo de item tem um peso, um volume que ocupa e uma quantidade
de itens disponíveis. Nos testes que fazem uso de programação
inteira, o peso é escolhido estocasticamente como um valor entre 1 e
50 (unidades de peso), o volume como um valor entre 1 e 100
(unidades de volume) e, a quantidade, como um valor entre 1 e
15. Dada a natureza estocástica dos dados, é necessário fazer o
teste com a mesma quantidade de tipos de itens mais de uma vez, e os
que têm mais tipos de itens devem ser testados mais do que os que
têm menos tipos de itens. A maneira escolhida foi de executar $j$ vezes
o teste com $2^j$ itens.

Infelizmente, realizar os mesmos testes com o método que faz uso de
\technical{branch and bound} foi impraticável, dado o tempo
computacional necessário para tanto ser, no geral, proibitivo para
$i \geq 6$\footnote{``No geral'', porque, os dados sendo aleatórios,
  às vezes o programa termina rapidamente. Esse, entretanto, não é o
  caso geral.}. No lugar disso, então, foram executados três testes
para esse, os dois primeiros como explicados anteriormente (no caso,
fazendo $i + 4 = 4$ e $i + 4 = 5$) e o terceiro caso de teste foi
feito com 50 itens, o qual foi executado 6 vezes (sendo a quantidade de
itens e de testes realizados as únicas diferenças entre esta
modalidade e a explanada acima).

Cada tipo tem um peso, um volume e uma quantidade de itens
disponível. Os pesos são definidos como um número aleatório entre 1 e
50 (unidades de peso), os volumes são definidos como um número
aleatório entre 1 e 100 (unidades de volume) e as quantidades de itens
como um número aleatório entre 1 e 15.

Os resultados são como segue na tabela a seguir (medições de tempo com
dois valores significativos). Os valores na coluna ``Programação
Inteira'' são referentes aos respectivos tempos para a resolução dos
problemas por meio de restrições provindas da biblioteca
\technical{eplex} de programação inteira, enquanto que os valores na
coluna ``Branch and Bound'' são referentes aos respectivos tempos para
a resolução dos problemas por meio de restrições provindas da
biblioteca \technical{branch\_and\_bound}. Como os testes para $2^7$
itens ou mais se tornaram inviáveis se fazendo uso da biblioteca
\technical{branch\_and\_bound}, os respectivos espaços na tabela foram
substituídos por ``NA''.

\begin{center}
  \bgroup \def\arraystretch{1.25}
  \begin{tabular}{| l | c | c |}
    \hline Tempos & Programação inteira & Branch and Bound \\ \hline
    \hline Menor tempo para $2^4$ & 0,00s & 0,00s \\ Maior tempo
    para $2^4$ & 0,01s & 0,09s \\ Tempo médio para $2^4$ & 0,00s &
    0,02s \\ \hline

    Menor tempo para $2^5$ & 0,00s & 0,00s \\ Maior tempo para $2^5$
    & 0,01s & 0,07s \\ Tempo médio para $2^5$ & 0,00s & 0,01s
    \\ \hline

    Menor tempo para $2^6/50$ & 0,00s & 0,08s \\ Maior tempo para
    $2^6/50$ & 0,01s & 4,70 \\ Tempo médio para $2^6/50$ & 0,01s &
    1.81 \\ \hline

    Menor tempo para $2^7$ & 0,01s & NA \\ Maior tempo para $2^7$ &
    0,01s & NA \\ Tempo médio para $2^7$ & 0,01s & NA \\ \hline

    Menor tempo para $2^8$ & 0,02s & NA \\ Maior tempo para $2^8$ &
    0,02s & NA \\ Tempo médio para $2^8$ & 0,02s & NA \\ \hline

    Menor tempo para $2^9$ & 0,02s & NA \\ Maior tempo para $2^9$ &
    0,04s & NA \\ Tempo médio para $2^9$ & 0,03s & NA \\ \hline

    Menor tempo para $2^{10}$ & 0,05s & NA \\ Maior tempo para
    $2^{10}$ & 0,07s & NA \\ Tempo médio para $2^{10}$ & 0,06s & NA
    \\ \hline

    Menor tempo para $2^{11}$ & 0,6s & NA \\ Maior tempo para
    $2^{11}$ & 0,8s & NA \\ Tempo médio para $2^{11}$ & 0,7s & NA
    \\ \hline

    Menor tempo para $2^{12}$ & 0,14s & NA \\ Maior tempo para
    $2^{12}$ & 0,16s & NA \\ Tempo médio para $2^{12}$ & 0,14s & NA
    \\ \hline
  \end{tabular}
  \egroup
\end{center}


\begin{thebibliography}{2}

\bibitem{kellerer} Kellerer, Hans e Pferschy, Ulrich e Pisinger,
  David, ``Knapsack Problems'', Springer

\bibitem{plutarch} Plutarch, ``Sulla'', traduzido por John Dryden

\bibitem{eclipse} \eclipse,
  \url{http://www.eclipseclp.org/examples/}

\end{thebibliography}

\end{document}
