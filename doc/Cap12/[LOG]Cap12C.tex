\documentclass{article}

\usepackage[utf8]{inputenc}
\usepackage{placeins}
\usepackage{graphicx}
\usepackage[ruled,vlined,linesnumbered]{algorithm2e}
\usepackage{listings}
\usepackage{amsthm}
\usepackage{amsmath}
\usepackage{caption}
\usepackage{epigraph}

\captionsetup[figure]{font=scriptsize}

\newtheorem{definition}{Definição}[section]
\theoremstyle{remark}
\newtheorem*{remark}{Observações}
\theoremstyle{theorem}
\newtheorem{theorem}{Teorema}[section]

\setlength{\parskip}{.5em}

\renewcommand\refname{Leituras adicionais}
\renewcommand\figurename{Figura}
\renewcommand{\lstlistingname}{Código}

\lstdefinestyle{prosty}{
  language=Prolog,
  basicstyle=\small
}



\usepackage{pgfgantt}

\begin{document}

\section{Montando Bicicletas}

Este exemplo foi inspirado de um exemplo de \cite{antonii} (que, por sua vez, foi inspirado em um
ensaio de Ronald Graham), que escrevou sobre uma linha de montagens de bicicletas fictícia chamada
ACME. Segue uma citação de Graham, sobre a linha de montagens:

``As coisas não tem ido bem para a seção de montagem da Companhia de Bicicletas ACME. Pelos últimos
seis meses, a seção consistentemente falhou em cumprir sua quota e cabeças estavam começando a
rolar. Um novo chefe foi trazido para a seção de montagem para remediar esse triste estado. Ele
percebe que essa é sua grande chance de capturar a atenção da alta gerência, de modo que, em seu
primeiro dia de serviço, ele arregaça as mangas e começa a descobrir sobre tudo o que acontece na seção.
A primeira coisa que ele aprende é que o trabalho de montar uma bicicleta é geralmente quebrado em
um número de tarefas mais simples:

\begin{enumerate}
  \item Preparação da armação, que inclui instalação do garfo frontal e de paralamas;
  \item Montagem e alinhamento da roda da frente;
  \item Montagem e alinhamento da roda de trás;
  \item Anexação do descarrilhador à armação;
  \item Instalação do conjunto de engrenagens;
  \item Anexação da manivela e da correnta da roda à armação;
  \item Montagem do pedal direito e do clip do dedo do pé;
  \item Anexações finais, que includem montagem e ajuste do guidão, assento, freis, etc.
\end{enumerate}

Ele também aprende que seus predecessores que partiram recentemente haviam coletado montes de dados
sobre quanto tempo (na média, em minutos) cada uma dessas tarefas um montador trinado leva para
montar, o que ele tinha convenientemente sumarizado na sequinte tabela:

\begin{tabular}{c c c c c c c c c c c}
Tarefas: & A & B & C & D & E & F & G & H & I & J \\
\hline
Tempo:   & 7 & 7 & 7 & 2 & 2 & 2 & 2 & 8 & 8 & 18  \\
\end{tabular}

Devido a restrições de espaço e de equipamento na loja, os 20 motadores na seção são geralmente
separados em pares de 10 times com 2 montadores cada, sendo que cada time montaria uma bicicleta
por vez. Ele fez um cálculo rápido: uma bicicleta requer 63 minutos de tempo de montagem, então m
time de dois \enphasis{deveria} requerir 31,5 minutos. Isso significa que em um dia de trabalho de
oito horas, cada time montaria 15,23 bicicletas e, com todos os 10 times fazendo isso, a quota de
152 bicicletas por dia pode ser cumprida. O novo chefe de montagem já pode sentir o gosto de sua
promoção no próximo ano.

Entretanto, seu entusiasmo diminui consideravelmente quando ele percebe que bicicletas não podem ser
montadas aleatoriamente. Certas tarefas devem ser feitas antes de outras. Por exemplo, é
extremamente inadequado montar o garfo frontal à armação de uma bicicleta se o guidão já tiver sido
anexada ao garfo. Similarmente, a manivela deve ser montada na armação antes dos pedais serem
  anexados. Depois de discussões demoradas com vários dos montadores experientes, o novo chefe prepara
a seguinte carta mostrando que tarefas devem ser feitas antes das outras durante a montagem:

\begin{tabular}{l c d}
A, B, C, D, E & deve ser feito antes de & J   \\
D, E, A       & deve ser feito antes de & C   \\
D             & deve ser feito antes de & E,F \\
E, F, G       & deve ser feito antes de & H,I \\
F             & deve ser feito antes de & G   \\
A             & deve ser feito antes de & B   \\
\end{tabular}

Além dessas restrições mecânicas, existem também duas regras que a gerência requer que sejam
observadas:

\begin{itemize}
  \item \textbf{Regra 1:} Nenhum montador ou montadora pode ficar parado(a) se existe alguma tarefa
    que ele(a) pode fazer;
  \item \textbf{Regra 2:} Uma vez que um(a) montador(a) começa uma tarefa, ele(a) deve continuar
    trabalhando na tarefa até que ela esteja completa
\end{itemize}

A ordem de montagem de bicicletas na ACME sempre foi a seguinte:

\begin{tabular}{c c c c c c c c c c c}
Tarefas: & A & B & C & D & E & F & G & H & I & J \\
\hline
Tempo de início:   & 1 & 8 & 9 & 1 & 7 & 3 & 5 & 15 & 23 & 16  \\
\end{tabular}

%Com o seguinte gráfico Gantt:

%\begin{ganttchart}[y unit title=0.4cm,
%y unit chart=0.5cm,
%vgrid,hgrid,
%title height=1,
%bar/.style={draw,fill=cyan},
%bar incomplete/.append style={fill=yellow!50},
%bar height=0.7]{1}{23}
 %\gantttitle{}{23}
 %\gantttitle{Junho}{12} \\
 %\gantttitlelist{20,...,31}{1}
 %\gantttitlelist{1,...,23}{1} \\
 %\ganttbar{Montador 1}{3}{7} \\
 %\ganttbar{Montador 2}{7}{12} \\
 %\ganttbar[progress=70]{Fase 3}{13}{18} \\
 %\ganttbar[progress=40]{Conclus\~ao}{20}{24} \\
  %rela\c c\~oes
 %\ganttlink{elem0}{elem1}
 %\ganttlink{elem1}{elem2}
 %\ganttlink{elem2}{elem3}
%\end{ganttchart}

O cronograma mostra a atividade de cada montador to time começando no tempo 1 e progredindo até o
tempo da montagem completa, chamado \enphasis{tempo de montagem total}, cerca de 33 minutos depois.
Apesar de esse cronograma obedecer todas as restrições requeridas, ele permite cada time completar
14,5 bicicletas por dia (na média). Assim, a saída total da seção é 145 bicicletas por ida, o que é
menos do que 152.''

%%
%
Depois de várias tentativas infrutíferas, ``o chefe decide, com pressa, fornecer a todos os
montadores ferramentas elétricas alugadas. Isso diminui o tempo de cada um dos serviços por
exatamente um minuto, então o tempo total requerido para todos os serviços é apenas 53 minutos.''

%%

Vendo que essa ação não foi o suficiente, o chefe de seção decide contratar 10 montadores a mais,
aumentando o custo de trabalho em 50\%. O especialista em CLP avisa ao chefe de seção que 10
montadores a mais não aumentariam a produção, devido às restrições na ordem das tarefas.




  \begin{thebibliography}{2}

    \bibitem{antonii}
    Niederliński, Antoni,
    ``A gentle guide to constraint logical programming via Eclipse'',
    3rd edition, Jacek Skalmierski Computer Studio, Gliwice, 2014



  \end{thebibliography}

\end{document}
