\documentclass{article}

\usepackage[utf8]{inputenc}
\usepackage{placeins}
\usepackage{graphicx}
\usepackage[ruled,vlined,linesnumbered]{algorithm2e}
\usepackage{listings}
\usepackage{amsthm}
\usepackage{amsmath}
\usepackage{caption}
\usepackage{epigraph}

\captionsetup[figure]{font=scriptsize}

\newtheorem{definition}{Definição}[section]
\theoremstyle{remark}
\newtheorem*{remark}{Observações}
\theoremstyle{theorem}
\newtheorem{theorem}{Teorema}[section]

\setlength{\parskip}{.5em}

\renewcommand\refname{Leituras adicionais}
\renewcommand\figurename{Figura}
\renewcommand{\lstlistingname}{Código}

\newcommand{\eclipse}{$ECL^iPS^e$}
\newcommand{\definicao}[1]{\textbf{#1}\marginpar{\small \textbf{#1}}}

\lstdefinestyle{prosty}{
  language=Prolog,
  basicstyle=\small
}




\begin{document}

\section{Desenvolvimento de Aeronaves por Programação Geométrica}

Neste capítulo vamos ver como podemos usar um pouco do que vimos até
agora para resolvermos um problema um pouco mais concreto do que os
apresentados como exemplos. O problema a ser analisado é o de projeto
(\foreign{design}) de aeronaves.

Pelas últimas décadas, métodos de otimização têm tomado cada vez maior
importância no desenvolvimento de máquinas complicadas, mas ainda com
muitas limitações. O problema é que, quando um projeto é tão grande e
complicado quanto o do desenvolvimento de aeronaves, que contam com
relações altamente não lineares entre muitas %TODO Quantas?
variáveis, a aplicação de um método homogêneo, ou mesmo de vários
métodos heterogêneos, fica difícil. Não é pouco usual a aplicação de
um método de otimização a uma parte específica do projeto (o da asa,
por exemplo) e deixá-lo rodando por dias para obter, no final, apenas
uma resposta aproximada, que pode ou não ser viável quando se leva em
consideração o resto do projeto.

Para estudar como melhor lidar com esse tipo de situação, surgiu o
campo de estudos de \enphasis{Otimização Multidisciplinar de Projetos}
(ou \foreign{Multidisciplinary Design Optimization}, \definicao{MDO}
na sigla em inglês, que será usada daqui para frente). O objetivo dos
métodos MDO são coordenar de maneira eficiente diferentes métodos de
otimização para um projeto único, na esperança de assim obter projetos
melhores (``melhores'' nos termos definidos pelo ou pela projetista)
do que os obtidos otimizando o projeto por partes isoladas.  No
entanto, mesmo depois de numerosos avanços em métodos MDO, aplicá-los
a um projeto inteiro de uma aeronave ainda é impraticável na maior
parte dos casos.

Em última instância, a utilização de métodos de otimização para esse
tipo de problema (o de projetos como o de uma aeronave, como no
exemplo) é feita a fim de se entender e analisar melhor o problema em
mãos. Isso porque, no início, o problema a ser resolvido pelo projeto
não é, no geral, bem posto (muitas vezes, os objetivos do projeto não
são sequer definidos\cite{hoburg}), e é de interesse entender o
formato de sua fronteira de Pareto\footnote{Uma alocação de recursos
  (codificada aqui como a atribuição de valores a variáveis) é dita
  Pareto eficiente se não é possível alterar tal alocação tal que um
  critério de otimalidade seja melhorado sem que outro seja
  piorado. Fronteira de Pareto é o conjunto de alocações Pareto
  eficientes.}, a fim de se entender a relevância de cada
característica (ou, poderíamos dizer, ``de cada variável'').  Isso,
aliado ao fato de que geralmente um projetista gostaria de otimizar
vários parâmetros, e não só um, aumenta a necessidade de se obter não
só uma ``resposta'' otimizada, mas também um maneira de entender a
vizinhança dessa resposta (isto é, realizar uma análise de
sensibilidade).

Vale também notar que há um impasse entre a realização de uma análise
de alta ou de baixa fidelidade. Para análises de alta fidelidade,
existem duas opções:

\begin{itemize}
\item Fazer uma análise de uma parte específica do projeto (como o da
  asa, ou do motor), o que possibilitaria a quem cuidar dele fazer
  decisões mais informadas sobre essa parte específica;
\item Fazer uma análise mais geral, envolvendo um subconjunto maior de
  ``partes específicas'', incorrendo em instâncias mais arbitrárias de
  problemas mais gerais, e leva, em geral, dias ou semanas para a
  obtenção de uma solução.
\end{itemize}

Esses tipos de problemas, assim como a possibilidade de
\foreign{overfitting} \footnote{Uma solução ``está em''
  \foreign{overfitting} quando levar a excelentes resultados para uma
  comparativamente pequena quantidade de casos de teste, mas a
  soluções ruins quando em situações mais gerais.} levam, em vários
casos, a uma escolha por uma análise de baixa fidelidade.


Antes de seguirmos um pouco mais a fundo, convém obtermos uma ideia
geral do desenvolvimento típico de uma aeronave. Esse desenvolvimento
é composto por três etapas\cite{hoburg} (ou, pelo menos, assim os
engenheiros o têm tratado por algumas décadas):
\begin{itemize}
\item Projetagem conceitual: etapa em que os objetivos do projeto são
  definidos, assim como requisitos de performance. Dependendo do
  projeto, esta etapa pode levar vários anos e pode envolver uma
  análise matemática razoavelmente detalhada;
\item Projetagem preliminar: envolve uma análise mais aprofundada da
  configuração do projeto, identificação e resolução de problemas de
  interferência aerodinâmica e de instabilidade é realizada e, em um
  certo ponto, a configuração básica da aeronave é ``congelada'', de
  modo a permitir que times que trabalham com subsistemas possam
  trabalhar independentemente, sem afetar demais os outros times;
\item Projetagem detalhada: envolve uma projetagem detalhada do
  projeto, desenhos em CAD, determinação de passagens hidráulicas,
  elétricas e de combustível, assim como fabricação de peças de
  produção\footnote{Peças que são pensadas para uso em um projeto
    específico.}.
\end{itemize}

Cada uma dessas etapas pode envolver grandes problemas de otimização
com restrições, envolvendo as dificuldades notadas acima.

\subsection{Programação Geométrica}

Nos últimos anos, técnicas de programação convexa têm se tornado cada
vez mais eficientes (aproximando mesmo técnicas de programação
linear).  Apesar disso, por incrível que possa parecer, tais técnicas
ainda são notavelmente ausentes em propostas MDO\cite{hoburg} e a
abordagem mais comum a esses problemas é por meio de métodos gerais de
programação não-linear. Assim, problemas muito gerais podem ser
abordados, mas dificilmente os critérios de confiança e e eficiência
são preenchidos.

No lugar disso, então, voltamos nossa atenção a métodos de otimização
convexa e, em particular, a um método que se mostrou particularmente
adequado para lidar com problemas de projeto de aeronave, o de
programação geométrica\footnote{Diferentemente do que se pode supor, o
  nome ``programação geométrica'' não é devido às suas propriedades
  geométricas no geral, mas sim à desigualdade geométrico-aritmética,
  que foi muito usada em sua análise.}. Quando esse método é
aplicável, entre as vantagens que ele oferece estão a obtenção de
soluções globalmente ótimas (ou um certificado de infactibilidade,
quando não existem tais soluções), com tempos de solução
comparativamente curtos, que escalam para problemas maiores (isto é,
quando o problema aumenta ``um pouco'', o tempo de solução aumenta só
``um pouco''). Ele tem a desvantagem de não ser um método geral, mas
como só estamos interessados (no momento) em resolver problemas de
projetagem de aeronaves, essa desvantagem não nos entristece muito.

Um problema de programação geométrica em forma padrão pode ser posto
da seguinte forma:

\begin{align*}
  minimize \;& \sum\limits_{i=1}^n f_0(x)\\ tal\; que\;& f_i(x) \leq\
  1,\ i = 1, ..., n,\\ & g_j(x) = 1,\ j = 1, ..., k,\\ & x \in R^n_{+},
\end{align*}

\noindent em que $f_i$ são posinômios ou monômios, os $g_j$ são
monômios e $x \in R^n$ (sendo $n$ a quantidade de monômios e $k$ a de
posinômios). Aqui, a definição de monômios (e de posinômios) é um
pouco diferente do usual em álgebra:

\begin{itemize}
\item Um \definicao{monômio} $g(x)$ é algo da forma $g(x) =
  kx^{\alpha}$, $\alpha \in R^n, k \in R^n_{+}$\footnote{Aqui, usamos
    a convenção de que, se $x \in R^n$ e $\alpha \in R^n$, $x^{\alpha}
    := \prod x_i^{\alpha_i}$ e, analogamente, se $x \in R$ e $\alpha \in R^n$, $x^\alpha :=
    (x^{\alpha_i}) \in R^n$.};
\item Um \definicao{posinômio} é algo da forma $f(x) = \sum g_i(x)$,
  em que cada $g_i$ é um monômio (em particular, monômios são
  posinômios).
\end{itemize}

Note que esse problema não só é altamente não-linear, mas também é
não-convexo. Nós podemos, no entanto, torná-lo convexo, com uma
mudança de coordenadas apropriada. Note que, se $f$ é um posinômio em
$x$, $log(f(e^x))$ é convexo e, se $g$ é um monômio em $x$,
$log(g(e^x))$ é, não só convexo, mas também afim. Realizando essa
mudança de variáveis, nosso problema de otimização pode ser expresso
como

  % minimize \;& log \sum e^{\alpha_{0k}'x + b_{0k}}\\ tal\; que\;& log
  % \sum e^{\alpha_{ik}' + b_{ik}} \leq 0\\ & x \in R^n_{+}.

\begin{align*}
  minimize\; & log\ f_0 (e^x)\\
  tal\; que\; & log\ f_i (e^x) \leq 0,\; i = 1, ..., n\\
  e\; & log\ g_j (e^x) = 0,\; para\; j = 1, ..., k.
\end{align*}

Nessa nova formulação, estamos tomando $\alpha$ como a matriz de
vetores que geram as potências dos monômios e posinômios. Note que,
fazendo essa mudança de variáveis ($v = log(x)$), monômios se tornam
funções afim e, posinômios, funções convexas.

É importante notar que monômios e posinômios são um conjunto fechado
sob as operações de divisão por monômios, então desigualdades do tipo
$f_1(x) = f_2(x)$, em que $f_2$ é um monômio e $f_1$ é um posinômio
poderiam ser lidadas fazendo $\frac{f_1(x)}{f_2(x)} = 1$. Ademais,
igualdades como $g(x) = 1$, se $g$ é um monômio, podem ser lidadas
fazendo $g(x) \leq 1$ e $\frac{1}{g(x)} \leq 1$\footnote{Em
  particular, isso mostra que um programa geométrico qualquer pode ser
  expresso com desigualdades do tipo $f_i(x) \leq 1$, o que é a
  entrada que alguns \foreign{solvers} requerem por padrão.}. A
restrição de que $k$ precisa ser positivo acaba sendo frequentemente
satisfeita sem esforços mas, quando esse não é o caso, é preciso
usar outro maquinário que não o da programação geométrica. Em
particular, programação signomial\footnote{Um posinômio é um signômio
  que é restrito a ter coeficientes positivos. Programação signomial é
  a cujas restrições são signômios.}, que faz uso de uma estrutura
parecida com a da geométrica, resolve esse caso (no geral, de forma
mais eficiente do que um método de programação não linear
genérico).

\subsection{Um modelo simples de aeronave}


A seguir é apresentado um modelo de aeronave simplificado, em grande
parte baseado no modelo da tese de Hoburg\cite{hoburg}. A modelagem do
problema será feita com base no pacote GPkit\cite{gpkit}, um
pacote de software para modelagem algébrica de problemas de
programação geométrica para a linguagem Python, que nos permite lidar
com esse tipo de problema de maneira semelhante (apesar de não em
sintaxe) ao que estávamos acostumados com o \eclipse.

\subsubsection{Modelo de peso e sustentação}

Assumimos que a aeronave esteja em vôo de cruzeiro\footnote{Quando o
  avião atinge sua altitude de vôo.}, de modo que a sustentação igual
ao peso. O peso da aeronave é a soma do peso de carga útil $P_0$, o da
asa $P_a$ e do combustível $P_c$:
\[
P \geq P_0 + P_a + P_c,
\]
Usamos o modelo abaixo para vôo estável\footnote{O ``l'' abaixo vem de
  ``\foreign{lift}'', ``sustentação'' em inglês.}:
\[
  P_0 + P_a + \frac{P_c}{2} \leq \frac{1}{2} \rho S C_l V^2,
\]

\noindent em que a sustentação da aeronave é igual ao peso dela com
metade de combustível, $C_l$ é o coeficiente de sustentação e $V$ a
velocidade do avião e $\rho$ é uma constante física.

Aqui usamos a equação para arrasto:
\[
  Arrasto = D := \frac{1}{2} \rho S C_d V^2,
\]

\noindent em que $\frac{1}{2}\rho V^2$ representa a pressão dinâmica,
$S$ corresponde à área da asa e $C_d$ ao coeficiente de arrasto
total\footnote{O ``d'' vem de ``\foreign{drag}'', arrasto em inglês.},
dado por:
\[
  C_{d_{fuse}} = \frac{CDA_0}{S} + kC_f \frac{S_{wet}}{S} +
  \frac{C_{l}^2}{\pi A e},
\]

\noindent em que $CDA_0$ corresponde à área de arrasto da
fuselagem\footnote{Definida como a área normal ao fluxo de ar que
  geraria o mesmo arrasto que o avião.}, $k$ é um fator de forma que
leva em conta arrasto de pressão, $\frac{S_{wet}}{S}$ é a razão de
área molhada\footnote{A área em contato com o fluxo de ar.} e $e$ é o
fator de eficiência de Oswald. $CDA_0$ é linearmente proporcional ao
volume de combustível na fuselagem\footnote{GPkit faz checagem de
  unidades, então precisamos corrigi-la explicitamente, do que o
  ``[metros]'' constitui um lembrete.}:
\[
  V_{f_{fase}} \leq CDA_0 \times 10 [metros].
\]




Assumindo um fluxo de Blasius turbulento sobre uma placa plana, o
coeficiente de fricção $C_f$ pode ser aproximado por:
\[
  C_f = \frac{0,074}{Re^{0,2}},
\]

\noindent em que $Re = \frac{\rho V}{\mu}\sqrt{\frac{S}{A}}$ é o
número de Reynolds na corda média $\sqrt{\frac{S}{A}}$\footnote{$A$
  representa a razão de aspecto, ie, a razão entre a distância entre
  as pontas das asas e a corda média.}.

Também gostaríamos que uma aeronave cheia de combustível seja capaz de
voar a uma velocidade mínima:
\[
  P \leq \frac{1}{2}\rho V_{min}^{2} S C_{l_{max}},
\]

\noindent em que $C_{l_{max}}$ é o coeficiente de sustentação máximo.

Uma medida de performance útil, o tempo de vôo, é dada pela distância
percorrida sobre velocidade:
\[
  T_{v\hat{o}o} \geq \frac{Dist\hat{a}ncia}{V}.
\]

A razão de força de sustentação/arrasto é dada por:
\[
  \frac{L}{D} = \frac{C_l}{C_d},
\]

\noindent em que $C_l$ é o coeficiente de sustentação.

Essas restrições se traduzem como

\inputminted{python}{../Exemplos/Cap15/prog1_wlm.py}

% \subsubsection{Modelo de Impulso e arrasto}

Assumimos que o Consumo Específico de Combustível por Unidade de
Empuxo (CEUE) é constante e que ele provê tanto impulso quanto
necessário. Como a força de impulso $F$ é maior ou igual que a força
de arrasto $D$, obtemos
\[
  P_f \geq CEUE \times T_{v\hat{o}o} \times D.
\]

E podemos completar nosso modelo da seguinte forma:

\inputminted{python}{../Exemplos/Cap15/prog2_tdm.py}

\subsubsection{Volume do combustível}

Definindo o volume requerido em função da densidade do combustível
$\rho f$, temos
\[
  V_c = \frac{P_c}{\rho_f g}.
\]

Pelo modelo aerodinâmico, temos que
\[
  V_{c_{asa}}^2 \leq 0,0009\frac{S\tau^2}{A \times Dist\hat{a}ncia}.
\]

O volume de combustível disponível $V_{c_{disp}}$ é limitado
superiormente pelos volumes disponíveis na asa e na fuselagem:
\[
  V_{c_{disp}} \leq V_{c_{asa}}+ V_{c_{fuse}}.
\]

\noindent Note que, devido à restrição acima, o programa resultante
será signomial.

Restringimos o volume total de combustível como menor do que o volume
disponível:
\[
  V_{c_{disp}} \geq V_c.
\]

\inputminted{python}{../Exemplos/Cap15/prog3_fvm.py}

\subsubsection{Acumulação de peso nas asas}

O peso na superfície da asa é uma função da área da asa

\[
  P_{p_{surp}} \geq P_{p_{coef2}}S.
\]

O peso estrutural na asa é uma expressão posinomial que leva em conta
o alívio de cisalhamento devido à presença de combustível nas asas e o
momento flexor:
\[
  P_{p_{strc}}^2 \geq \frac{P_{p_{coeff1}}^2}{\tau^2}(N_{ult}^2
  AR^3((P_0 + \rho_f g V_{c_{fuse}})PS)).
\]

O peso total da asa é limitado:
\[
  P_p \geq P_{p_{surp}} + P_{p_{strc}}.
\]

\inputminted{python}{../Exemplos/Cap15/prog4_wwb.py}

\subsubsection{Possíveis objetivos}

Dadas essas restrições, algumas potenciais funções objetivo (a serem
minimizadas) com as quais poderíamos querer trabalhar são:

\begin{itemize}
\item $P_c$, o peso do combustível (o objetivo padrão);
\item $P$, o peso total da aeronave;
\item $\frac{P_c}{T_{v\hat{o}o}}$, o peso do combustível dividido pelo
  tempo de vôo (podemos pensar nisso como uma medida de eficiência no
  uso do combustível);
\item $P_c + c \times T_{v\hat{o}o}$, uma combinação linear antre peso
  do combustível e tempo de vôo.
\end{itemize}

\subsection{Modelo de asa}
Como um exemplo simples de aplicação de programação geométrica, é
apresentado a seguir um projeto de asa simples, tirado de
\cite{warren}. As fórmulas usadas aqui são semelhantes às vistas na
seção anterior.  Queremos dimensionar uma asa de área total $S$,
envergadura $b$ e alongamento $A = \frac{b^2}{S}$. Gostaríamos de
escolher esses parâmetros de forma a minimizar o arrasto total, $D =
\frac{1}{2}\rho V^2 C_D S$.  O coeficiente de arrasto é modelado como
a soma de arrasto parasita da fuselagem, arrasto parasita da asa e
arrasto induzido,

\[
  C_D = \frac{CDA_0}{S} + kC_f \frac{S_{wet}}{S} + \frac{C_{L}^2}{\pi A
    e},
\]

\noindent em que $CDA_0$ representa a área de arrasto da fuselagem,
$k$ é o fator de forma, que leva em conta o arrasto de pressão,
$S_wet/S$ é a razão de área molhada e $e$ é o fator de eficiência de
Oswald.

O coeficiente de fricção $C_f$ é aproximado por:
\[
  C_f = \frac{0,074}{Re^{0,2}},
\]

\noindent em que $Re = \frac{\rho V}{\mu} \sqrt{\frac{S}{A}}$. O peso
total da aeronave é modelado como a soma de um peso fixo $W_0$ e o
peso da asa $W_w$, $W = W_0 + W_w$. O peso da asa é modelado como:
\[
  W_w = 45,42S + 8,71\times10^{-5} \frac{N_{lift}b^3\sqrt{W_0W}}{S\tau},
\]

\noindent em que $N_{lift}$ é o fator de carga para dimensionamento estrutural e
$\tau$ é a razão de espessura-para-corda do aerofólio.

Tomamos também que
\[
W = \frac{1}{2} \rho V^2 C_L S,
\]

Além disso, a aeronave deve ser capaz de voar a uma velocidade mínima
$V_{min}$
\[
  \frac{2W}{\rho V_{min}^2 S} \leq C_{L,max}.
\]


Note que as restrições acima são expressões monomiais e posinomiais,
de modo que podemos lidar com este modelo como um problema de programação
geométrica.

Resolvendo o problema com o código no final da seção, obtemos um custo ótimo de 303,1 e os seguintes
valores para as variáveis livres:

\begin{tabular}{|r c c r|}
  \hline
  A & : & 8,46 & Alongamento \\ $C_D$ & : & 0,02059 & Coeficiente de
  arrasto da asa \\ $C_L$ & : & 0,4988 & Coeficiente de sustentação da
  asa \\ $C_f$ & : & 0,003599 & Coeficiente de fricção \\ D & : &
  303,1 (N) & Força de arrasto total \\ Re & : & 3,675e+06 & Número de
  Reynold \\ S & : & 16,44 ($m^2$) & Área total da asa \\ V & : &
  38,15 (m/s) & Velocidade de cruzeiro \\ W & : & 7341 (N) & Peso
                                                             total da aeronave \\ $W_w$ & : & 2401 (N) & Peso da asa \\
  \hline
\end{tabular}

É importante para um projetista ter também ideia da sensibilidade de
algumas variáveis (isto é, mudando um pouco o valor de uma variável, é
de interesse saber quanto o valor da solução muda, se muito ou
pouco). Os valores a seguir buscam refletir isso (os valores são
referentes à derivada logarítmica $\frac{d(log(y)}{d(log(x)}$):

\begin{tabular}{|l c  c  l|}
  \hline
  $W_0$ & : & +1 & Peso da aeronave sem asa \\ e & : & -0,48 & Fator
  de eficiência de Oswald \\ ($\frac{S}{S_{wet}}$) & : & +0,43 & Razão
  de área molhada \\ $k$ & : & +0,43 & Fator de forma \\ $V_{min}$ & : &
                                                                         -0,37 & Velocidade de decolagem \\
  \hline
\end{tabular}


Frequentemente o conhecimento da fronteira de Pareto é de grande
importância. Para obtermos algum conhecimento sobre ela, podemos
atribuir valores a algumas variáveis e checar como as soluções se
comportam para tais valores. Fazemos isso com as variáveis $V$ e
$V_{min}$:

\begin{tabular}{|c c  c  c  c  c  l  l|}
  \hline
  V & : & 45 & 45 & 55 & 55 & m/s & Velocidade de cruzeiro \\ \hline
  $V_{min}$ & : & 20 & 25 & 20 & 25 & m/s & Velocidade de decolagem \\  \hline
  Custos & : & 338 & 294 & 396 & 326 & &\\
  \hline
\end{tabular}

Os valores das variáveis livres são os seguintes:

\begin{tabular}{|c c c c c c c l|}
  \hline
  A & : & 6,2 & 8,84 & 4,77 & 7,16 & & Alongamento \\ $C_D$ & : &
  0,0146 & 0,0196 & 0,0123 & 0,0157 & & Coeficiente de arrasto da asa
  \\ $C_L$ & : & 0,296 & 0,463 & 0,198 & 0,31 & & Lift coefficent of
  wing \\ $C_f$ & : & 0,00333 & 0,00361 & 0,00314 & 0,00342 & &
  Coeficiente de fricção \\ D & : & 338 & 294 & 396 & 326 & N & Força
  de arrasto total \\ Re & : & 5,38e+06 & 3,63e+06 & 7,24e+06 &
  4,75e+06 & & Número de Reynolds \\ S & : & 18,6 & 12,1 & 17,3 & 11,2
  & $m^2$ & Área total da asa \\ W & : & 6,85e+03 & 6,97e+03 & 6,4e+03
  & 6,44e+03 & N & Peso total da aeronave \\ $W_w$ & : & 1,91e+03 &
                                                                    2,03e+03 & 1,46e+03 & 1,5e+03 & N & Peso da asa \\
  \hline
\end{tabular}

As variáveis mais sensíveis, como apresentado pelo modelo (veja
\cite{gpkit} para mais detalhes), são:

\begin{tabular}{|c c c c c c l|}
  \hline
  $ W_0$ &: & +0,92 & +0,95 & +0,85 & +0,85 & Peso da aeronave sem a asa
  \\ $ V_{min}$ &: & -0,82 & -0,41 & -1 & -0,71 & Velocidade de
  decolagem \\ $ V$ &: & +0,59 & +0,25 & +0,97 & +0,75 & Velocidade de
  cruzeiro \\ $(\frac{S}{S_{wet}})$ &: & +0,56 & +0,45 & +0,63 & +0,54
  & Razão de área molhada \\ $ k$ &: & +0,56 & +0,45 & +0,63 & +0,54 &
                                                                       Fator de forma \\
  \hline
\end{tabular}

\pagebreak
\subsection*{Programa}
O programa usado para obter os dados acima é o seguinte:

\inputminted{python}{../Exemplos/Cap15/wing.py}



\begin{thebibliography}{1}

\bibitem{hoburg} Hoburg, W. Warren, ``Aircraft Design
  Optimization as a Geometric Program'' 2013.

\bibitem{gpkit} E.~Burnell and W.~Hoburg, ``Gpkit software for
  geometric programming.''
  \url{https://github.com/convexengineering/gpkit}, 2018.
  \newblock Version 0.7.0.

\bibitem{simpleac} B.~Ozturk, ``SimPleAC''
  \url{https://github.com/convexengineering/gplibrary/blob/master/gpkitmodels/SP/SimPleAC/simpleac.pdf},
  2018.

\bibitem{warren} Hoburg, W. Warren, ``Geometric Program for
  Aircraft Design Optimization'' 2014.

  
\end{thebibliography}

\end{document}
