\documentclass{article}

\usepackage[utf8]{inputenc}
\usepackage{placeins}
\usepackage{graphicx}
\usepackage[ruled,vlined,linesnumbered]{algorithm2e}
\usepackage{listings}
\usepackage{amsthm}
\usepackage{amsmath}
\usepackage{caption}
\usepackage{epigraph}

\captionsetup[figure]{font=scriptsize}

\newtheorem{definition}{Definição}[section]
\theoremstyle{remark}
\newtheorem*{remark}{Observações}
\theoremstyle{theorem}
\newtheorem{theorem}{Teorema}[section]

\setlength{\parskip}{.5em}

\renewcommand\refname{Leituras adicionais}
\renewcommand\figurename{Figura}
\renewcommand{\lstlistingname}{Código}

\newcommand{\eclipse}{$ECL^iPS^e$}
\newcommand{\definicao}[1]{\textbf{#1}\marginpar{\small \textbf{#1}}}

\lstdefinestyle{prosty}{
  language=Prolog,
  basicstyle=\small
}




\begin{document}

\section{Desenvolvimento de Aeronaves por Programação Geométrica}

\subsection{Introdução}

Neste capítulo vamos ver como podemos usar um pouco do que vimos até agora para resolvermos um problema um pouco mais concreto
do que os apresentados como exemplos. O problema a ser analisado é o de projeto (\foreign{design}) de aeronaves.

Pelas últimas décadas, métodos de otimização têm tomado cada vez maior importância no desenvolvimento de máquinas
complicadas, mas ainda com muitas limitações. O problema é que, quando um projeto é tão grande e complicado quanto o
do desenvolvimento de aeronaves, que contam com relações altamente não lineares entre muitas % QUESTION Quantas?
variáveis, a aplicação de um método homogêneo, ou mesmo de vários métodos heterogêneos, fica difícil. Não é pouco usual
a aplicação de um método de otimização a uma parte específica do projeto (o da asa, por exemplo) e deixá-lo rodando por dias
para obter, no final, apenas uma resposta aproximada, que pode ou não ser viável quando se leva em consideração o resto do projeto.

Para estudar como melhor lidar com esse tipo de situação, surgiu o campo de estudos de \enphasis{Otimização Multidisciplinar de Projetos}
(ou \foreign{Multidisciplinary Design Optimization}, \definicao{MDO} na sigla em inglês, que será usada daqui para frente). O objetivo
dos métodos MDO são coordenar de maneira eficiente diferentes métodos de otimização para um projeto único, na esperança de assim obter projetos
melhores (``melhores'' nos termos definidos pelo ou pela projetista) do que os obtidos otimizando o projeto por partes isoladas. 
No entanto, mesmo depois de numerosos avanços em métodos MDO, aplicá-los a um projeto inteiro de uma aeronave ainda é impraticável
na maior parte dos casos.

Em última instância, a utilização de métodos de otimização para esse tipo de problema (o de projetos como o de uma aeronave, como no exemplo)
é feita a fim de se entender e analisar melhor o problema em mãos. Isso porque, no início, o problema a ser resolvido pelo projeto não é, no geral, bem posto (muitas vezes, os objetivos do projeto não são sequer definidos\cite{hoburg}),
e é de interesse entender o formato de sua fronteira de Pareto\footnote{Uma alocação de recursos (codificada aqui como a atribuição de valores
  a variáveis) é dita Pareto eficiente se não é possível alterar tal alocação tal que um critério de otimalidade seja melhorado sem que
  outro seja piorado. Fronteira de Pareto é o conjunto de alocações Pareto eficientes.}, a fim de se entender a relevância de cada característica (ou, poderíamos dizer, ``de cada variável'').
Isso, aliado ao fato de que geralmente um
projetista gostaria de otimizar vários parâmetros, e não só um, aumenta a necessidade de se obter não só uma  ``resposta'' otimizada, mas
também um maneira de entender a vizinhança dessa resposta (isto é, realizar uma análise de sensitividade).

Vale também notar que há um impasse entre a realização de uma análise de alta ou de baixa fidelidade. Para análises de alta fidelidade, existem duas opções:

\begin{itemize}
\item Fazer uma análise de uma parte específica (como o da asa, ou do motor) do projeto, o que possibilitaria a quem cuidar dele
    fazer decisões mais informadas sobre essa parte específica;
\item Fazer uma análise mais geral, envolvendo um subconjunto maior de ``partes específicas'', incorrendo em instâncias mais arbitrárias
  de problemas mais gerais, e leva, em geral, dias ou semanas para a obtenção de uma solução.
\end{itemize}

Esses tipos de problemas, assim como a possibilidade de \foreign{overfitting} \footnote{Uma solução ``está em'' \foreign{overfitting} quando
  levar a excelentes resultados para uma comparativamente pequena quantidade de casos de teste, mas a soluções ruins quando em situações
  mais gerais.} levam, em vários casos, a uma escolha por uma análise de baixa fidelidade.


Antes de seguirmos um pouco mais a fundo, convém obtermos uma ideia geral do desenvolvimento típico de uma aeronave. Esse desenvolvimento
é composto por três etapas\cite{hoburg} (ou, pelo menos, assim os engenheiros o têm tratado por algumas décadas):
\begin{itemize}
\item Projetagem conceitual: etapa em que os objetivos do projeto são definidos, assim como requisitos de performance. Dependendo
  do projeto, esta etapa pode levar vários anos e pode envolver uma análise matemática razoavelmente detalhada;  
\item Projetagem preliminar:
  Envolve uma análise mais aprofundada da configuração do projeto, identificação e resolução de problemas de interferência aerodinâmica e
  de instabilidade é realizada e, em um certo ponto, a configuração básica da aeronave é ``congelada'', de modo a permitir que times que
  trabalham com subsistemas possam trabalhar independentemente, sem afetar demais os outros timas;
\item Projetagem detalhada:
  Envolve uma projetagem detalhada do projeto, desenhos em CAD, determinação de passagens hidráulicas, elétricas e de combustível, assim
  como fabriação de peças de produção\footnote{Peças que são pensadas para uso em um projeto específico.}.
\end{itemize}

Cada uma dessas etapas pode ser envolve grandes problemas de otimização com restrições, envolvendo as dificuldades notadas acima.
    
\subsection{Programação Geométrica}

Nos últimos anos, técnicas de programação convexa têm se tornado cada vez mais eficientes (aproximando mesmo técnicas de programação linear).
Apesar disso, por incrível que possa pareça, tais técnicas ainda são notavelmente ausentes em propostas MDO\cite{hoburg} e a abordagem mais
comum a esses problemas é por meio de métodos gerais de programação não-linear. Assim, problemas muito gerais podem ser abordados, mas
dificilmente os critérios de confiança e e eficiência são preenchidos.

No lugar disso, então, voltamos nossa atenção a métodos de otimização convexa e, em particular, a um método que se mostrou particularmente
adequado para lidar com problemas de projeto de aeronave, o de programação geométrica\footnote{Diferente do que se pode supor, o nome
  ``programação geométrica'' não é devido às suas propriedades geométricas no geral, mas sim à desigualdade geométrico-aritmética,
  que foi muito usada em sua análise.}. Quando esse método é aplicável, entre as vantagens
que ele oferecem estão a obtenção de soluções globalmente ótimas (ou um certificado de infactibilidade, quando não existem tais soluções),
com tempos te solução comparativamente curtos, que escalam para problemas maiores (isto é, quando o problema aumenta ``um pouco'', o tempo
de solução aumenta só ``um pouco''). Ele tem a desvantagem de não ser um método geral, mas como, só estamos interessados (no momento)
em resolver problemas de projetagem de aeronaves, essa desvantagem não nos entristece muito.

Um problema de programação geométrica em forma padrão pode ser posto da seguinte forma:

\begin{align*}
  minimize \;& \sum\limits_{i=1}^n  f_0(x)\\
  tal\; que\;& f_i(x) \leq 1, i  = 1, ..., n\\
             & g_i(x) = 1, i = 1, ..., k,\\
             & x \in R^n_{+},
\end{align*}

\noindent em que $f_i$ são posinômios ou monômios, os $g_i$ são monômios e $x \in R^n$. Aqui, no entando, a definição
de monômios (e de posinômios) é um pouco diferente do usual em álgebra:

\begin{itemize}
\item Um \definicao{monômio} $g(x)$ é algo da forma $g(x) = kx^{\alpha}$, $\alpha \in R^n, k \in R^n_{+}$\footnote{Aqui, usamos a convenção de que, se $x \in R^n$ e $\alpha \in R^n$, $x^{\alpha} := \prod x_i^{\alpha_i}$.};
  \item Um \definicao{posinômio} é algo da forma $f(x) = \sum g_i(x)$, em que cada $g_i$ é um monômio (em particular, monômios são posinômios).
\end{itemize}

Note que esse problema não só é altamente não-linear, mas também é não-convexo. Nós podemos, no entanto, torná-lo convexo, com uma mudança de coordenadas apropriada. Note que, se $f$ é um posinômio em $x$, $log(f(e^x))$ é convexo e, se $g$ é um monômio em $x$, $log(g(e^x))$ é, não só
convexo, mas também afim. Realizando essa mudança de variáveis, nosso problema de otimização pode ser expresso como

\begin{align*}
  minimize \;& log \sum  e^{\alpha_{0k}'x + b_{0k}}\\
  tal\; que\;& log \sum e^{\alpha_{ik}' + b_{ik}} \leq 0\\
             & x \in R^n_{+}.
\end{align*}

Nessa nova formulação, estamos tomando $\alpha$ como a matriz de vetores que geram as potências dos monômios e posinômios. Note que, fazendo
essa mudança de variáveis ($v = log(x)$), monômios se tornam funções afim e, posinômios, funções convexas.

É importante notar que monômios e posinômios são um conjunto fechado sob as operações de divisão por monômios, então desigualdades do tipo $f_1(x) = f_2(x)$, em que $f_2$ é um monômio e $f_1$ é um posinômio poderiam ser lidadas fazendo $\frac{f_1(x)}{f_2(x)} = 1$. Ademais, igualdades como
$g(x) = 1$, se $g$ é um monômio, podem ser lidadas fazendo $g(x) \leq 1$ e $\frac{1}{g(x)} \leq 1$\footnote{Em particular, isso mostra que um programa geométrico qualquer pode ser expresso com desigualdades do tipo $f_i(x) \leq 1$, o que é a entrada que alguns \foreign{solvers} requerem por padrão.}. A restrição de que $k$ precisa ser positivo acaba sendo frequentemente satisfeita sem esforços mas, quando esse não é o caso, é preciso se usar outro maquinário que não o da programação geométrica. Em particular, programação signomial, que faz uso de uma estrutura parecida com a da geométrica, resolve esse caso (e mais eficiente do que um método de programação linear genérico).



    \begin{thebibliography}{1}

      \bibitem{hoburg}
        Hoburg, W. Warren,
        ``Aircraft Design Optimization as a Geometric Program''       
        2013.


     \end{thebibliography}

\end{document}
