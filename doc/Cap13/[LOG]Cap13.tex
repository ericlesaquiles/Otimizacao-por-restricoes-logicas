\documentclass{article}

\usepackage[utf8]{inputenc}
\usepackage{placeins}
\usepackage{graphicx}
\usepackage[ruled,vlined,linesnumbered]{algorithm2e}
\usepackage{listings}
\usepackage{amsthm}
\usepackage{amsmath}
\usepackage{caption}
\usepackage{epigraph}

\captionsetup[figure]{font=scriptsize}

\newtheorem{definition}{Definição}[section]
\theoremstyle{remark}
\newtheorem*{remark}{Observações}
\theoremstyle{theorem}
\newtheorem{theorem}{Teorema}[section]

\setlength{\parskip}{.5em}

\renewcommand\refname{Leituras adicionais}
\renewcommand\figurename{Figura}
\renewcommand{\lstlistingname}{Código}

\newcommand{\eclipse}{$ECL^iPS^e$}
\newcommand{\definicao}[1]{\textbf{#1}\marginpar{\small \textbf{#1}}}

\lstdefinestyle{prosty}{
  language=Prolog,
  basicstyle=\small
}





\DeclareMathAlphabet{\mathpzc}{OT1}{pzc}{m}{it}

\begin{document}

%COMMENT: acho que teve um salto dos capitulos anteriores para este. Ficou faltando uma "liga", que talvez o capitulo 12 possa dar.

\section{Dualidade e relaxação}

\epigraph{\cit{ - Q: Qual é o contrário do ``espaço dual''?\\
                - R: O espaço ``donothing''.}}{Eduardo Tengan, em uma aula qualquer}

A noção de dualidade não é uma noção perene só na álgebra linear e suas áreas adjacentes (conhecidas
em conjunto como ``matemática''), também ocupa um lugar especial na teoria de otimização. Em
otimização, dualidade vem em diversas formas: dualidade de programação linear, de programação
linear inteira (lagrangeana), entre outras. Frequentemente, os métodos de otimização mais
confiavelmente eficicientes são métodos primal-dual.

A dualidade é um tema unificador porque podemos expressar todos os duais como dual de inferência ou
de relaxação (frequentemente, como ambos). Um dual de inferência pode ser visto como o
problema de inferir das restrições nas variáveis um maior limitante inferior na imagem da função
objetivo (assumimos daqui para frente, sem perda de generalidade, que queremos minimizar a função). Assim, a busca ocorre
sobre provas, no lugar de sobre valores das variáveis. Mais precisamente, se tivermos um problema de
otimização da forma

\begin{align*}
  &  min \; f(x),\\
  &  R(x),\\
  &  x \in X,
\end{align*}

\noindent em que $R(x)$ denota as restrições sobre $x$, sendo $X$ o seu domínio, seu respectivo
\definicao{dual de inferência} nos coloca o problema de maximizar o limitante inferior em $f(x)$ que
é inferível das restrições. Isso ocorre na forma de uma busca por uma prova do limitante ótimo (ou
``bom o suficiente'', se não pudermos pedir por um ``ótimo''):

\begin{align*}
  max \; & v,\\
  C(x) & \vdash^{P} (f(x) \geq v),\\
  v \in \mathbb{R}&, P \in \mathcal{P},
\end{align*}

\noindent em que $C(x) \vdash^{P} (f(x) \geq v)$ indica que \var{P} é uma prova de $(f(x) \geq v)$,
usando o que é assumido em $C(x)$ (no caso, $C(x)$ indica as restrições). O domínio de \var{P} é uma família de provas $\mathcal{P}$, sendo
$(v,P)$ um par de solução dual. Quando o problema inicial (dito ``\definicao{primal}'') não tem
soluções, o conjunto de soluções para o dual é ilimitado.

Quando o primal for um problema de programação linear, por exemplo, a família de provas pode
consistir de combinações lineares não-negativas de restrições de desigualdade, sendo a solução dual
identificada com os multiplicadores na combinação linear que deriva o maior limitante.

Direto da definição, podemos inferir o \definicao{princípio de dualidade fraca}: um valor factível
\var{v} do dual de inferência nunca pode ser maior do que qualquer valor factível do primal, o que
nos possibilita limitar o valor para o ótimo do primal.
Ademais, a diferença (a ``lacuna'') entre os valores ótimos do primal e do dual é chamada
\definicao{lacuna de dualidade}. A lacuna de dualidade pode ser igual a zero, mas não é garantido
que esse possa ser o caso (isto é, não é garantido que as soluções ótimas primal e dual possam ter o
mesmo custo). Em particular, pode ser que exista um maior limitante inferior a $f(x)$, mas
esse limitante inferior não ser inferível em $\mathcal{P}$. Quando esse é o caso, $\mathcal{P}$ é
dita incompleta. Mais em geral, se $C(x)$ implica \var{C} mas não existe $P \in
\mathcal{P}$ que infira \var{C}, $\mathcal{P}$ é dita incompleta. Caso contrário, ela é dita
completa. Quando $\mathcal{P}$ for completa, a lacuna de dualidade pode, ao menos em princípio, ser
reduzida a zero pelo dual, o que
constitui o princípio de \definicao{dualidade forte}\footnote{Os ``princípios'' de dualidade fraca e
  forte são às vezes referidos como teoremas (porque precisam ser provados). Suas provas, no
  entanto, são tão simples, que foram omitidas a fim de não aborrecer a leitora.}.

A ideia de dualidade por inferência provê um plano unificador para várias ideias centrais em
otimização no geral e em programação por restrições em particular. Entre outras coisas, dualidade
por inferência nos provê métodos para \technical{aprendizado de cláusulas}, frequentemente muito
úteis. Buscas com \technical{aprendizado de cláusulas} cresceram de pesquisa em
\technical{aprendizado baseado em explicações}, um ramo de inteligência artificial que busca melhorar
a eficiência de métodos de busca por \technical{backtracking} por meio de ``explicações de falha'',
na forma de novas restrições ao problema original. Para CSPs em geral, tais ``explicações'' são
chamadas \technical{nogood} (ou, às vezes, ``conflitos''). Foi mostrado que uma grande quantidade de
problemas que não podem (até o presente momento) ser resolvidos por outras técnicas podem ser
resolvidos de maneira eficiente por \technical{aprendizado de cláusulas}. Em particular, problemas
(anteriormente) abertos em teoria de grupos foram resolvidos com essa técnica (entre outras). Para
mais informações, veja~\cite{beame}

\subsection{Duais}

Existem na literatura de programação por restrições, assim como na de programação matemática,
diversas noções específicas de dual, para diversos tipos de problemas diferentes. Entre elas, destacam-se
como especialmente úteis os duais \technical{substituto} e \technical{lagrangeano}, aos quais devotaremos alguma atenção
nesta seção e na próxima, colocando ênfase especial na relação entre eles. Então, nossa atenção se
voltará ao \technical{dual por ramificação}, que será útil em nosso posterior estudo de técnicas de
ramificação.

\subsubsection{Dual Substituto}

Seja um problema do tipo

\begin{equation}
  \begin{split}
    & min f(x),\\
    & g(x) \geq 0,\\
    & x \in X,
  \end{split}\label{eqt:geral}
\end{equation}

\noindent em que $g(x)$ não necessariamente é linear e  $x \in X$ indica as demais restrições em $x$. Note
que problemas de programação inteira mista são um caso particular em que $g(x)$ é uma função linear e $X =
\mathbb{R}^n + \mathbb{Z}^m$. Obtemos o dual de inferência desse problema tomando combinações
lineares não negativas e implicação como método de inferência, fazendo

\begin{equation}
  \begin{split}
    & max \; v,\\
    & (g(x) \geq 0) \vdash^P (f(x) \geq v),\\
    & P \in \mathcal{P}, v \in \mathbb{R},
  \end{split}\label{eqt:surro}
\end{equation}

\noindent em que cada prova em $\mathcal{P}$ corresponde a um vetor $u$ de multiplicadores e a prova
$P$ deduz $f(x) \geq v$ de $g(x) \geq 0$ quando a $ug(x) \geq 0 \Rightarrow f(x) \geq
v$. Isto é o mesmo que dizer que o mínimo de $f(x)$ com $ug(x) \geq 0$ e $x \in X$ é no mínimo $v$.
Daí segue que essa formulação é equivalente a

\begin{align*}
  max \;& \sigma(u),\\
  \sigma(u) &= min_{x\in X}\{f(x)|ug(x)\geq0\},\\
  u &\geq 0.
\end{align*}

Pergunta: o \technical{dual substituto} é completo?


\subsubsection{Dual Lagrangeano}

O dual lagrangeano do problema (\ref{eqt:geral}) é semelhante ao dual substituto, exceto que, no dual
lagrangeano, as provas fazem uso de combinações lineares e %QUESTION: era pra ser um "e" aqui mesmo?
\technical{dominação}\footnote{Dizemos que uma
desigualdade $g(x) \geq 0$ domina uma $h(x) \geq 0$ se $g(x) \leq h(x) \;\forall x$, ou se não existe $x$
tal que $g(x) \geq 0$.}. O dual é o problema (\ref{eqt:surro}), em que $\mathcal{P}$ consiste de provas
que mesclam combinação linear com dominação: $f(x) \geq v$ é inferido de $g(x) \geq 0$ se $\exists
\; l \geq 0: lg(x)$ domina $f(x) \geq v$.

Como dominação é um requisito mais fraco do que implicação, o dual lagrangeano resulta em um dual de
inferência mais fraco do que o \technical{substituto} e, assim, possibilita limitantes mais fracos
em relação ao \technical{substituto}. Apesar disso, para outros fatores o lagrangeano goza de propriedades mais
interessantes do que o \technical{substituto}, o que o torna atraente para uma grande variedade de problemas. Por exemplo,
o conjunto de soluções factíveis para o \technical{dual lagrangeano} é côncavo (o que não vale, no
geral, para o \technical{dual substituto}). Isso pode ser visto
facilmente se, supondo que $lg(x) \geq 0$ é factível para $l\geq0$, observarmos que o
lagrangeano maximiza $v$ com $l \geq 0$ e $lg(x) \geq f(x) - v \; \forall x \in X$. Reescrevendo $lg(x)
\leq f(x) - v$ como $v \leq f(x) - lg(x)$ e fazendo $\lambda(l,x) = f(x) - lg(x)$, nosso
lagrangeano é equivalente a

%\begin{equation}
  \begin{align}
     max &\; \lambda(l),\\
     \lambda(l) &:=  inf_{x \in X}\{f(x) - lg(x)\},\label{eqt:lambda}\\
     l & \geq 0,
  \end{align}
%\end{equation}

\noindent e o conjunto de ínfimos de funções afim (como em (\ref{eqt:lambda})) é um conjunto côncavo
($\lambda$ é afim para $x$ fixo). O vetor $l$ é popularmente conhecido como vetor de multiplicadores
de lagrange.

\subsubsection{Dualidade por Ramificação}

Intuitivamente, se temos uma árvore de busca para um problema de otimização, essa árvore vista ``de
baixo para cima'', isto é, das folhas para a raíz, nos provê um certificado de otimalidade a uma
solução ótima. Além disso, dada tal árvore de busca, cada subárvore dela constitui uma relaxação ao
problema original e provê um limitante ao custo ótimo do problema original.
Assim, parece natural definirmos um dual por ramificação ao problema de otimizar um $f(x)$ nas
folhas de uma árvore de busca $\mathpzc{T}$ da seguinte forma:

\begin{equation}
  \begin{split}
    max\; B(\mathpzc{T'}),\\
    \mathpzc{T'} \in \mathcal{T},
  \end{split}
\end{equation}

\noindent em que $B$ é o limitante de $f$ em $\mathpzc{T'}$ e $\mathcal{T}$ é uma coleção de
subárvores de $\mathpzc{T}$.

As estratégias de ramificação que veremos logo mais podem ser vistas como instâncias do dual de
ramificação. Para mais detalhes sobre o dual de inferência veja \cite{hooker}.


\subsection{Complexidade}

Um certificado de factibilidade (infactilibidade/otimalidade) é um pedaço de
informação que permite a verificação da factibilidade (infactibilidade/otimalidade) da instância de
um problema. Dizemos que um problema de otimização pertence ao NP (de \foreign{nondeterministic
  polynomial}) se existe um certificado de factibilidade polinomial para qualquer instância factível
do problema: a computação requerida para testar factibilidade, pelo certificado, é limitada por uma
função polinomial do tamanho da instância do problema. Um problema de otimização é co-NP se existe
um certificado polinomial de otimalidade para qualquer instância.

Se um problema de otimização primal for co-NP, seu dual de inferência é NP para alguma família de
provas $\mathcal{P}$: se o problema for co-NP, seja $\mathcal{P}$ a família de provas $P_i$ (que
assumimos polinomiais) de otimalidade da instância $i$, de valor ótimo $v_i$, provê um certificado
polinomial de factibilidade do dual (que é, portanto, NP).

Programação linear pertence a ambos NP e co-NP. Problemas desse tipo são tidos ter ``boa
caracterização'', no sentido de que soluções factíveis e provas de otimalidade são facilmente
escritas.

  \begin{thebibliography}{1}

    \bibitem{beame}
    Beame Paul and Kautz Henry and Sabharwal Ashish,
    Towards Understanding and Harnessing the Potential of Clause Learning,
    Journal of Artificial Intelligence Research 22 (2004) 319-351.


    \bibitem{hooker}
    Hooker N. John, Integrated Methods for Optimization, Second Edition, Springer Verlag, 2012.

  \end{thebibliography}

\end{document}
