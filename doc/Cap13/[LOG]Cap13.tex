\documentclass{article}

\usepackage[utf8]{inputenc}
\usepackage{placeins}
\usepackage{graphicx}
\usepackage[ruled,vlined,linesnumbered]{algorithm2e}
\usepackage{listings}
\usepackage{amsthm}
\usepackage{amsmath}
\usepackage{caption}
\usepackage{epigraph}

\captionsetup[figure]{font=scriptsize}

\newtheorem{definition}{Definição}[section]
\theoremstyle{remark}
\newtheorem*{remark}{Observações}
\theoremstyle{theorem}
\newtheorem{theorem}{Teorema}[section]

\setlength{\parskip}{.5em}

\renewcommand\refname{Leituras adicionais}
\renewcommand\figurename{Figura}
\renewcommand{\lstlistingname}{Código}

\newcommand{\eclipse}{$ECL^iPS^e$}
\newcommand{\definicao}[1]{\textbf{#1}\marginpar{\small \textbf{#1}}}

\lstdefinestyle{prosty}{
  language=Prolog,
  basicstyle=\small
}




\begin{document}

\section{Dualidade e relaxação}

\epigraph{\cit{ - Q: Qual é o contrário do ``espaço dual''?\\
                - R: O espaço ``donothing''.}}{Eduardo Tengan, em uma aula qualquer}

A noção de dualidade não é uma noção perene só na álgebra linear e suas áreas adjacentes (conhecidas
em conjunto como ``matemática''), também ocupa um lugar especial na teoria de otimização. Em
otimização, dualidade vem em diversas formas: dualidade de programação linear, de programação
linear inteira (lagrangeana), entre outras. Frequentemente, os métodos de otimização mais
confiavelmente eficicientes são métodos primal-dual.

A dualidade é um tema unificador porque podemos expressar todos os duais como dual de inferência ou
de relaxação (frequentemente vezes, como ambos). Um dual de inferência pode ser visto como o
problema de inferir das restrições nas variáveis um maior limitante inferior na imagem da função
objetivo (assumimos daqui para frente, sem perda de generalidade, que queremos minimizar a função). Assim, a busca ocorre
sobre provas, no lugar de sobre valores das variáveis. Mais precisamente, se tivermos um problema de
otimização da forma

\begin{align}
  &  min f(x)\\
  &  R(x)\\
  &  x \in D
\end{align}

\noindent em que $R(x)$ denota as restrições sobre x e D o seu domínio, seu respectivo
\definicao{dual de inferência} nos coloca o problema de maximizar o limitante inferior em $f(x)$ que
é inferível das restrições. Isso ocorre na forma de uma busca por uma prova do limitante ótimo (ou
``bom o suficiente'', se não pudermos pedir por um ``ótimo''):

\begin{align}
  max \; & v\\
  C(x) & \vdash^{P} (f(x) \geq v)\\
  v \in \mathbb{R}&, P \in \mathcal{P}
\end{align}

\noindent em que $C(x) \vdash^{P} (f(x) \geq v)$ indica que \var{P} é uma prova de $(f(x) \geq v)$,
usando o que é assumido em $C(x)$. O domínio de \var{P} é uma família de provas $\mathcal{P}$, sendo
$(v,P)$ um par de solução dual. Quando o problema inicial (dito ``\definicao{primal}'') não tem
soluções, o conjunto de soluções para o dual é ilimitado.

Quando o primal for um problema de programação linear, por exemplo, a família de provas pode
consistir de combinações lineares não-negativas de restrições de desigualdade, sendo a solução dual
identificada com os multiplicadores na combinação linear que deriva o maior limitante.

Direto da definição, podemos inferir o \definicao{princípio de dualidade fraca}: um valor factível
\var{v} do dual de inferência nunca pode ser maior do que qualquer valor factível do primal.
Ademais, a diferença (a ``lacuna'') entre os valores ótimos do primal e do dual é chamada
\definicao{lacuna de dualidade}. A lacuna de dualidade pode ser igual a zero, mas não é garantido
que esse vá ser o caso. Em particular, pode ser que exista um maior limitante inferior a $f(x)$, mas
esse limitante inferior não ser inferível em $\mathcal{P}$. Quando isso acontece, $\mathcal{P}$ é
incompleta. Mais no geral, se \var{C} é uma restrição implicada por $C(x)$ mas não existe $P \in
\mathcal{P}$ que infira \var{C}, $\mathcal{P}$ é dita incompleta. Caso contrário, ela é dita
completa. Quando $\mathcal{P}$ for completa, a lacuna de dualidade pode ser reduzida a zero, o que
constitui o princípio de \definicao{dualidade forte}\footnote{Os princípios de dualidade fraca e
  forte são às vezes referidos como teoremas (porque precisam ser provados). Suas provas, no
  entanto, são tão simples, que foram omitidas a fim de não aborrecer o leitor.}.

\subsection{Complexidade}

Um certificado de factibilidade (infactilibidade/otimalidade) é um pedaço de
informação que permite a verificação da factibilidade (infactibilidade/otimalidade) da instância de
um problema. Dizemos que um problema de otimização pertence ao NP (de \foreign{nondeterministic
  polynomial}) se existe um certificado de factibilidade polinomial para qualquer instância factível
do problema: a computação requerida para testar factibilidade, pelo certificado, é limitada por uma
função polinomial do tamanho da instância do problema. Um problema de otimização é co-NP se existe
um certificado polinomial de otimalidade para qualquer instância.

Se um problema de otimização primal for co-NP, seu dual de inferência é NP para alguma família de
provas $\mathcal{P}$: se o problema for co-NP, seja $\mathcal{P}$ a família de provas $P_i$ (que
assumimos polinomiais) de otimalidade da instância i, de valor ótimo $v_i$, provê um certificado
polinomial de factibilidade do dual (que é, portanto, NP).

Programação linear pertence a ambos NP e co-NP. Problemas desse tipo são tidos ter ``boa
caracterização'', no sentido de que soluções factíveis e provas de otimalidade são facilmente
escritas.

  \begin{thebibliography}{1}

    \bibitem{carnap}

  \end{thebibliography}

\end{document}
