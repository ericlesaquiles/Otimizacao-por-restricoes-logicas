%\documentclass{article}

%\documentclass[tikz,border=5]{standalone}
%\usetikzlibrary{graphs,graphdrawing,arrows.meta}
%\usegdlibrary{trees}

%\usepackage[utf8]{inputenc}
\usepackage{placeins}
\usepackage{graphicx}
\usepackage[ruled,vlined,linesnumbered]{algorithm2e}
\usepackage{listings}
\usepackage{amsthm}
\usepackage{amsmath}
\usepackage{caption}
\usepackage{epigraph}

\captionsetup[figure]{font=scriptsize}

\newtheorem{definition}{Definição}[section]
\theoremstyle{remark}
\newtheorem*{remark}{Observações}
\theoremstyle{theorem}
\newtheorem{theorem}{Teorema}[section]

\setlength{\parskip}{.5em}

\renewcommand\refname{Leituras adicionais}
\renewcommand\figurename{Figura}
\renewcommand{\lstlistingname}{Código}

\newcommand{\eclipse}{$ECL^iPS^e$}
\newcommand{\definicao}[1]{\textbf{#1}\marginpar{\small \textbf{#1}}}

\lstdefinestyle{prosty}{
  language=Prolog,
  basicstyle=\small
}



\forestset{
  tree angle/.style={
    tikz={\path () coordinate (A) -- (!u) coordinate (B) -- (!n) coordinate (C) pic [draw] {angle};}
  }
}
\usetikzlibrary{arrows.meta,angles}


%
% TODO: revisar terminar esta seção!
%
%


%\begin{document}

\section{Programas como árvores}

Será conveniente termos uma interpretação mais visual de programas lógicos. Isso nos dará uma ideia melhor de em qual ordem ocorrem as tentativas de unificações e como lidar com elas.
Considere o seguinte programa, representando um pedaço da árvore genealógica da dinastia Julio-Claudiana, junto com a relação ancestral\footnote{\foreign{``It is, of course, obvious at once that `ancestor' must be capable of definition in terms of
`parent', but until Frege developed his generalised theory of induction, no one could have defined `ancestor' precisely in terms of `parent.' ''} --- Bertrand Russel,\foreign{Introduction to Mathematical Philosophy}.
Para sermos honestos: na verdade, ele se referia a uma versão mais geral dessa relação do que a apresentada aqui. Voltaremos a esse tema depois.}:

    \begin{listing}[H]
\inputminted{prolog}{../Exemplos/Cap5/prog1_ancestor.pl}
\caption{Ancestral 0}
    \end{listing}

Por conveniência, no lugar das relações filhx/2 e ancestral/2 e dos nomes mostrados acima, usaremos
os mostrados no Código \ref{lst:ancestor1}\.

    \begin{listing}[H]
\inputminted{prolog}{../Exemplos/Cap5/prog2_ancestor.pl}
\caption{Ancestral 1}\label{lst:ancestor1}
    \end{listing}


O goal \codigo{a(au, ne)?} define implicitamente uma árvore, que começa da forma a seguir.

\begin{center}
  {\footnotesize
    \begin{forest}
      for tree={delay={where content={}{content={\phantom{00}}}{}},s sep+=5mm,l+=5mm}
      [\codigo{a(au, ne)?}
        [ \codigo{f(ne, au)?},edge={dashed,black,thick}
          [falha]
        ]
        [\codigo{f(ne,C1), a(au,C1)?}
          [ {C1 = ag},edge=dashed
            [\codigo{f(ne, ag)?}]
            [\codigo{a(au, ag)?}
                [\codigo{f(ag,au)?},edge=dashed
                  [falha]
                ]
              [\codigo{f(ag,C2), a(au,C2)?}
                [$\vdots$]
              ]
            ]
          ]
          [ {C1 = gn},edge=dashed
            [\codigo{f(ne, gn)?}]
            [\codigo{a(au, gn)?}
              [\codigo{f(gn,C2), a(au,C2)?}
                [$\vdots$]
              ]
              [\codigo{f(gn, C2)?},edge=dashed
                [falha]
              ]
            ]
          ]
        ]
      ]
    \end{forest}
  }
\end{center}

A árvore inteira não será colocada por questões de espaço, mas continua de maneira semelhante. Os ramos tracejados indicam
um ``ou'' lógico, enquanto que os sólidos indicam um ``e'' lógico. O que isso significa é que falha em um dos ramos dos arcos
sólidos indica que todo o arco é falho, mas uma falha em um ramo tracejado só afeta o ramo tracejado.

A execução de um programa lógico consiste em uma busca, em uma árvore como essa, por uma folha de ``sucesso''. Se, ao invés
de ``sucesso'', é encontrada ``falha'', é feito o \technical{backtracking}. Quando alguém fala sobre um ``modelo computacional de
programa lógico'', fala essencialmente sobre como percorrer essa árvore e como fazer esse \technical{backtracking}.
Como pode ver, ela pode crescer muito em largura a cada nível (não cresceu tanto no exemplo acima porque colocamos uma quantidade reduzida de substituições). Por isso, é compreensível que prefiramos uma busca em profundidade
do que em largura. Isso pode, é claro, levar a problemas técnicos e filosóficos, já que impõe uma escolha arbitrária sobre a ordem
de avaliação das cláusulas e dos termos de cada cláusula, o que pode levar a comportamentos inesperados ao programador desatento a esse modelo\footnote{Convém lembrar que não estamos lidando com programação paralela.
No modelo de programação paralela a avaliação pode ocorrer de maneira
diferente.}. É claro, dizer apenas que a busca é em profundidade não é o suficiente: precisamos dizer que ela é em profundidade e
à esquerda.



O seguinte exemplo, um homem reclamando de sua vida, mostra que questões de parentesco podem ser bem
mais complicadas do que isso (o exemplo é baseado na história retirada de \cite{antoni}, frequentemente
atribuída a Mark Twain):

\cit{
  Me casei com uma viúva com uma filha crescida. Meu pai, que nos visitava frequentemente, se
  apaixonou com a %QUESTION: copiou de algum lugar ou traduziu/modificou? Se for a segunda opcao, fica melhor trocar o "com a" por "pela"
  filha e a tomou como sua esposa. Isso fez do meu pai o meu filho adotado, e, de
  minha filha adotada, minha madrasta.\\
  Depois de um ano, minha esposa deu à luz um filho, que se tornou o irmão adotado do meu pai e, ao
  mesmo tempo, meu tio, já que ele era o irmão de minha madrasta.\\
  Mas a esposa de meu pai, isto é, minha filha adotada, também deu à luz um filho. Então, ele era
  meu irmão e também meu neto, já que ele era o filho de minha filha.\\
  Isso quer dizer que eu me casei com minha avó, já que ela era a mãe de minha mãe. Como o marido de
  minha esposa, eu também era o neto adotado dela.\\
  Nossos amigos dizem que eu sou meu próprio avô. Isso é verdade?
}

O personagem dessa história, não sabendo Prolog, teve uma certa dificuldade em responder a essa
questão. Mas nós, como que por reflexo, fazemos o seguinte programa:

%TODO fix
    \begin{listing}[H]
\inputminted{prolog}{../Exemplos/Cap5/prog1_family.pl}
\caption{Família}
    \end{listing}

O último bloco de código na verdade não faz parte do programa, mas foi deixado por conveniência. Ele
indica o goal (na verdade, só queremos saber se o pobre coitado é avô de si mesmo, mas os outros
termos foram deixados por completeza). Você tem ideia de como é a árvore de busca para esse goal?

%QUESTION: os goals nao iam ter uma ? no final, em vez de ponto?

  \begin{thebibliography}{2}

    \bibitem{antoni}
    Niederliński, Antoni,
    ``A gentle guide to constraint logical programming via Eclipse'',
    3rd edition, Jacek Skalmierski Computer Studio, Gliwice, 2014

    \bibitem{russel} Russell, Bertrand (1919), Introduction to Mathematical Philosophy, George Allen and Unwin, London, UK. Reprinted, John G. Slater (intro.), Routledge, London, UK, 1993
    \\Esse livro está datado em alguns pontos, mas permanece interessante. Está gratuitamente
    disponível (em inglês) em \url{http://people.umass.edu/klement/russell-imp.html}

  \end{thebibliography}

%\end{document}
