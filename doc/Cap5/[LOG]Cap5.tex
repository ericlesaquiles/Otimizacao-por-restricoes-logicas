\documentclass{article}

%\documentclass[tikz,border=5]{standalone}
%\usetikzlibrary{graphs,graphdrawing,arrows.meta}
%\usegdlibrary{trees}

\usepackage[utf8]{inputenc}
\usepackage{placeins}
\usepackage{graphicx}
\usepackage[ruled,vlined,linesnumbered]{algorithm2e}
\usepackage{listings}
\usepackage{amsthm}
\usepackage{amsmath}
\usepackage{caption}
\usepackage{epigraph}

\captionsetup[figure]{font=scriptsize}

\newtheorem{definition}{Definição}[section]
\theoremstyle{remark}
\newtheorem*{remark}{Observações}
\theoremstyle{theorem}
\newtheorem{theorem}{Teorema}[section]

\setlength{\parskip}{.5em}

\renewcommand\refname{Leituras adicionais}
\renewcommand\figurename{Figura}
\renewcommand{\lstlistingname}{Código}

\lstdefinestyle{prosty}{
  language=Prolog,
  basicstyle=\small
}



\forestset{
  tree angle/.style={
    tikz={\path () coordinate (A) -- (!u) coordinate (B) -- (!n) coordinate (C) pic [draw] {angle};}
  }
}
\usetikzlibrary{arrows.meta,angles}


%
% TODO: revisar terminar esta seção!
%
%


\begin{document}

\section{Programas como árvores}

Será conveniente termos uma interpretação mais visual de programas lógicos. Isso nos dará uma ideia melhor de em qual ordem ocorrem as tentativas de unificações e como lidar com elas.
Considere o seguinte programa, representando um pedaço da árvore genealógica da dinastia Julio-Claudiana, junto com a relação ancestral\footnote{\foreign{``It is, of course, obvious at once that `ancestor' must be capable of definition in terms of
`parent', but until Frege developed his generalised theory of induction, no one could have defined `ancestor' precisely in terms of `parent.' ''} --- Bertrand Russel,\foreign{Introduction to Mathematical Philosophy}.
Para sermos honestos: na verdade, ele se referia a uma versão mais geral dessa relação do que a apresentada aqui. Voltaremos a esse tema depois.}:

\inputminted{prolog}{../Exemplos/Cap5/prog1_ancestor.pl}

Por conveniência, no lugar das relações filhx/2 e ancestral/2 e dos nomes mostrados acima, usaremos os mostrados abaixo:

\inputminted{prolog}{../Exemplos/Cap5/prog2_ancestor.pl}


O goal \codigo{a(au, ne)?} define implicitamente uma árvore, que começa da seguinte forma:
\begin{center}
  {\footnotesize
    \begin{forest}
      for tree={delay={where content={}{content={\phantom{00}}}{}},s sep+=5mm,l+=5mm}
      [\codigo{a(au, ne)?}
        [ \codigo{f(ne, au)?},edge={dashed,black,thick}
          [falha]
        ]
        [\cocigo{f(ne,C1), a(au,C1)?}
          [ {C1 = ag},edge=dashed
            [\codigo{f(ne, ag)?}]
            [\codigo{a(au, ag)?}
                [\codigo{f(ag,au)?},edge=dashed
                  [falha]
                ]
              [\codigo{f(ag,C2), a(au,C2)?}
                [$\vdots$]
              ]
            ]
          ]
          [ {C1 = gn},edge=dashed
            [\codigo{f(ne, gn)?}]
            [\codigo{a(au, gn)?}
              [\codigo{f(gn,C2), a(au,C2)?}
                [$\vdots$]
              ]
              [\codigo{f(gn, C2)?},edge=dashed
                [falha]
              ]
            ]
          ]
        ]
      ]
    \end{forest}
  }
\end{center}

A árvore inteira não será colocada por questões de espaço, mas continua de maneira semelhante. Os ramos tracejados indicam
um ``ou'' lógico, enquanto que os sólidos indicam um ``e'' lógico. O que isso significa é que falha em um dos ramos dos arcos
sólidos indica que todo o arco é falho, mas uma falha em um ramo tracejado só afeta o ramo tracejado.

A execução de um programa lógico consiste em uma busca, em uma árvore como essa, por uma folha de ``sucesso''. Se, ao invés
de ``sucesso'' é encontrada ``falha'', é feito o \technical{backtracking}. Quando alguém fala sobre um ``modelo computacional de
programa lógico'', fala essencialmente sobre como travessar essa árvore e como fazer esse \technical{backtracking}.
Como pode ver, ela pode crescer muito em largura a cada nível (não cresceu tanto no exemplo acima porque colocamos uma quantidade reduzida de substituições). Por isso, é compreensível que prefiramos uma busca em profundidade
do que em largura. Isso pode, é claro, levar a problemas técnicos e filosóficos, já que impõe uma escolha arbitrária sobre a ordem
de avaliação das cláusulas e dos termos de cada cláusula, o que pode levar a comportamentos inesperados ao programador desatento a esse modelo\footnote{Convém lembrar que não estamos lidando com programação paralela.
No modelo de programação paralela a avaliação pode ocorrer de maneira
diferente}. É claro, dizer apenas que a busca é em profundidade não é o suficiente: precisamos dizer que ela é em profundidade e
à esquerda.






  \begin{thebibliography}{2}

    \bibitem{russel} Russell, Bertrand (1919), \textit{Introduction to Mathematical Philosophy, George Allen and Unwin, London, UK. Reprinted, John G. Slater (intro.), Routledge, London, UK, 1993}
    \\Esse livro está datado em alguns pontos, mas permanece interessante. Está gratuitamente disponível (em inglês) em [http://people.umass.edu/klement/russell-imp.html].

  \end{thebibliography}

\end{document}
