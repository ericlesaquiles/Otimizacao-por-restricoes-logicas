%\documentclass{article}

%\usepackage[utf8]{inputenc}
\usepackage{placeins}
\usepackage{graphicx}
\usepackage[ruled,vlined,linesnumbered]{algorithm2e}
\usepackage{listings}
\usepackage{amsthm}
\usepackage{amsmath}
\usepackage{caption}
\usepackage{epigraph}

\captionsetup[figure]{font=scriptsize}

\newtheorem{definition}{Definição}[section]
\theoremstyle{remark}
\newtheorem*{remark}{Observações}
\theoremstyle{theorem}
\newtheorem{theorem}{Teorema}[section]

\setlength{\parskip}{.5em}

\renewcommand\refname{Leituras adicionais}
\renewcommand\figurename{Figura}
\renewcommand{\lstlistingname}{Código}

\lstdefinestyle{prosty}{
  language=Prolog,
  basicstyle=\small
}



%\begin{document}

\section{Restrições ativas}

Regras de propagação como as vistas no capítulo passado possibilitam o
que é chamado ``restrição ativa''. Sem essas regras de propagação,
tínhamos que buscar a solução de nossos problemas ``na mão'', isto é,
por aquele método primitivo de testar cada solução e dascartar as que
falham. Isso ainda era verdade com o uso da bilioteca
\technical{suspend}. Considere, por exemplo, a restrição
\codigo{suspend:(X or Y), X = 0.}. Essa restrição tem a solução
${X=0,Y=1}$, mas, na forma como está escrita, a restrição não nos
permite encontrar essa solução (\codigo{X or Y} estará suspenso até
que Y receba um valor, o que não acontece, já que \funtor{or/2} só
lida com variáveis instanciadas). Existem, no entanto, no sistema
\eclipse, bibliotecas que lidam com a restrição de forma ``ativa''
(isto é, fazem uso de propagações). Veremos como lidar com duas delas
momentaneamente, a \technical{ic\_symbolic} e a \technical{ic}.

Considere a seguinte situação (adaptada de \cite{antoni}):

Depois da queda do comunismo em Absurdolândia, vários clubes de debate
``Preto e Branco'' cresceram rapidamente pelo país. Eles eram clubes
exclusivos: apenas antigos \enphasis{Colaboradores Secretos} (do
antigo Serviço de Segurança Comunista) ou antigos
\technical{Oposicionistas Honrados} (que costumavam ser caçados pelos
antigos membros Serviço de Segurança Comunista). Esse tipo de
associação fez bastante sucesso, já que proveu um solo fértil para
discussões contraditórias, amadas pelo público, pela mídia televisiva
e por jornalistas. Isso também impulsionou o consumo de todo aquele
tipo de bebida que tem uma merecida reputação de facilitar o
entendimento de assuntos complicados. A atratividade das discussões
foi ainda mais realçada pelo conhecimento comum de que
\technical{Oposicionistas Honrados} sempre dizem a verdade, enquanto
que \enphasis{Colaboradores Secertos} diziam alternadamente a verdade
e a mentira. O bem conhecido tablóide local ``Notícias da Cloaca''
delegou a um dos clubes um Jornalista Celebrado para fazer uma
reportagem promovendo a ideia de reconciliação entre os inimigos do
passado. Infelizmente, o Jornalista Celebrado encontrou um problema:
no momento de sua chegada, o clube tinha apenas três membros, dos
quais Membro1 e Membro2 argumentavam ferozmente, evidentemente porque
pertenciam a diferentes grupos de membros. O jornalista, não querendo
importunar os adversários, perguntou ao Membro3, que não tomava parte
no argumento, se ele costumava ser um \enphasis{Oposicionista Honrado}
ou um \enphasis{Colaborador Secreto}. Infelizmente, Membro3 já tinha
tomado muito das bebidas supramencionadas e resmungou algo
ininteligível. O Jornalista Celebrado perguntou então aos dois membros
restantes sobre o que o Membro3 havia dito. Membro1, que talvez devido
a alguma habilidade que havia praticado, pôde entender a resposta do
Membro3, afirmou que o Membro3 disse que havia sido um
\enphasis{Oposicionista Honrado}. No entanto, Membro2 primeiro disse
que Membro3 foi um \enphasis{Colaborador Secreto} e, então, adicionou
que o Membro3 havia mentido. O Celebrado Jornalista tem informação
suficiente para inferir quem é quem?

Para resolver esse problema, faremos uso das bibliotecas
\technical{ic} e \technical{ic\_symbolic}, a qual é uma adição à
biblioteca \technical{ic}, implementando variáveis e restrições sobre
sobre domínios simbólicos ordenados (como já mencionado
anteriormente). Para modelarmos esse problema, faremos uso das
seguintes observações básicas:

\begin{itemize}
\item Membro1 e Membro3 disseram alguma coisa.
\item Membro2 disse duas coisas.
\item Cada coisa que foi dita pode ser verdadeira ou falsa.
\item Sabemos o que Membro1 e Membro2 disseram, mas só sabemos o que
  o Membro3 possivelmente disse.
\end{itemize}

Assim, temos uma variável para cada membro, que é ou um
\enphasis{Oposicionista Honrado} ou um \enphasis{Colaborador Secreto},
assim como para cada coisa dita por eles (daqui para frente referida
como ``resmungo''), que pode ser verdadeira ou falsa. Mais importante
são as relações entre essas variáveis.

Para a resolução do problema, notamos que os domínios a serem usados
pela \technical{ic\_symbolic} precisam ser declarados, o que faremos
por uso da construção \codigo{:- local
  domain(domain\_name($domain\_value_1$, ..., $domain\_value_n$))} em
que ``domain\_name'' representa o nome do domínio e $domain\_value_i$
representa o $i$-ésimo valor (simbólico) no domínio.  O valor de cada
variável de resmungo pode ser ``verdadeiro'' ou ``falso'', aqui
representados pelos valores aritméticos 0 e 1. O valor de cada
variável de Membro pode receber os valores ``oposicionista\_honrado''
e ``colaborador\_secreto''. O que cada Membro disse é uma afirmação
sobre o grupo ao qual outro membro faz parte ou sobre a veracidade do
que outro Membro disse. Essas observações são traduzidas no seguinte
código:

\begin{listing}[H]
  \inputminted{prolog}{../Exemplos/Cap10/prog1_pretoEbranco.ecl}
  \caption{Preto e Branco}
\end{listing}

É um código simples e que não precisa de muitos comentários, mas cabem
alguns:

\begin{itemize}
\item As relações entre os Membros (termos simbólicos) são
  realizadas por restrições de    \technical{ic\_symbolic} (vale
  lembrar, os ``\#'' indicam que a restrição têm o significado
  usual, mas nos inteiros).
\item As entre o que eles disseram (ou possivelmente disseram), por
  restrições aritméticas (de \technical{ic}).
\item Enquanto \codigo{[Membro\_1, Membro\_2, Membro\_3] \&::
  membro\_do\_clube} indica o domínio das variáveis,
  \codigo{ic\_symbolic:indomain(Membro\_1)} faz a atribuição de
  valores (segundo as restrições).
\item Em \codigo{resmungo\_unico(Membro, Verdade):- (Membro \&=}
  \\\codigo{oposicionista\_honrado) => Verdade}, o símbolo
  ``\codigo{=>}'' é o de implicação como em lógica clássica.
\item Restrições como \codigo{(Membro \&= colaborador\_secreto)} são
  reificadas, isto é, assumem um valor de ``verdadeiro'' ou
  ``falso'' (na prática, 0 ou 1).
\item As linhas tais como \codigo{writeln("Membro\_1":Membro\_1)}
  escrevem ``Membro1~:~valor'', na qual ``valor'' é o valor de
  Membro1.
\end{itemize}

Uma particularidade desse código é que ele resolve o problema sem a
necessidade de \technical{backtracking}, no que é chamado de
\technical{backtracking-free search} (apesar de ser uma propagação e
não uma busca propriamente dita). Essa foi uma situação excepcional,
uma vez que geralmente é necessário fazer uma busca para se chegar à
solução (na verdade, mesmo quando não é necessário usar busca, sua
introdução pode agilizar o processo).

\subsection{Backtracking raso}

Busca por \technical{backtracing} raso é um processo de busca no qual
é permitido \technical{backtracking}, mas de forma limitada: ao
atribuir um valor a uma variável, o processo de propagação é
desencadeado e, se ocorre uma falha, o próximo valor é tentado. Para
realizar essa busca, precisamos de predicados que nos permitam o
acesso do domínio atual de uma variável. Um desses predicados é o
\funtor{get\_domain\_as\_list/2}. Ele toma como primeiro argumento uma
variável com um domínio e o segundo argumento é instanciado a uma
lista com os valores contidos do domínio da variável. Com essa lista
em mãos, podemos testar cada valor a partir de \funtor{member/2}:
\codigo{get\_domain\_as\_list(X, Dominio), member(X, Dominio)}. Essa
combinação (de \funtor{get\_domain\_as\_list/2} e \funtor{member/2}) é
tão usada que foi criado o predicado
\funtor{indomain/1}\footnote{Perceba como ele foi usado no programa
  anterior.} para realizar a mesma função (exceto que de forma mais
eficiente). O predicado \funtor{indomain/1} age como se definido por:

\begin{listing}[H]
  \inputminted{prolog}{../Exemplos/Cap10/prog2_indomain.pl}
  \caption{Indomain}
\end{listing}

Em mãos desse predicado, podemos fazer o \technical{backtracking} raso
como se segue:

\begin{listing}[H]
  \inputminted{prolog}{../Exemplos/Cap10/prog3_backraso.ecl}
  \caption{Back Raso}
\end{listing}

Note que busca por \technical{backtracking} raso não é um resolvedor
completo.

\subsection{Busca por backtracking}

Fazer uma busca por \technical{backtracking} por meio da enumeração de
todos os valores no domínio não é eficiente na presença de
propagadores, já que os domínios diminuem (ou, ao menos, assim
esperamos). No lugar disso, seria mais interessante fazer uma busca
por \backtracking\ que faça uso apenas dos valores nos domínios
atuais. Essa observação nos leva a uma revisão do programa de busca
apresentado no Capítulo 8: %TODO adicionar link

\begin{listing}[H]
  \inputminted{prolog}{../Exemplos/Cap10/prog4_busca.ecl}
  \caption{Busca}
\end{listing}

Sendo um procedimento de busca tão comum, ele está disponível em
algumas bibliotecas \eclipse (na \enphasis{ic} e na \enphasis{sd}, por
exemplo) sob o nome \funtor{labeling/1} (o nome \technical{labeling} é
devido a razões históricas).



\begin{thebibliography}{2}

\bibitem{antoni} Niederliński, Antoni, ``A gentle guide to
  constraint logical programming via Eclipse'', 3rd edition, Jacek
  Skalmierski Computer Studio, Gliwice, 2014

\end{thebibliography}

%\end{document}
