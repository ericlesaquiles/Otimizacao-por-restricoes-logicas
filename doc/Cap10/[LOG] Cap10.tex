\documentclass{article}

\usepackage[utf8]{inputenc}
\usepackage{placeins}
\usepackage{graphicx}
\usepackage[ruled,vlined,linesnumbered]{algorithm2e}
\usepackage{listings}
\usepackage{amsthm}
\usepackage{amsmath}
\usepackage{caption}
\usepackage{epigraph}

\captionsetup[figure]{font=scriptsize}

\newtheorem{definition}{Definição}[section]
\theoremstyle{remark}
\newtheorem*{remark}{Observações}
\theoremstyle{theorem}
\newtheorem{theorem}{Teorema}[section]

\setlength{\parskip}{.5em}

\renewcommand\refname{Leituras adicionais}
\renewcommand\figurename{Figura}
\renewcommand{\lstlistingname}{Código}

\lstdefinestyle{prosty}{
  language=Prolog,
  basicstyle=\small
}



\begin{document}

\section{Restrições ativas}

Regras de propagação como as vistas no Capítulo passado possibilitam o que é chamado ``restrição ativa''. Se essas regras de
propagação, tínhamos que buscar a solução de nossos problemas ``na mão'', isto é, por aquele método primitivo de testar cada
solução e dascartar as que falham. Isso ainda era verdade com o uso da bilioteca \textit{suspend}. Considere, por exemplo, a restrição {\tt suspend:(X or Y), X = 0.}. Essa restrição tem a solução ${X=0,Y=1}$, mas, na forma como está escrita, a restrição
não nos permite encontrar essa solução ({\tt X or Y} estará suspenso até que Y receba um valor, o que não acontece, já que $or/2$
só lida com variáveis instanciadas). Existem, no entanto, no \eclipse, bibliotecas que lidam com a restrição de forma ``ativa''
(isto é, fazem uso de propagações). Veremos como lidar com duas delas momentaneamente, a {\it sd} e a {\it ic}.


\end{document}
